\documentclass{article}
\usepackage{amsmath} % Untuk lingkungan matriks dan perintah matematika lainnya
\usepackage{amsfonts} % Untuk simbol
\usepackage{amssymb} % Untuk simbol \mathbb{R}
\usepackage{geometry} % Untuk mengatur margin
\geometry{a4paper, margin=1in}

\title{Aljabar Linear: Basis dan Dimensi}
\author{Rafi Kamindra 2201006}
\date{} % Kosongkan tanggal jika tidak ingin ditampilkan

\newcommand{\vektor}[1]{\mathbf{#1}} % Perintah untuk vektor
\newcommand{\R}{\mathbb{R}} % Simbol R
\newcommand{\Poly}{\mathbb{P}} % Simbol P untuk polinom
\newcommand{\M}{\mathbb{M}} % Simbol M untuk matriks

\begin{document}
\maketitle
\pagenumbering{gobble} % Menghilangkan nomor halaman jika tidak diinginkan untuk halaman judul

\section*{Problem}

\subsection*{1. Periksa apakah $U = \{(2,1,3), (-1,3,-1), (3,-2,1)\}$ basis untuk $\R^3$.}
\textbf{Jawaban:}\\
Untuk menjadi basis $\R^3$, himpunan $U$ harus terdiri dari 3 vektor yang linear independen dan merentang $\R^3$. Karena ada 3 vektor, kita cukup memeriksa apakah mereka linear independen dengan menghitung determinan matriks yang dibentuk oleh vektor-vektor ini.
Misalkan $\vektor{u}_1 = (2,1,3)$, $\vektor{u}_2 = (-1,3,-1)$, $\vektor{u}_3 = (3,-2,1)$.
\[ A = \begin{pmatrix} 2 & -1 & 3 \\ 1 & 3 & -2 \\ 3 & -1 & 1 \end{pmatrix} \]
$\det(A) = 2 \begin{vmatrix} 3 & -2 \\ -1 & 1 \end{vmatrix} - (-1) \begin{vmatrix} 1 & -2 \\ 3 & 1 \end{vmatrix} + 3 \begin{vmatrix} 1 & 3 \\ 3 & -1 \end{vmatrix}$
$= 2(3 \cdot 1 - (-2) \cdot (-1)) + 1(1 \cdot 1 - (-2) \cdot 3) + 3(1 \cdot (-1) - 3 \cdot 3)$
$= 2(3 - 2) + 1(1 + 6) + 3(-1 - 9)$
$= 2(1) + 1(7) + 3(-10)$
$= 2 + 7 - 30 = -21$.
Karena $\det(A) = -21 \neq 0$, vektor-vektor tersebut linear independen dan oleh karena itu membentuk basis untuk $\R^3$.
Jadi, $U$ adalah basis untuk $\R^3$.

\subsection*{2. Periksa, apakah $U = \{1+2x-x^2, 2+x^2, -1+3x+2x^2\}$ basis untuk $\Poly_2$.}
\textbf{Jawaban:}\\
Ruang polinom $\Poly_2$ memiliki dimensi 3. Himpunan $U$ memiliki 3 polinom. Kita perlu memeriksa apakah polinom-polinom ini linear independen. Kita representasikan polinom sebagai vektor koordinat terhadap basis standar $\{1, x, x^2\}$.
$p_1(x) = 1+2x-x^2 \rightarrow \vektor{p}_1 = (1,2,-1)$
$p_2(x) = 2+x^2 \rightarrow \vektor{p}_2 = (2,0,1)$
$p_3(x) = -1+3x+2x^2 \rightarrow \vektor{p}_3 = (-1,3,2)$
Bentuk matriks dari vektor-vektor koordinat ini:
\[ A = \begin{pmatrix} 1 & 2 & -1 \\ 2 & 0 & 3 \\ -1 & 1 & 2 \end{pmatrix} \]
$\det(A) = 1 \begin{vmatrix} 0 & 3 \\ 1 & 2 \end{vmatrix} - 2 \begin{vmatrix} 2 & 3 \\ -1 & 2 \end{vmatrix} + (-1) \begin{vmatrix} 2 & 0 \\ -1 & 1 \end{vmatrix}$
$= 1(0 \cdot 2 - 3 \cdot 1) - 2(2 \cdot 2 - 3 \cdot (-1)) - 1(2 \cdot 1 - 0 \cdot (-1))$
$= 1(0 - 3) - 2(4 + 3) - 1(2 - 0)$
$= 1(-3) - 2(7) - 1(2)$
$= -3 - 14 - 2 = -19$.
Karena $\det(A) = -19 \neq 0$, polinom-polinom tersebut linear independen dan membentuk basis untuk $\Poly_2$.
Jadi, $U$ adalah basis untuk $\Poly_2$.

\subsection*{3. Jika $U = \{2+x-cx^2, 2+x^2, -1\}$ adalah basis untuk $\Poly_2$, berapa nilai $c$ yang mungkin?}
\textbf{Jawaban:}\\
Agar $U$ menjadi basis untuk $\Poly_2$, polinom-polinom dalam $U$ harus linear independen. Representasikan sebagai vektor koordinat terhadap basis $\{1, x, x^2\}$:
$p_1(x) = 2+x-cx^2 \rightarrow \vektor{p}_1 = (2,1,-c)$
$p_2(x) = 2+x^2 \rightarrow \vektor{p}_2 = (2,0,1)$
$p_3(x) = -1 \rightarrow \vektor{p}_3 = (-1,0,0)$
Bentuk matriks dari vektor-vektor koordinat ini:
\[ A = \begin{pmatrix} 2 & 2 & -1 \\ 1 & 0 & 0 \\ -c & 1 & 0 \end{pmatrix} \]
Agar linear independen, $\det(A) \neq 0$.
$\det(A) = 2 \begin{vmatrix} 0 & 0 \\ 1 & 0 \end{vmatrix} - 2 \begin{vmatrix} 1 & 0 \\ -c & 0 \end{vmatrix} + (-1) \begin{vmatrix} 1 & 0 \\ -c & 1 \end{vmatrix}$
$= 2(0 \cdot 0 - 0 \cdot 1) - 2(1 \cdot 0 - 0 \cdot (-c)) - 1(1 \cdot 1 - 0 \cdot (-c))$
$= 2(0) - 2(0) - 1(1)$
$= 0 - 0 - 1 = -1$.
Karena $\det(A) = -1 \neq 0$ untuk semua nilai $c$, maka himpunan $U$ selalu linear independen untuk semua nilai $c \in \R$.
Jadi, $U$ adalah basis untuk $\Poly_2$ untuk semua nilai $c \in \R$.

\subsection*{4. Diketahui $\vektor{u}, \vektor{v}, \vektor{w} \in \R^3$ semuanya tidak nol dan satu sama lain saling ortogonal. Buktikan bahwa $\vektor{u}, \vektor{v}, \vektor{w}$ bebas linear.}
\textbf{Jawaban:}\\
Untuk membuktikan bahwa $\vektor{u}, \vektor{v}, \vektor{w}$ bebas linear, kita harus menunjukkan bahwa satu-satunya solusi untuk persamaan $c_1\vektor{u} + c_2\vektor{v} + c_3\vektor{w} = \vektor{0}$ adalah $c_1=c_2=c_3=0$.
Ambil hasil kali titik dari persamaan $c_1\vektor{u} + c_2\vektor{v} + c_3\vektor{w} = \vektor{0}$ dengan $\vektor{u}$:
$\vektor{u} \cdot (c_1\vektor{u} + c_2\vektor{v} + c_3\vektor{w}) = \vektor{u} \cdot \vektor{0}$
$c_1(\vektor{u} \cdot \vektor{u}) + c_2(\vektor{u} \cdot \vektor{v}) + c_3(\vektor{u} \cdot \vektor{w}) = 0$.
Karena $\vektor{u}, \vektor{v}, \vektor{w}$ saling ortogonal, maka $\vektor{u} \cdot \vektor{v} = 0$ dan $\vektor{u} \cdot \vektor{w} = 0$.
Juga, karena $\vektor{u}$ tidak nol, $\vektor{u} \cdot \vektor{u} = ||\vektor{u}||^2 \neq 0$.
Maka persamaan menjadi $c_1||\vektor{u}||^2 + c_2(0) + c_3(0) = 0 \Rightarrow c_1||\vektor{u}||^2 = 0$.
Karena $||\vektor{u}||^2 \neq 0$, maka haruslah $c_1 = 0$.

Dengan cara yang sama, ambil hasil kali titik dari persamaan $c_1\vektor{u} + c_2\vektor{v} + c_3\vektor{w} = \vektor{0}$ dengan $\vektor{v}$:
$\vektor{v} \cdot (c_1\vektor{u} + c_2\vektor{v} + c_3\vektor{w}) = \vektor{v} \cdot \vektor{0}$
$c_1(\vektor{v} \cdot \vektor{u}) + c_2(\vektor{v} \cdot \vektor{v}) + c_3(\vektor{v} \cdot \vektor{w}) = 0$.
Karena $\vektor{v} \cdot \vektor{u} = 0$, $\vektor{v} \cdot \vektor{w} = 0$, dan $||\vektor{v}||^2 \neq 0$ (karena $\vektor{v}$ tidak nol), maka $c_2||\vektor{v}||^2 = 0 \Rightarrow c_2 = 0$.

Terakhir, ambil hasil kali titik dari persamaan $c_1\vektor{u} + c_2\vektor{v} + c_3\vektor{w} = \vektor{0}$ dengan $\vektor{w}$:
$\vektor{w} \cdot (c_1\vektor{u} + c_2\vektor{v} + c_3\vektor{w}) = \vektor{w} \cdot \vektor{0}$
$c_1(\vektor{w} \cdot \vektor{u}) + c_2(\vektor{w} \cdot \vektor{v}) + c_3(\vektor{w} \cdot \vektor{w}) = 0$.
Karena $\vektor{w} \cdot \vektor{u} = 0$, $\vektor{w} \cdot \vektor{v} = 0$, dan $||\vektor{w}||^2 \neq 0$ (karena $\vektor{w}$ tidak nol), maka $c_3||\vektor{w}||^2 = 0 \Rightarrow c_3 = 0$.
Karena satu-satunya solusi adalah $c_1=0, c_2=0, c_3=0$, maka $\vektor{u}, \vektor{v}, \vektor{w}$ bebas linear.

\subsection*{5. Berapakah dimensi ruang matriks $\M_{2\times2}$?}
\textbf{Jawaban:}\\
Ruang matriks $\M_{2\times2}$ adalah ruang dari semua matriks berukuran $2 \times 2$.
Sebuah matriks umum $A \in \M_{2\times2}$ dapat ditulis sebagai:
\[ A = \begin{pmatrix} a & b \\ c & d \end{pmatrix} \]
dimana $a, b, c, d \in \R$.
Matriks ini dapat ditulis sebagai kombinasi linear dari matriks-matriks basis:
\[ A = a \begin{pmatrix} 1 & 0 \\ 0 & 0 \end{pmatrix} + b \begin{pmatrix} 0 & 1 \\ 0 & 0 \end{pmatrix} + c \begin{pmatrix} 0 & 0 \\ 1 & 0 \end{pmatrix} + d \begin{pmatrix} 0 & 0 \\ 0 & 1 \end{pmatrix} \]
Himpunan matriks $\left\{ \begin{pmatrix} 1 & 0 \\ 0 & 0 \end{pmatrix}, \begin{pmatrix} 0 & 1 \\ 0 & 0 \end{pmatrix}, \begin{pmatrix} 0 & 0 \\ 1 & 0 \end{pmatrix}, \begin{pmatrix} 0 & 0 \\ 0 & 1 \end{pmatrix} \right\}$ adalah himpunan yang merentang $\M_{2\times2}$ dan juga linear independen. Oleh karena itu, himpunan ini adalah basis untuk $\M_{2\times2}$.
Jumlah vektor (matriks) dalam basis adalah 4.
Jadi, dimensi ruang matriks $\M_{2\times2}$ adalah 4.
Secara umum, dimensi $\M_{m \times n}$ adalah $m \times n$. Untuk $\M_{2\times2}$, dimensinya adalah $2 \times 2 = 4$.

\end{document}
