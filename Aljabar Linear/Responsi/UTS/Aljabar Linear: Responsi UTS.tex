\documentclass{article}
\usepackage{amsmath} % Untuk lingkungan matriks dan perintah matematika lainnya
\usepackage{amsfonts} % Untuk simbol
\usepackage{amssymb} % Untuk simbol \mathbb{R}
\usepackage{geometry} % Untuk mengatur margin
\geometry{a4paper, margin=1in}

\title{Aljabar Linear: Responsi UTS}
\author{Rafi Kamindra 2201006}
\date{} % Kosongkan tanggal jika tidak ingin ditampilkan

\newcommand{\vektor}[1]{\mathbf{#1}} % Perintah untuk vektor
\newcommand{\R}{\mathbb{R}} % Simbol R
\newcommand{\Poly}{\mathbb{P}} % Simbol P untuk polinom
\newcommand{\M}{\mathbb{M}} % Simbol M untuk matriks
\newcommand{\spanof}[1]{\text{span}\{#1\}} % Perintah untuk span
\newcommand{\coord}[2]{[\vektor{#1}]_{#2}} % Notasi koordinat

\begin{document}
\maketitle
\pagenumbering{gobble} % Menghilangkan nomor halaman jika tidak diinginkan untuk halaman judul

\section*{Problem}

\subsection*{1. Carilah basis ruang baris dan basis ruang kolom dari matriks: $A = \begin{pmatrix} 3 & -2 & 0 & -1 \\ -2 & 1 & 1 & 1 \\ 0 & 2 & 2 & 0 \end{pmatrix}$.}
\textbf{Jawaban:}\\
Lakukan Operasi Baris Elementer (OBE) pada $A$ untuk mendapatkan bentuk eselon baris.
\[ A = \begin{pmatrix} 3 & -2 & 0 & -1 \\ -2 & 1 & 1 & 1 \\ 0 & 2 & 2 & 0 \end{pmatrix} \]
$R_1 \leftrightarrow R_2$ (untuk mendapatkan -2 atau 1 di awal, lalu $R_2 \rightarrow -R_2$ jika perlu):
\[ \begin{pmatrix} -2 & 1 & 1 & 1 \\ 3 & -2 & 0 & -1 \\ 0 & 2 & 2 & 0 \end{pmatrix} \]
$R_1 \rightarrow -\frac{1}{2}R_1$:
\[ \begin{pmatrix} 1 & -1/2 & -1/2 & -1/2 \\ 3 & -2 & 0 & -1 \\ 0 & 2 & 2 & 0 \end{pmatrix} \]
$R_2 \rightarrow R_2 - 3R_1$:
\[ \begin{pmatrix} 1 & -1/2 & -1/2 & -1/2 \\ 0 & -2 - (-3/2) & 0 - (-3/2) & -1 - (-3/2) \\ 0 & 2 & 2 & 0 \end{pmatrix} = \begin{pmatrix} 1 & -1/2 & -1/2 & -1/2 \\ 0 & -4/2 + 3/2 & 3/2 & -2/2 + 3/2 \\ 0 & 2 & 2 & 0 \end{pmatrix} = \begin{pmatrix} 1 & -1/2 & -1/2 & -1/2 \\ 0 & -1/2 & 3/2 & 1/2 \\ 0 & 2 & 2 & 0 \end{pmatrix} \]
$R_2 \rightarrow -2R_2$:
\[ \begin{pmatrix} 1 & -1/2 & -1/2 & -1/2 \\ 0 & 1 & -3 & -1 \\ 0 & 2 & 2 & 0 \end{pmatrix} \]
$R_3 \rightarrow R_3 - 2R_2$:
\[ \begin{pmatrix} 1 & -1/2 & -1/2 & -1/2 \\ 0 & 1 & -3 & -1 \\ 0 & 0 & 2 - (-6) & 0 - (-2) \end{pmatrix} = \begin{pmatrix} 1 & -1/2 & -1/2 & -1/2 \\ 0 & 1 & -3 & -1 \\ 0 & 0 & 8 & 2 \end{pmatrix} \]
Baris-baris tak nol dalam bentuk eselon baris membentuk basis untuk ruang baris.
Basis ruang baris: $\{(1, -1/2, -1/2, -1/2), (0, 1, -3, -1), (0, 0, 8, 2)\}$.
Dimensi ruang baris adalah 3.
Kolom-kolom pada matriks eselon baris yang memiliki elemen pivot adalah kolom 1, 2, dan 3.
Maka, kolom-kolom yang bersesuaian pada matriks $A$ asli membentuk basis untuk ruang kolom.
Basis ruang kolom: $\{(3,-2,0), (-2,1,2), (0,1,2)\}$.
Dimensi ruang kolom juga 3.

\subsection*{2. Selidiki, apakah $K = \{(-1,2,0,1), (2,-3,1,1), (1,-1,1,1)\}$ bebas linier di $\R^4$.}
\textbf{Jawaban:}\\
Untuk menyelidiki kebebasan linear, kita bentuk matriks dengan vektor-vektor ini sebagai baris (atau kolom) dan periksa rank-nya. Jika rank sama dengan jumlah vektor (yaitu 3), maka himpunan tersebut bebas linear.
\[ M = \begin{pmatrix} -1 & 2 & 0 & 1 \\ 2 & -3 & 1 & 1 \\ 1 & -1 & 1 & 1 \end{pmatrix} \]
$R_2 \rightarrow R_2 + 2R_1$:
$R_3 \rightarrow R_3 + R_1$:
\[ \begin{pmatrix} -1 & 2 & 0 & 1 \\ 0 & 1 & 1 & 3 \\ 0 & 1 & 1 & 2 \end{pmatrix} \]
$R_3 \rightarrow R_3 - R_2$:
\[ \begin{pmatrix} -1 & 2 & 0 & 1 \\ 0 & 1 & 1 & 3 \\ 0 & 0 & 0 & -1 \end{pmatrix} \]
Matriks ini dalam bentuk eselon baris dan memiliki 3 baris tak nol. Rank matriks adalah 3.
Karena rank matriks (3) sama dengan jumlah vektor (3), maka himpunan $K$ bebas linier di $\R^4$.

\subsection*{3. Berikan contoh matriks $A, B \in \M_2$ dengan kondisi:}
\subsubsection*{a. $R_A = C_A$ dan $\text{rank}(A) \ge 1$.}
\textbf{Jawaban:}\\
$R_A$ adalah ruang baris $A$, $C_A$ adalah ruang kolom $A$.
Kondisi $R_A = C_A$ sering terjadi pada matriks simetris.
Misalkan $A = \begin{pmatrix} 1 & 0 \\ 0 & 1 \end{pmatrix}$ (matriks identitas).
Ruang baris $R_A = \spanof{(1,0), (0,1)} = \R^2$.
Ruang kolom $C_A = \spanof{(1,0)^T, (0,1)^T} = \R^2$.
Jadi $R_A = C_A = \R^2$.
$\text{rank}(A) = 2 \ge 1$.
Contoh lain: $A = \begin{pmatrix} 1 & 1 \\ 1 & 1 \end{pmatrix}$.
$R_A = \spanof{(1,1)}$.
$C_A = \spanof{(1,1)^T}$.
Dalam representasi vektor, $R_A$ dan $C_A$ keduanya adalah $\spanof{(1,1)}$.
$\text{rank}(A)=1 \ge 1$.

\subsubsection*{b. $R_B \neq C_B$ tetapi $\dim(R_B) = \dim(C_B)$.}
\textbf{Jawaban:}\\
Dimensi ruang baris selalu sama dengan dimensi ruang kolom (ini adalah rank matriks). Jadi, kondisi $\dim(R_B) = \dim(C_B)$ selalu terpenuhi. Kita perlu $R_B \neq C_B$.
Misalkan $B = \begin{pmatrix} 1 & 2 \\ 0 & 0 \end{pmatrix}$.
$R_B = \spanof{(1,2)}$. Ini adalah subruang dari $\R^2$.
$C_B = \spanof{(1,0)^T, (2,0)^T} = \spanof{(1,0)^T}$. Ini adalah subruang dari $\R^2$.
$\dim(R_B) = 1$, $\dim(C_B) = 1$.
Apakah $R_B = C_B$? Tidak. $R_B$ adalah himpunan kelipatan $(1,2)$, sedangkan $C_B$ adalah himpunan kelipatan $(1,0)$ (jika dilihat sebagai vektor baris).
Lebih tepatnya, $R_B = \{ (k, 2k) \mid k \in \R \}$.
$C_B = \{ (k, 0)^T \mid k \in \R \}$.
Jika kita membandingkan ruang baris (vektor baris) dengan ruang yang direntang oleh vektor kolom (yang juga vektor baris jika ditranspos), maka:
$R_B = \spanof{(1,2)}$.
Vektor kolom adalah $(1,0)^T$ dan $(2,0)^T$. Ruang yang direntang oleh kolom-kolom ini adalah $\spanof{(1,0)}$.
Karena $\spanof{(1,2)} \neq \spanof{(1,0)}$, maka $R_B \neq C_B$ (jika $C_B$ diinterpretasikan sebagai ruang yang direntang oleh transpos dari vektor kolom).
$\dim(R_B) = 1$ dan $\dim(C_B) = 1$.

\subsection*{4. Carilah semua $c$ sehingga $\vektor{x}=(0,-2,c,2)^T$ berada pada ruang kolom $B = \begin{pmatrix} 1 & -2 & 1 \\ 0 & 2 & -4 \\ 1 & 2 & 0 \\ 3 & 0 & -1 \end{pmatrix}$.}
\textbf{Jawaban:}\\
Vektor $\vektor{x}$ berada pada ruang kolom $B$ jika sistem $B\vektor{k} = \vektor{x}$ mempunyai solusi untuk $\vektor{k}=(k_1, k_2, k_3)^T$.
Bentuk matriks augmented $[B|\vektor{x}]$:
\[ \left[ \begin{array}{ccc|c} 1 & -2 & 1 & 0 \\ 0 & 2 & -4 & -2 \\ 1 & 2 & 0 & c \\ 3 & 0 & -1 & 2 \end{array} \right] \]
$R_3 \rightarrow R_3 - R_1$:
$R_4 \rightarrow R_4 - 3R_1$:
\[ \left[ \begin{array}{ccc|c} 1 & -2 & 1 & 0 \\ 0 & 2 & -4 & -2 \\ 0 & 4 & -1 & c \\ 0 & 6 & -4 & 2 \end{array} \right] \]
$R_2 \rightarrow \frac{1}{2}R_2$:
\[ \left[ \begin{array}{ccc|c} 1 & -2 & 1 & 0 \\ 0 & 1 & -2 & -1 \\ 0 & 4 & -1 & c \\ 0 & 6 & -4 & 2 \end{array} \right] \]
$R_3 \rightarrow R_3 - 4R_2$:
$R_4 \rightarrow R_4 - 6R_2$:
\[ \left[ \begin{array}{ccc|c} 1 & -2 & 1 & 0 \\ 0 & 1 & -2 & -1 \\ 0 & 0 & -1 - (-8) & c - (-4) \\ 0 & 0 & -4 - (-12) & 2 - (-6) \end{array} \right] = \left[ \begin{array}{ccc|c} 1 & -2 & 1 & 0 \\ 0 & 1 & -2 & -1 \\ 0 & 0 & 7 & c+4 \\ 0 & 0 & 8 & 8 \end{array} \right] \]
Dari baris ke-4: $8k_3 = 8 \Rightarrow k_3 = 1$ (jika kita menganggap ini adalah bentuk eselon, meskipun belum selesai).
Agar sistem konsisten, rank matriks koefisien harus sama dengan rank matriks augmented.
Perhatikan baris ke-3 dan ke-4:
$7k_3 = c+4$
$8k_3 = 8 \Rightarrow k_3 = 1$.
Substitusikan $k_3=1$ ke persamaan $7k_3 = c+4$:
$7(1) = c+4 \Rightarrow 7 = c+4 \Rightarrow c = 7-4 = 3$.
Jadi, agar $\vektor{x}$ berada pada ruang kolom $B$, nilai $c$ haruslah 3.

\subsection*{5. Carilah basis untuk $L = \{ (a,0,c) \mid a,c \in \R \}$.}
\textbf{Jawaban:}\\
Setiap vektor $\vektor{v} \in L$ berbentuk $(a,0,c)$.
Vektor ini dapat ditulis sebagai:
$\vektor{v} = (a,0,c) = a(1,0,0) + c(0,0,1)$.
Himpunan $\{(1,0,0), (0,0,1)\}$ merentang $L$.
Untuk memeriksa kebebasan linear, misalkan $k_1(1,0,0) + k_2(0,0,1) = (0,0,0)$.
$(k_1, 0, k_2) = (0,0,0)$.
Ini berarti $k_1=0$ dan $k_2=0$. Jadi, vektor-vektor tersebut linear independen.
Maka, basis untuk $L$ adalah $\{(1,0,0), (0,0,1)\}$.

\subsection*{6. Diketahui $\vektor{x}=(1,1,2)$, $\vektor{y}=(2,3,0)$. Carilah $\vektor{z} \in \R^3$ sehingga $\{\vektor{x}, \vektor{y}, \vektor{z}\}$ basis untuk $\R^3$.}
\textbf{Jawaban:}\\
Agar $\{\vektor{x}, \vektor{y}, \vektor{z}\}$ menjadi basis untuk $\R^3$, ketiga vektor tersebut harus linear independen.
$\vektor{x}$ dan $\vektor{y}$ sudah linear independen karena bukan kelipatan skalar satu sama lain.
Kita perlu memilih $\vektor{z}$ sehingga tidak terletak pada bidang yang direntang oleh $\vektor{x}$ dan $\vektor{y}$.
Cara mudah adalah memilih $\vektor{z}$ yang ortogonal terhadap bidang tersebut, misalnya $\vektor{z} = \vektor{x} \times \vektor{y}$.
$\vektor{z} = \vektor{x} \times \vektor{y} = \begin{vmatrix} \vektor{i} & \vektor{j} & \vektor{k} \\ 1 & 1 & 2 \\ 2 & 3 & 0 \end{vmatrix}$
$= \vektor{i}(1 \cdot 0 - 2 \cdot 3) - \vektor{j}(1 \cdot 0 - 2 \cdot 2) + \vektor{k}(1 \cdot 3 - 1 \cdot 2)$
$= \vektor{i}(0-6) - \vektor{j}(0-4) + \vektor{k}(3-2)$
$= -6\vektor{i} + 4\vektor{j} + 1\vektor{k} = (-6, 4, 1)$.
Jadi, salah satu contoh $\vektor{z}$ adalah $(-6,4,1)$.
Kita bisa periksa determinannya: $\begin{vmatrix} 1 & 2 & -6 \\ 1 & 3 & 4 \\ 2 & 0 & 1 \end{vmatrix} = 1(3-0) - 2(1-8) - 6(0-6) = 3 - 2(-7) - 6(-6) = 3 + 14 + 36 = 53 \neq 0$.
Jadi, $\{\vektor{x}, \vektor{y}, (-6,4,1)\}$ adalah basis untuk $\R^3$.
(Banyak pilihan lain, misalnya $\vektor{z}=(0,0,1)$ juga bisa jika $\det(\begin{pmatrix} \vektor{x} & \vektor{y} & (0,0,1) \end{pmatrix}) \neq 0$.
$\begin{vmatrix} 1 & 2 & 0 \\ 1 & 3 & 0 \\ 2 & 0 & 1 \end{vmatrix} = 1 \begin{vmatrix} 1 & 2 \\ 1 & 3 \end{vmatrix} = 1(3-2) = 1 \neq 0$. Jadi $\vektor{z}=(0,0,1)$ juga bisa.)

\subsection*{7. Diketahui $p=-1+x^2$, $q=1+x$. Carilah $r \in \Poly_2$ sehingga $\{p,q,r\}$ basis untuk $\Poly_2$.}
\textbf{Jawaban:}\\
Representasikan $p$ dan $q$ sebagai vektor koordinat terhadap basis standar $\{1,x,x^2\}$:
$p = -1+0x+1x^2 \rightarrow \vektor{v}_p = (-1,0,1)$
$q = 1+1x+0x^2 \rightarrow \vektor{v}_q = (1,1,0)$
Vektor $\vektor{v}_p$ dan $\vektor{v}_q$ linear independen. Kita perlu mencari vektor $\vektor{v}_r = (a,b,c)$ sehingga $\vektor{v}_p, \vektor{v}_q, \vektor{v}_r$ linear independen.
Artinya, $\det \begin{pmatrix} -1 & 1 & a \\ 0 & 1 & b \\ 1 & 0 & c \end{pmatrix} \neq 0$.
$\det = -1(c-0) - 1(0-b) + a(0-1) = -c + b - a$.
Kita perlu $-a+b-c \neq 0$.
Pilihan sederhana adalah memilih $\vektor{v}_r$ yang merupakan salah satu vektor basis standar yang belum "tercakup" oleh $\vektor{v}_p$ dan $\vektor{v}_q$.
Misalkan kita pilih $r=1$, sehingga $\vektor{v}_r = (1,0,0)$.
Maka determinannya menjadi $\begin{vmatrix} -1 & 1 & 1 \\ 0 & 1 & 0 \\ 1 & 0 & 0 \end{vmatrix} = -1(0-0) - 1(0-0) + 1(0-1) = -1 \neq 0$.
Jadi, $r=1$ adalah pilihan yang valid.
Maka $\{ -1+x^2, 1+x, 1 \}$ adalah basis untuk $\Poly_2$.
Pilihan lain, misal $r=x$, maka $\vektor{v}_r = (0,1,0)$.
$\begin{vmatrix} -1 & 1 & 0 \\ 0 & 1 & 1 \\ 1 & 0 & 0 \end{vmatrix} = -1(0-0) - 1(0-1) + 0 = 1 \neq 0$.
Jadi, $r=x$ juga pilihan yang valid.
Pilihan lain, misal $r=x^2$, maka $\vektor{v}_r = (0,0,1)$.
$\begin{vmatrix} -1 & 1 & 0 \\ 0 & 1 & 0 \\ 1 & 0 & 1 \end{vmatrix} = -1(1-0) - 1(0-0) + 0 = -1 \neq 0$.
Jadi, $r=x^2$ juga pilihan yang valid.

\subsection*{8. Selidiki kebenaran pernyataan berikut: "Jika $A, B \in \M_3$ dengan rank masing-masing 2 dan 3, maka $AB$ dapat memiliki rank 3".}
\textbf{Jawaban:}\\
Pernyataan ini \textbf{SALAH}.
Kita tahu bahwa $\text{rank}(AB) \le \min(\text{rank}(A), \text{rank}(B))$.
Diberikan $\text{rank}(A)=2$ dan $\text{rank}(B)=3$.
Maka $\text{rank}(AB) \le \min(2,3) = 2$.
Karena $\text{rank}(AB) \le 2$, maka $AB$ tidak mungkin memiliki rank 3.

\subsection*{9. Selidiki, apakah SPL $\begin{pmatrix} 1 & 2 & -1 \\ 2 & 4 & -2 \end{pmatrix} \begin{pmatrix} x_1 \\ x_2 \end{pmatrix} = \begin{pmatrix} a \\ b \end{pmatrix}$ selalu mempunyai solusi untuk semua $a,b \in \R$.}
(Catatan: Matriks koefisien $2 \times 3$ dan vektor $x$ adalah $2 \times 1$. Ini tidak konsisten. Asumsikan $x_3$ ada atau matriksnya $2 \times 2$).
Jika SPL adalah $\begin{pmatrix} 1 & 2 \\ 2 & 4 \end{pmatrix} \begin{pmatrix} x_1 \\ x_2 \end{pmatrix} = \begin{pmatrix} a \\ b \end{pmatrix}$:
\textbf{Jawaban:}\\
Bentuk matriks augmented: $\left[ \begin{array}{cc|c} 1 & 2 & a \\ 2 & 4 & b \end{array} \right]$.
$R_2 \rightarrow R_2 - 2R_1$:
$\left[ \begin{array}{cc|c} 1 & 2 & a \\ 0 & 0 & b-2a \end{array} \right]$.
Agar SPL mempunyai solusi, baris terakhir tidak boleh menghasilkan kontradiksi.
Ini berarti $0x_1 + 0x_2 = b-2a$ harus konsisten, yaitu $b-2a=0 \Rightarrow b=2a$.
Karena SPL hanya mempunyai solusi jika $b=2a$, maka SPL tidak selalu mempunyai solusi untuk semua $a,b \in \R$.
Misalnya, jika $a=1, b=3$, maka $b-2a = 3-2(1) = 1 \neq 0$, sehingga tidak ada solusi.

Jika SPL adalah $\begin{pmatrix} 1 & 2 & -1 \\ 2 & 4 & -2 \end{pmatrix} \begin{pmatrix} x_1 \\ x_2 \\ x_3 \end{pmatrix} = \begin{pmatrix} a \\ b \end{pmatrix}$:
\textbf{Jawaban:}\\
Bentuk matriks augmented: $\left[ \begin{array}{ccc|c} 1 & 2 & -1 & a \\ 2 & 4 & -2 & b \end{array} \right]$.
$R_2 \rightarrow R_2 - 2R_1$:
$\left[ \begin{array}{ccc|c} 1 & 2 & -1 & a \\ 0 & 0 & 0 & b-2a \end{array} \right]$.
Agar SPL mempunyai solusi, $b-2a=0 \Rightarrow b=2a$.
SPL tidak selalu mempunyai solusi untuk semua $a,b \in \R$.

\subsection*{10. Diketahui $B=\{\vektor{u}, \vektor{v}, \vektor{w}\}$ merupakan basis untuk $V$.}
\subsubsection*{a. Buktikan $B'=\{\vektor{u}, \vektor{u}+\vektor{v}, \vektor{u}+\vektor{v}+\vektor{w}\}$ juga basis untuk $V$.}
\textbf{Jawaban:}\\
Karena $B$ adalah basis untuk $V$ dan $\dim(V)=3$, kita cukup menunjukkan bahwa $B'$ linear independen.
Misalkan $c_1(\vektor{u}) + c_2(\vektor{u}+\vektor{v}) + c_3(\vektor{u}+\vektor{v}+\vektor{w}) = \vektor{0}$.
$(c_1+c_2+c_3)\vektor{u} + (c_2+c_3)\vektor{v} + c_3\vektor{w} = \vektor{0}$.
Karena $B=\{\vektor{u}, \vektor{v}, \vektor{w}\}$ adalah basis, maka $\vektor{u}, \vektor{v}, \vektor{w}$ linear independen.
Oleh karena itu, koefisien-koefisiennya harus nol:
$c_1+c_2+c_3 = 0 \quad (1)$
$c_2+c_3 = 0 \quad (2)$
$c_3 = 0 \quad (3)$
Dari (3), $c_3=0$.
Substitusikan $c_3=0$ ke (2): $c_2+0=0 \Rightarrow c_2=0$.
Substitusikan $c_2=0$ dan $c_3=0$ ke (1): $c_1+0+0=0 \Rightarrow c_1=0$.
Karena $c_1=c_2=c_3=0$ adalah satu-satunya solusi, maka $B'$ linear independen.
Karena $B'$ terdiri dari 3 vektor linear independen dalam ruang vektor $V$ berdimensi 3, maka $B'$ juga merupakan basis untuk $V$.

\subsubsection*{b. Tentukan matriks transisi dari $B$ ke $B'$.}
\textbf{Jawaban:}\\
Matriks transisi $P_{B \rightarrow B'}$ memiliki kolom-kolom yang merupakan koordinat vektor-vektor basis $B$ relatif terhadap basis $B'$.
Lebih mudah mencari $P_{B' \rightarrow B}$, yang kolom-kolomnya adalah koordinat vektor-vektor basis $B'$ relatif terhadap basis $B$.
Vektor basis $B'$:
$\vektor{u}'_1 = \vektor{u} = 1\vektor{u} + 0\vektor{v} + 0\vektor{w} \Rightarrow [\vektor{u}'_1]_B = (1,0,0)^T$.
$\vektor{u}'_2 = \vektor{u}+\vektor{v} = 1\vektor{u} + 1\vektor{v} + 0\vektor{w} \Rightarrow [\vektor{u}'_2]_B = (1,1,0)^T$.
$\vektor{u}'_3 = \vektor{u}+\vektor{v}+\vektor{w} = 1\vektor{u} + 1\vektor{v} + 1\vektor{w} \Rightarrow [\vektor{u}'_3]_B = (1,1,1)^T$.
Maka $P_{B' \rightarrow B} = \begin{pmatrix} 1 & 1 & 1 \\ 0 & 1 & 1 \\ 0 & 0 & 1 \end{pmatrix}$.
Matriks transisi dari $B$ ke $B'$ adalah $P_{B \rightarrow B'} = (P_{B' \rightarrow B})^{-1}$.
Kita hitung invers dari $P_{B' \rightarrow B}$. Ini adalah matriks segitiga atas, determinannya adalah $1 \cdot 1 \cdot 1 = 1$.
Untuk mencari inversnya:
$\left[ \begin{array}{ccc|ccc} 1 & 1 & 1 & 1 & 0 & 0 \\ 0 & 1 & 1 & 0 & 1 & 0 \\ 0 & 0 & 1 & 0 & 0 & 1 \end{array} \right]$
$R_1 \rightarrow R_1 - R_3$:
$R_2 \rightarrow R_2 - R_3$:
$\left[ \begin{array}{ccc|ccc} 1 & 1 & 0 & 1 & 0 & -1 \\ 0 & 1 & 0 & 0 & 1 & -1 \\ 0 & 0 & 1 & 0 & 0 & 1 \end{array} \right]$
$R_1 \rightarrow R_1 - R_2$:
$\left[ \begin{array}{ccc|ccc} 1 & 0 & 0 & 1 & -1 & 0 \\ 0 & 1 & 0 & 0 & 1 & -1 \\ 0 & 0 & 1 & 0 & 0 & 1 \end{array} \right]$
Jadi, $P_{B \rightarrow B'} = (P_{B' \rightarrow B})^{-1} = \begin{pmatrix} 1 & -1 & 0 \\ 0 & 1 & -1 \\ 0 & 0 & 1 \end{pmatrix}$.

\end{document}
