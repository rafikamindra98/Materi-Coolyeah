\documentclass{article}
\usepackage{amsmath}
\usepackage{amssymb}

\title{Aljabar Linear: Ujian Tengah Semester 2023}
\author{Rafi Kamindra 2201006}
\date{}

\begin{document}
\maketitle

\section*{Soal dan Jawaban Aljabar Linier}

Berikut adalah salinan soal dari gambar yang Anda berikan beserta jawaban dan penjelasannya.

\subsection*{Bagian A}

Nyatakan \textbf{BENAR} atau \textbf{SALAH} terhadap pernyataan berikut. Berikan alasan.

\subsubsection*{1. Himpunan $I = \{a + (2b)x + (a - b)x^2 \in P_2 \mid a, b \in \mathbb{R}\}$ adalah subruang yang berdimensi 3.}

\paragraph{Jawaban:} \textbf{SALAH.}

\paragraph{Alasan:}
Sebuah vektor (polinomial) $p(x)$ di dalam himpunan $I$ dapat ditulis sebagai:
\begin{align*}
p(x) &= a + 2bx + (a-b)x^2 \\
&= a + 2bx + ax^2 - bx^2 \\
&= a(1 + x^2) + b(2x - x^2)
\end{align*}

Dari bentuk di atas, kita dapat melihat bahwa setiap elemen di $I$ merupakan kombinasi linear dari dua polinomial, yaitu $\{1+x^2\}$ dan $\{2x - x^2\}$. Kedua polinomial ini saling bebas linear, karena tidak ada skalar $c$ sehingga $(1+x^2) = c(2x - x^2)$.

Oleh karena itu, himpunan $\{1+x^2, 2x - x^2\}$ adalah basis untuk subruang $I$. Karena basisnya terdiri dari 2 vektor, maka \textbf{dimensi dari subruang $I$ adalah 2}, bukan 3. Ruang polinom $P_2$ sendiri memiliki dimensi 3 dengan basis standar $\{1, x, x^2\}$.

\subsubsection*{2. Setiap matriks persegi $A = [a_{ij}]$ dengan $a_{ij} \neq 0$, untuk setiap $1 \le i, j \le n$ selalu mempunyai rank $n$.}

\paragraph{Jawaban:} \textbf{SALAH.}

\paragraph{Alasan:}
Rank sebuah matriks adalah jumlah baris (atau kolom) yang bebas linear. Meskipun semua entri matriks tidak nol, baris atau kolomnya bisa saja tidak bebas linear (bergantung linear).

Sebagai \textbf{contoh penyangkal (counterexample)}, ambil matriks persegi $2 \times 2$:
$$A = \begin{bmatrix} 1 & 1 \\ 1 & 1 \end{bmatrix}$$
Semua entri $a_{ij} \neq 0$. Namun, baris kedua adalah kelipatan dari baris pertama (Baris 2 = 1 $\times$ Baris 1). Ini berarti baris-barisnya tidak bebas linear.

Dengan melakukan Operasi Baris Elementer (OBE):
$$ \begin{bmatrix} 1 & 1 \\ 1 & 1 \end{bmatrix} \xrightarrow{R_2 - R_1} \begin{bmatrix} 1 & 1 \\ 0 & 0 \end{bmatrix} $$
Bentuk eselon baris dari matriks A hanya memiliki satu baris tak-nol, sehingga \textbf{rank(A) = 1}, yang mana tidak sama dengan $n=2$.

\subsection*{Bagian B}

\subsubsection*{3. Perhatikan matriks}
$$ A = \begin{bmatrix} 1 & -2 & 1 & -1 \\ -1 & 1 & 0 & 0 \\ 0 & -1 & 1 & -1 \end{bmatrix} $$

\paragraph{i. Berikan 2 buah basis berbeda untuk ruang kolom ($C_A$).}

\paragraph{Jawaban:}
Untuk menemukan basis ruang kolom, kita perlu mencari bentuk eselon baris dari matriks A untuk mengidentifikasi kolom-kolom pivot.
$$ \begin{bmatrix} 1 & -2 & 1 & -1 \\ -1 & 1 & 0 & 0 \\ 0 & -1 & 1 & -1 \end{bmatrix} \xrightarrow{R_2 + R_1} \begin{bmatrix} 1 & -2 & 1 & -1 \\ 0 & -1 & 1 & -1 \\ 0 & -1 & 1 & -1 \end{bmatrix} \xrightarrow{R_3 - R_2} \begin{bmatrix} 1 & -2 & 1 & -1 \\ 0 & -1 & 1 & -1 \\ 0 & 0 & 0 & 0 \end{bmatrix} $$
Kolom pivot berada di kolom 1 dan kolom 2. Maka, basis untuk ruang kolom $C_A$ dapat diambil dari kolom 1 dan kolom 2 pada matriks \textbf{A} asli.

\paragraph{Basis 1:}
$$ B_1 = \left\{ \begin{bmatrix} 1 \\ -1 \\ 0 \end{bmatrix}, \begin{bmatrix} -2 \\ 1 \\ -1 \end{bmatrix} \right\} $$

Untuk mencari basis kedua yang berbeda, kita bisa membuat kombinasi linear dari vektor-vektor pada basis pertama. Misalkan $v_1 = \begin{bmatrix} 1 \\ -1 \\ 0 \end{bmatrix}$ dan $v_2 = \begin{bmatrix} -2 \\ 1 \\ -1 \end{bmatrix}$.
\begin{itemize}
    \item Vektor baru 1: $v'_1 = v_1 + v_2 = \begin{bmatrix} 1 \\ -1 \\ 0 \end{bmatrix} + \begin{bmatrix} -2 \\ 1 \\ -1 \end{bmatrix} = \begin{bmatrix} -1 \\ 0 \\ -1 \end{bmatrix}$
    \item Vektor baru 2: $v'_2 = v_1 - v_2 = \begin{bmatrix} 1 \\ -1 \\ 0 \end{bmatrix} - \begin{bmatrix} -2 \\ 1 \\ -1 \end{bmatrix} = \begin{bmatrix} 3 \\ -2 \\ 1 \end{bmatrix}$
\end{itemize}
Kedua vektor baru ini ($v'_1$ dan $v'_2$) juga bebas linear dan merentang ruang kolom yang sama.

\paragraph{Basis 2:}
$$ B_2 = \left\{ \begin{bmatrix} -1 \\ 0 \\ -1 \end{bmatrix}, \begin{bmatrix} 3 \\ -2 \\ 1 \end{bmatrix} \right\} $$

\paragraph{ii. Tentukan semua $r$ sehingga $u = (0, -2, 2, r)$ berada di ruang baris matriks A.}

\paragraph{Jawaban:}
Sebuah vektor berada di ruang baris jika ia dapat dinyatakan sebagai kombinasi linear dari vektor-vektor basis ruang baris. Basis untuk ruang baris dapat diambil dari baris-baris tak-nol pada bentuk eselon baris matriks A.

Dari perhitungan sebelumnya, basis untuk ruang baris A adalah: \\
$w_1 = (1, -2, 1, -1)$ \\
$w_2 = (0, -1, 1, -1)$

Kita mencari skalar $c_1$ dan $c_2$ sehingga:
\begin{align*}
u &= c_1 w_1 + c_2 w_2 \\
(0, -2, 2, r) &= c_1(1, -2, 1, -1) + c_2(0, -1, 1, -1) \\
(0, -2, 2, r) &= (c_1, -2c_1 - c_2, c_1 + c_2, -c_1 - c_2)
\end{align*}

Dengan menyamakan komponen-komponen vektor, kita mendapatkan sistem persamaan linear:
\begin{enumerate}
    \item $c_1 = 0$
    \item $-2c_1 - c_2 = -2$
    \item $c_1 + c_2 = 2$
    \item $-c_1 - c_2 = r$
\end{enumerate}

Substitusi nilai $c_1=0$ dari persamaan (1) ke persamaan (2):
$-2(0) - c_2 = -2 \implies c_2 = 2$.

Kita verifikasi dengan persamaan (3):
$c_1 + c_2 = 0 + 2 = 2$. (Sesuai)

Terakhir, kita cari nilai $r$ menggunakan persamaan (4):
$r = -c_1 - c_2 = -(0) - (2) = -2$.

Jadi, nilai $r$ yang memenuhi adalah \textbf{-2}.

\end{document}
