\documentclass[12pt, a4paper]{article}
\usepackage[utf8]{inputenc}
\usepackage[T1]{fontenc}
\usepackage{amsmath}
\usepackage{amssymb}
\usepackage{geometry}
\geometry{a4paper, margin=1in}

\title{Aljabar Linear Elementer: Ujian Akhir Semester 2023}
\author{Rafi Kamindra 2201006}
\date{}

\begin{document}
\maketitle

% ------------------------------------------------------------------
% Soal 1
% ------------------------------------------------------------------
\section*{Soal 1}
Diketahui transformasi $T: \mathbb{R}^3 \to P_1$, dengan
\[ T(a, b, c) = (a - b + c) + (2a - b + c)x. \]

\subsection*{a. Carilah matriks transformasi untuk T terhadap basis baku.}
\textbf{Penyelesaian:} \\
Basis baku untuk domain $\mathbb{R}^3$ adalah $B = \{\mathbf{e}_1, \mathbf{e}_2, \mathbf{e}_3\}$, di mana:
\[ \mathbf{e}_1 = (1, 0, 0), \quad \mathbf{e}_2 = (0, 1, 0), \quad \mathbf{e}_3 = (0, 0, 1) \]
Basis baku untuk kodomain $P_1$ (polinom derajat $\le 1$) adalah $C = \{1, x\}$.

Kita akan memetakan setiap vektor basis dari $B$ menggunakan transformasi $T$ dan menyatakan hasilnya dalam koordinat terhadap basis $C$.

\begin{enumerate}
    \item Untuk $\mathbf{e}_1 = (1, 0, 0)$:
    \[ T(\mathbf{e}_1) = T(1, 0, 0) = (1 - 0 + 0) + (2(1) - 0 + 0)x = 1 + 2x \]
    Vektor koordinatnya terhadap basis $C$ adalah $[T(\mathbf{e}_1)]_C = \begin{pmatrix} 1 \\ 2 \end{pmatrix}$.

    \item Untuk $\mathbf{e}_2 = (0, 1, 0)$:
    \[ T(\mathbf{e}_2) = T(0, 1, 0) = (0 - 1 + 0) + (2(0) - 1 + 0)x = -1 - x \]
    Vektor koordinatnya terhadap basis $C$ adalah $[T(\mathbf{e}_2)]_C = \begin{pmatrix} -1 \\ -1 \end{pmatrix}$.

    \item Untuk $\mathbf{e}_3 = (0, 0, 1)$:
    \[ T(\mathbf{e}_3) = T(0, 0, 1) = (0 - 0 + 1) + (2(0) - 0 + 1)x = 1 + 1x \]
    Vektor koordinatnya terhadap basis $C$ adalah $[T(\mathbf{e}_3)]_C = \begin{pmatrix} 1 \\ 1 \end{pmatrix}$.
\end{enumerate}

Matriks transformasi $[T]_{C,B}$ dibentuk dengan menjadikan vektor-vektor koordinat ini sebagai kolom-kolomnya.
\[ [T] = \begin{pmatrix} 1 & -1 & 1 \\ 2 & -1 & 1 \end{pmatrix} \]

\subsection*{b. Apa kesimpulan saudara tentang nulitas dan rank (T)?}
\textbf{Penyelesaian:} \\
Rank dari $T$ adalah dimensi dari ruang kolom matriks $[T]$. Nulitas dari $T$ adalah dimensi dari ruang nol (kernel) dari $T$. Kita akan menggunakan eliminasi Gauss pada matriks $[T]$ untuk menemukan rank-nya.
\[ \begin{pmatrix} 1 & -1 & 1 \\ 2 & -1 & 1 \end{pmatrix} \xrightarrow{R_2 \to R_2 - 2R_1} \begin{pmatrix} 1 & -1 & 1 \\ 0 & 1 & -1 \end{pmatrix} \]
Matriks hasil reduksi baris memiliki dua pivot (entri utama tak-nol di setiap baris). Jadi, rank dari matriks adalah 2.
\[ \text{rank}(T) = 2 \]
Menurut Teorema Rank-Nulitas, untuk sebuah transformasi linear $T: V \to W$:
\[ \text{rank}(T) + \text{nulitas}(T) = \dim(V) \]
Dalam kasus ini, domain $V = \mathbb{R}^3$, sehingga $\dim(V) = 3$.
\[ 2 + \text{nulitas}(T) = 3 \]
\[ \text{nulitas}(T) = 3 - 2 = 1 \]
\textbf{Kesimpulan:} Transformasi $T$ memiliki $\text{rank}(T) = 2$ dan $\text{nulitas}(T) = 1$. Ini berarti bahwa citra (image) dari $T$ adalah sebuah ruang berdimensi 2 (yang mencakup seluruh $P_1$), dan kernel dari $T$ adalah sebuah ruang berdimensi 1 (sebuah garis yang melalui titik asal di $\mathbb{R}^3$).

\newpage

% ------------------------------------------------------------------
% Soal 2
% ------------------------------------------------------------------
\section*{Soal 2}
Selidiki, apakah matriks
\[ A = \begin{pmatrix} 0 & a & a \\ a & 0 & a \\ a & a & 0 \end{pmatrix} \]
terdiagonalkan untuk setiap bilangan real $a$?

\textbf{Penyelesaian:} \\
Sebuah matriks dapat didiagonalkan jika dan hanya jika ia simetris. Matriks $A$ dikatakan simetris jika $A = A^T$.
Mari kita cari transpos dari matriks $A$:
\[ A^T = \begin{pmatrix} 0 & a & a \\ a & 0 & a \\ a & a & 0 \end{pmatrix}^T = \begin{pmatrix} 0 & a & a \\ a & 0 & a \\ a & a & 0 \end{pmatrix} \]
Karena $A = A^T$, matriks $A$ adalah matriks simetris.

Menurut teorema spektral, setiap matriks simetris dengan entri bilangan real dapat didiagonalkan secara ortogonal.

\textbf{Kesimpulan:} Ya, matriks $A$ dapat didiagonalkan untuk setiap bilangan real $a$, karena matriks tersebut adalah matriks simetris.

\vspace{1cm}

% ------------------------------------------------------------------
% Soal 3
% ------------------------------------------------------------------
\section*{Soal 3}
Perhatikan ruang $P_2 = \{f = a + bx + cx^2 \mid a,b,c \in \mathbb{R}\}$ yang dilengkapi dengan hasil kali dalam $\langle f,g \rangle = \int_{0}^{1} f(t)g(t) \,dt$. Selidiki, apakah ada $m, n$ sehingga $u = m + nx$ ortogonal terhadap $v = x^2$?

\textbf{Penyelesaian:} \\
Dua vektor (dalam hal ini, polinom) $u$ dan $v$ dikatakan ortogonal jika hasil kali dalamnya sama dengan nol, yaitu $\langle u,v \rangle = 0$.

Kita akan menghitung hasil kali dalam antara $u(t) = m + nt$ dan $v(t) = t^2$.
\begin{align*}
    \langle u, v \rangle &= \int_{0}^{1} u(t)v(t) \,dt \\
    &= \int_{0}^{1} (m + nt)(t^2) \,dt \\
    &= \int_{0}^{1} (mt^2 + nt^3) \,dt \\
    &= \left[ \frac{m t^3}{3} + \frac{n t^4}{4} \right]_{0}^{1} \\
    &= \left( \frac{m(1)^3}{3} + \frac{n(1)^4}{4} \right) - \left( \frac{m(0)^3}{3} + \frac{n(0)^4}{4} \right) \\
    &= \frac{m}{3} + \frac{n}{4}
\end{align*}
Agar $u$ dan $v$ ortogonal, hasil kali dalamnya harus nol:
\[ \frac{m}{3} + \frac{n}{4} = 0 \]
Kita dapat menyederhanakan persamaan ini dengan mengalikan kedua sisi dengan 12:
\[ 4m + 3n = 0 \]
Persamaan ini adalah persamaan linear dengan dua variabel, $m$ dan $n$. Persamaan ini memiliki tak hingga banyaknya solusi non-trivial. Kita bisa memilih nilai untuk salah satu variabel dan menyelesaikan untuk yang lain. Misalnya, jika kita nyatakan $n$ dalam $m$:
\[ n = -\frac{4}{3}m \]
\textbf{Kesimpulan:} Ya, ada nilai $m$ dan $n$ sehingga $u$ ortogonal terhadap $v$. Bahkan, ada tak hingga banyaknya pasangan $(m, n)$ yang memenuhi syarat tersebut, selama mereka memenuhi hubungan $4m + 3n = 0$.

Sebagai contoh, jika kita memilih $m=3$, maka $n = -\frac{4}{3}(3) = -4$. Jadi, polinom $u = 3 - 4x$ adalah ortogonal terhadap $v = x^2$ dalam ruang hasil kali dalam ini.

\end{document}
