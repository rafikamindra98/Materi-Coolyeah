\documentclass{article}
\usepackage[a4paper, margin=2cm]{geometry}
\usepackage{amsmath}
\usepackage{amssymb}
\usepackage{amsfonts}
\usepackage[bahasa]{babel}

\title{Aljabar Linear: Ujian Akhir Semester 2023}
\author{Rafi Kamindra 2201006}
\date{}

\begin{document}
\maketitle

\section*{Soal 1}
Diketahui transformasi $T: \mathbb{R}^3 \to \mathbb{R}^2$, dengan
$T(x, y, z) = (x - y - z, -x + y + rz)$.

\subsection*{a. Carilah matriks transformasi untuk T.}
\textbf{Penyelesaian:}
Untuk mencari matriks standar untuk transformasi linear $T$, kita menerapkan $T$ pada vektor-vektor basis standar dari $\mathbb{R}^3$, yaitu $e_1 = (1,0,0)$, $e_2 = (0,1,0)$, dan $e_3 = (0,0,1)$. Hasilnya akan menjadi kolom-kolom dari matriks transformasi $[T]$.

\begin{align*}
    T(e_1) = T(1,0,0) &= (1-0-0, -1+0+r(0)) = (1, -1) \\
    T(e_2) = T(0,1,0) &= (0-1-0, -0+1+r(0)) = (-1, 1) \\
    T(e_3) = T(0,0,1) &= (0-0-1, -0+0+r(1)) = (-1, r)
\end{align*}

Jadi, matriks transformasi $[T]$ adalah:
\[
[T] = 
\begin{pmatrix}
1 & -1 & -1 \\
-1 & 1 & r
\end{pmatrix}
\]

\subsection*{b. Selidiki, apakah nilai r berpengaruh terhadap rank(T)?}
\textbf{Penyelesaian:}
Untuk menentukan rank dari matriks $[T]$, kita akan mengubahnya menjadi bentuk eselon baris melalui Operasi Baris Elementer (OBE).
\[
[T] = 
\begin{pmatrix}
1 & -1 & -1 \\
-1 & 1 & r
\end{pmatrix}
\]
Tambahkan baris 1 ke baris 2 ($R_2 \to R_2 + R_1$):
\[
\xrightarrow{R_2 + R_1}
\begin{pmatrix}
1 & -1 & -1 \\
0 & 0 & r-1
\end{pmatrix}
\]
Sekarang kita analisis hasilnya:
\begin{itemize}
    \item \textbf{Kasus 1: Jika $r-1 \neq 0$ (atau $r \neq 1$)} \\
    Matriks akan memiliki dua pivot (yaitu 1 dan $r-1$). Jumlah pivot sama dengan rank matriks.
    Maka, $rank(T) = 2$.
    
    \item \textbf{Kasus 2: Jika $r-1 = 0$ (atau $r = 1$)} \\
    Matriks menjadi $\begin{pmatrix} 1 & -1 & -1 \\ 0 & 0 & 0 \end{pmatrix}$.
    Matriks ini hanya memiliki satu pivot.
    Maka, $rank(T) = 1$.
\end{itemize}
\textbf{Kesimpulan:} Ya, nilai $r$ berpengaruh terhadap $rank(T)$. Ranknya adalah 2 jika $r \neq 1$, dan ranknya adalah 1 jika $r = 1$.

\section*{Soal 2}
Selidiki, apakah matriks 
$ A = \begin{pmatrix} 0 & 3 & 3 \\ 3 & 0 & 3 \\ 3 & 3 & 0 \end{pmatrix} $
terdiagonalkan?

\textbf{Penyelesaian:}
Sebuah matriks dapat didiagonalkan jika dan hanya jika untuk setiap nilai eigen, multiplisitas aljabar sama dengan multiplisitas geometrisnya.
Namun, ada teorema yang menyatakan bahwa \textbf{setiap matriks simetris dengan entri bilangan riil dapat didiagonalkan}.

Kita periksa apakah matriks $A$ simetris. Matriks $A$ simetris jika $A = A^T$.
\[
A^T = 
\begin{pmatrix}
0 & 3 & 3 \\
3 & 0 & 3 \\
3 & 3 & 0
\end{pmatrix}^T
=
\begin{pmatrix}
0 & 3 & 3 \\
3 & 0 & 3 \\
3 & 3 & 0
\end{pmatrix} = A
\]
Karena $A = A^T$, maka $A$ adalah matriks simetris.

\textbf{Kesimpulan:} Berdasarkan teorema, karena $A$ adalah matriks simetris, maka matriks $A$ \textbf{dapat didiagonalkan}.

\section*{Soal 3}
Perhatikan ruang $P_2 = \{f = a + bx + cx^2 \mid a, b, c \in \mathbb{R}\}$ yang dilengkapi dengan hasil kali dalam $\langle f, g \rangle = \int_{-1}^{1} f(t)g(t) \,dt$. Carilah semua vektor yang ortogonal terhadap $v = x^2 + 1$.

\textbf{Penyelesaian:}
Kita mencari semua vektor $w = a + bx + cx^2$ di $P_2$ sedemikian sehingga $\langle w, v \rangle = 0$.
\[
\langle w, v \rangle = \int_{-1}^{1} (a + bt + ct^2)(1 + t^2) \,dt = 0
\]
Kita ekspansi integran:
\[
(a + bt + ct^2)(1 + t^2) = a + at^2 + bt + bt^3 + ct^2 + ct^4 = ct^4 + bt^3 + (a+c)t^2 + bt + a
\]
Sekarang kita integralkan:
\[
\int_{-1}^{1} (ct^4 + bt^3 + (a+c)t^2 + bt + a) \,dt
\]
Kita dapat memisahkan integral berdasarkan suku genap dan ganjil. Integral dari fungsi ganjil ($t^3, t$) dari $-1$ ke $1$ adalah nol.
\[
= \int_{-1}^{1} (ct^4 + (a+c)t^2 + a) \,dt + \int_{-1}^{1} (bt^3 + bt) \,dt
\]
\[
= \left[ \frac{c}{5}t^5 + \frac{a+c}{3}t^3 + at \right]_{-1}^{1} + 0
\]
\[
= \left( \frac{c}{5} + \frac{a+c}{3} + a \right) - \left( -\frac{c}{5} - \frac{a+c}{3} - a \right)
\]
\[
= 2 \left( \frac{c}{5} + \frac{a+c}{3} + a \right)
\]
Kita atur agar hasil kali dalamnya sama dengan nol:
\begin{align*}
    2 \left( \frac{c}{5} + \frac{a+c}{3} + a \right) &= 0 \\
    \frac{c}{5} + \frac{a}{3} + \frac{c}{3} + a &= 0 \\
\end{align*}
Kalikan kedua sisi dengan 15 untuk menghilangkan penyebut:
\begin{align*}
    15 \left( \frac{c}{5} + \frac{a}{3} + \frac{c}{3} + a \right) &= 0 \\
    3c + 5a + 5c + 15a &= 0 \\
    20a + 8c &= 0 \\
    5a + 2c &= 0 \\
    c &= -\frac{5}{2}a
\end{align*}
Koefisien $b$ tidak muncul dalam persamaan, yang berarti $b$ bisa berupa bilangan riil apa pun.
Jadi, semua vektor $w$ yang ortogonal terhadap $v$ memiliki bentuk:
\[
w(x) = a + bx + \left(-\frac{5}{2}a\right)x^2
\]
dimana $a, b \in \mathbb{R}$.

\textbf{Kesimpulan:} Himpunan semua vektor yang ortogonal terhadap $v = x^2+1$ adalah himpunan polinomial berbentuk $w(x) = a(1 - \frac{5}{2}x^2) + bx$ untuk sembarang skalar $a$ dan $b$. Ini adalah subruang dari $P_2$ yang direntang oleh $\{1 - \frac{5}{2}x^2, x\}$.

\end{document}
