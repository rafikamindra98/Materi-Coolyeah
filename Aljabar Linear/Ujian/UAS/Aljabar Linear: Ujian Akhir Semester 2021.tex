\documentclass[12pt]{article}
\usepackage{amsmath}
\usepackage{amsfonts}
\usepackage{amssymb}
\usepackage{geometry}
\geometry{a4paper, margin=1in}

% --- Definisi Perintah Kustom ---
\newcommand{\R}{\mathbb{R}} % Simbol untuk himpunan bilangan real
\newcommand{\vektor}[1]{\mathbf{#1}} % Cetak tebal untuk vektor
\DeclareMathOperator{\Ker}{Ker} % Operator Kernel
\DeclareMathOperator{\proj}{proj} % Operator Proyeksi

\title{Aljabar Linear: Ujian Akhir Semester 2021}
\author{Rafi Kamindra 2201006}
\date{}

\begin{document}
\maketitle

\section*{Bagian A: Pernyataan BENAR atau SALAH}

\begin{enumerate}
    \item \textbf{Pernyataan:} Kernel transformasi linier $T: \R^2 \to \R^3$ dengan $T(a, b) = (a-2b, 0, -a+2b)$ adalah $\{ \vektor{0} \}$.

    \vspace{0.5cm}
    \textbf{Jawaban: SALAH}

    \textbf{Alasan:}
    Kernel dari $T$, dinotasikan $\Ker(T)$, adalah himpunan semua vektor $\vektor{v} = (a, b) \in \R^2$ sedemikian sehingga $T(\vektor{v}) = \vektor{0}$.
    $$ T(a,b) = (a-2b, 0, -a+2b) = (0,0,0) $$
    Ini memberikan kita sistem persamaan:
    \begin{align*}
        a - 2b &= 0 \\
        -a + 2b &= 0
    \end{align*}
    Dari persamaan pertama, kita dapatkan $a = 2b$. Persamaan kedua, $- (2b) + 2b = 0$, juga konsisten.
    Ini berarti setiap vektor dalam kernel harus memenuhi kondisi $a = 2b$. Kita dapat menulis vektor $(a,b)$ sebagai $(2b, b) = b(2,1)$.
    Jadi, kernelnya adalah:
    $$ \Ker(T) = \{ b(2,1) \mid b \in \R \} = \text{span}\left\{ \begin{pmatrix} 2 \\ 1 \end{pmatrix} \right\} $$
    Karena kernel mengandung vektor-vektor tak-nol (misalnya, untuk $b=1$, kita punya vektor $(2,1)$), maka kernelnya bukan hanya himpunan yang berisi vektor nol.

    \item \textbf{Pernyataan:} Mengacu pada hasil kali titik, untuk sembarang vektor $\vektor{x}, \vektor{y} \in \R^3$ di mana dua-duanya bukan vektor nol, vektor proyeksi $\proj_{\vektor{y}}\vektor{x}$ pasti bukan vektor nol.

    \vspace{0.5cm}
    \textbf{Jawaban: SALAH}

    \textbf{Alasan:}
    Rumus untuk proyeksi ortogonal vektor $\vektor{x}$ ke vektor $\vektor{y}$ adalah:
    $$ \proj_{\vektor{y}}\vektor{x} = \left( \frac{\vektor{x} \cdot \vektor{y}}{\vektor{y} \cdot \vektor{y}} \right) \vektor{y} $$
    Karena $\vektor{y}$ bukan vektor nol, maka $\vektor{y} \cdot \vektor{y} = \| \vektor{y} \|^2 \neq 0$.
    Vektor proyeksi akan menjadi vektor nol jika dan hanya jika skalar $\frac{\vektor{x} \cdot \vektor{y}}{\vektor{y} \cdot \vektor{y}}$ bernilai nol. Hal ini terjadi ketika pembilangnya, yaitu hasil kali titik $\vektor{x} \cdot \vektor{y}$, sama dengan nol.
    
    Dua vektor tak-nol dapat memiliki hasil kali titik nol jika keduanya saling ortogonal (tegak lurus).
    Sebagai contoh, ambil $\vektor{x} = (1, 0, 0)$ dan $\vektor{y} = (0, 1, 0)$. Keduanya adalah vektor tak-nol di $\R^3$.
    Hasil kali titiknya adalah:
    $$ \vektor{x} \cdot \vektor{y} = (1)(0) + (0)(1) + (0)(0) = 0 $$
    Maka, vektor proyeksinya adalah:
    $$ \proj_{\vektor{y}}\vektor{x} = \left( \frac{0}{\| \vektor{y} \|^2} \right) \vektor{y} = 0 \cdot \vektor{y} = \vektor{0} $$
    Jadi, mungkin bagi vektor proyeksi untuk menjadi vektor nol meskipun $\vektor{x}$ dan $\vektor{y}$ keduanya bukan vektor nol.
\end{enumerate}

\newpage

\section*{Bagian B: Analisis Matriks}
Perhatikan matriks
$$ F = \begin{bmatrix} 1 & -1 & 0 \\ 0 & 2 & c \\ 0 & 0 & 2 \end{bmatrix} $$

\begin{enumerate}
    \item[(i)] \textbf{Tentukan polinom karakteristik dan semua nilai eigen dari $F$.}
    
    Polinom karakteristik $p(\lambda)$ ditemukan dengan menghitung $\det(F - \lambda I)$:
    $$ F - \lambda I = \begin{bmatrix} 1-\lambda & -1 & 0 \\ 0 & 2-\lambda & c \\ 0 & 0 & 2-\lambda \end{bmatrix} $$
    Karena ini adalah matriks segitiga atas, determinannya adalah hasil kali entri-entri diagonalnya.
    $$ p(\lambda) = \det(F - \lambda I) = (1-\lambda)(2-\lambda)(2-\lambda) = (1-\lambda)(2-\lambda)^2 $$
    \textbf{Polinom karakteristiknya adalah $p(\lambda) = (1-\lambda)(2-\lambda)^2$.}

    Nilai-nilai eigen adalah akar-akar dari polinom karakteristik, yaitu saat $p(\lambda)=0$.
    $$ (1-\lambda)(2-\lambda)^2 = 0 $$
    \textbf{Nilai-nilai eigen adalah $\lambda_1 = 1$ (dengan multiplisitas aljabar 1) dan $\lambda_2 = 2$ (dengan multiplisitas aljabar 2).} Nilai eigen ini tidak bergantung pada nilai $c$.

    \item[(ii)] \textbf{Adakah vektor eigen yang tidak bergantung pada nilai $c$? Jelaskan.}

    \textbf{Jawaban: Ya, ada.}
    
    Kita cari ruang eigen untuk setiap nilai eigen.
    \begin{itemize}
        \item Untuk $\lambda_1 = 1$: Cari $\vektor{v}$ sehingga $(F-I)\vektor{v} = \vektor{0}$.
        $$ \begin{bmatrix} 0 & -1 & 0 \\ 0 & 1 & c \\ 0 & 0 & 1 \end{bmatrix} \begin{pmatrix} v_1 \\ v_2 \\ v_3 \end{pmatrix} = \begin{pmatrix} 0 \\ 0 \\ 0 \end{pmatrix} $$
        Sistem persamaannya: $-v_2=0 \implies v_2=0$, lalu $v_2+cv_3=0 \implies 0=0$, dan $v_3=0$.
        Jadi, $v_2=0$ dan $v_3=0$, sedangkan $v_1$ bebas. Vektor eigennya berbentuk $t(1,0,0)$.
        \textbf{Vektor eigen yang bersesuaian dengan $\lambda=1$, yaitu $t(1,0,0)$, tidak bergantung pada $c$.}

        \item Untuk $\lambda_2 = 2$: Cari $\vektor{v}$ sehingga $(F-2I)\vektor{v} = \vektor{0}$.
        $$ \begin{bmatrix} -1 & -1 & 0 \\ 0 & 0 & c \\ 0 & 0 & 0 \end{bmatrix} \begin{pmatrix} v_1 \\ v_2 \\ v_3 \end{pmatrix} = \begin{pmatrix} 0 \\ 0 \\ 0 \end{pmatrix} $$
        Sistem persamaannya: $-v_1-v_2=0 \implies v_1=-v_2$, dan $c v_3 = 0$.
        Solusi untuk $v_3$ bergantung pada $c$. Jika $c \neq 0$, maka $v_3=0$. Jika $c=0$, $v_3$ bebas.
        Karena solusinya bergantung pada $c$, maka vektor eigen yang bersesuaian dengan $\lambda=2$ juga bergantung pada $c$.
    \end{itemize}

    \item[(iii)] \textbf{Apakah ada $c$ yang membuat $F$ terdiagonalkan? Jelaskan.}
    
    \textbf{Jawaban: Ya, ada, yaitu ketika $c=0$.}

    \textbf{Alasan:}
    Sebuah matriks $n \times n$ dapat didiagonalkan jika dan hanya jika untuk setiap nilai eigen, multiplisitas geometris (dimensi ruang eigen) sama dengan multiplisitas aljabarnya.
    \begin{itemize}
        \item Untuk $\lambda_1 = 1$: Multiplisitas aljabar adalah 1. Ruang eigennya adalah $\text{span}\{(1,0,0)\}$, yang dimensinya 1. Jadi, multiplisitas geometris = multiplisitas aljabar.
        
        \item Untuk $\lambda_2 = 2$: Multiplisitas aljabar adalah 2. Kita perlu memeriksa multiplisitas geometrisnya, yaitu dimensi ruang solusi dari $(F-2I)\vektor{v}=\vektor{0}$.
        
        Dari (ii), kita punya sistem $-v_1-v_2=0$ dan $c v_3=0$.
        \begin{itemize}
            \item \textbf{Kasus 1: $c \neq 0$.} Persamaan $cv_3=0$ mengakibatkan $v_3=0$. Persamaan $-v_1-v_2=0$ berarti $v_1=-v_2$. Vektor eigen berbentuk $t(-1,1,0)$. Ruang eigennya adalah $\text{span}\{(-1,1,0)\}$, yang dimensinya 1.
            Di sini, multiplisitas geometris (1) $\neq$ multiplisitas aljabar (2). Jadi, $F$ \textbf{tidak} dapat didiagonalkan.
            
            \item \textbf{Kasus 2: $c = 0$.} Persamaan $cv_3=0$ menjadi $0 \cdot v_3=0$, yang benar untuk semua $v_3$. Jadi, $v_3$ adalah variabel bebas. Kita juga punya $v_1=-v_2$.
            Vektor solusinya dapat ditulis sebagai:
            $$ \begin{pmatrix} v_1 \\ v_2 \\ v_3 \end{pmatrix} = \begin{pmatrix} -v_2 \\ v_2 \\ v_3 \end{pmatrix} = v_2 \begin{pmatrix} -1 \\ 1 \\ 0 \end{pmatrix} + v_3 \begin{pmatrix} 0 \\ 0 \\ 1 \end{pmatrix} $$
            Ruang eigennya adalah $\text{span}\{(-1,1,0), (0,0,1)\}$, yang dimensinya 2.
            Di sini, multiplisitas geometris (2) = multiplisitas aljabar (2).
        \end{itemize}
    \end{itemize}
    Karena semua nilai eigen memiliki multiplisitas geometris yang sama dengan multiplisitas aljabarnya hanya ketika $c=0$, maka \textbf{$F$ dapat didiagonalkan jika dan hanya jika $c=0$.}
\end{enumerate}

\end{document}
