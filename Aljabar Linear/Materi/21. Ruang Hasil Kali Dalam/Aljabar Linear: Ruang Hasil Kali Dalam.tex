\documentclass{article}
\usepackage{amsmath} % Untuk lingkungan matriks dan perintah matematika lainnya
\usepackage{amsfonts} % Untuk simbol
\usepackage{amssymb} % Untuk simbol \mathbb{R}
\usepackage{geometry} % Untuk mengatur margin
\geometry{a4paper, margin=1in}
\usepackage{esvect}

\title{Aljabar Linear: Ruang Hasil Kali Dalam}
\author{Rafi Kamindra 2201006}
\date{} % Kosongkan tanggal jika tidak ingin ditampilkan

\newcommand{\R}{\mathbb{R}} % Simbol R
\newcommand{\Poly}{\mathbb{P}} % Simbol P untuk polinom
\newcommand{\innerprod}[2]{\langle #1, #2 \rangle} % Hasil kali dalam
\newcommand{\norm}[1]{\| #1 \|} % Norma
\newcommand{\spanof}[1]{\text{span}\{#1\}} % Perintah untuk span
\newcommand{\vektor}[1]{\mathbf{#1}} % Perintah untuk vektor tebal

\begin{document}
\maketitle
\pagenumbering{gobble} % Menghilangkan nomor halaman jika tidak diinginkan untuk halaman judul

\section*{Problem}
\textbf{$\Poly_2$ adalah ruang hasil kali dalam melalui $\innerprod{f}{g} = \int_{0}^{1} f(t)g(t)dt$.}

\subsection*{a. Hitung $\|-\frac{1}{2} - \frac{1}{2}x\|$.}
\textbf{Jawaban:}\\
Misalkan $h(x) = -\frac{1}{2} - \frac{1}{2}x$.
$\norm{h(x)}^2 = \innerprod{h(x)}{h(x)} = \int_{0}^{1} \left(-\frac{1}{2} - \frac{1}{2}t\right) \left(-\frac{1}{2} - \frac{1}{2}t\right) dt$
$= \int_{0}^{1} \left(\frac{1}{4} + \frac{1}{2}t + \frac{1}{4}t^2\right) dt$
$= \left[ \frac{1}{4}t + \frac{1}{2}\frac{t^2}{2} + \frac{1}{4}\frac{t^3}{3} \right]_{0}^{1}$
$= \left[ \frac{1}{4}t + \frac{1}{4}t^2 + \frac{1}{12}t^3 \right]_{0}^{1}$
$= \left(\frac{1}{4} + \frac{1}{4} + \frac{1}{12}\right) - (0)$
$= \frac{1}{2} + \frac{1}{12} = \frac{6}{12} + \frac{1}{12} = \frac{7}{12}$.
Maka, $\norm{-\frac{1}{2} - \frac{1}{2}x} = \sqrt{\frac{7}{12}} = \frac{\sqrt{7}}{\sqrt{12}} = \frac{\sqrt{7}}{2\sqrt{3}} = \frac{\sqrt{21}}{6}$.

\subsection*{b. Untuk $u=x$ dan $v=x^2$, perlihatkan bahwa $|\innerprod{u}{v}| \le \norm{u}\norm{v}$.}
\textbf{Jawaban:}\\
Ini adalah pertidaksamaan Cauchy-Schwarz.
$\innerprod{u}{v} = \innerprod{x}{x^2} = \int_{0}^{1} t \cdot t^2 dt = \int_{0}^{1} t^3 dt = \left[ \frac{t^4}{4} \right]_{0}^{1} = \frac{1}{4}$.
Jadi, $|\innerprod{u}{v}| = \frac{1}{4}$.

$\norm{u}^2 = \innerprod{x}{x} = \int_{0}^{1} t \cdot t dt = \int_{0}^{1} t^2 dt = \left[ \frac{t^3}{3} \right]_{0}^{1} = \frac{1}{3}$.
Maka $\norm{u} = \sqrt{\frac{1}{3}} = \frac{1}{\sqrt{3}}$.

$\norm{v}^2 = \innerprod{x^2}{x^2} = \int_{0}^{1} t^2 \cdot t^2 dt = \int_{0}^{1} t^4 dt = \left[ \frac{t^5}{5} \right]_{0}^{1} = \frac{1}{5}$.
Maka $\norm{v} = \sqrt{\frac{1}{5}} = \frac{1}{\sqrt{5}}$.

$\norm{u}\norm{v} = \frac{1}{\sqrt{3}} \cdot \frac{1}{\sqrt{5}} = \frac{1}{\sqrt{15}}$.
Kita perlu membandingkan $\frac{1}{4}$ dan $\frac{1}{\sqrt{15}}$.
$\frac{1}{4} = \frac{\sqrt{15}}{4\sqrt{15}}$ dan $\frac{1}{\sqrt{15}} = \frac{4}{4\sqrt{15}}$.
Karena $\sqrt{15} < \sqrt{16} = 4$, maka $\sqrt{15} < 4$.
Jadi, $\frac{\sqrt{15}}{4\sqrt{15}} < \frac{4}{4\sqrt{15}}$, yang berarti $\frac{1}{4} < \frac{1}{\sqrt{15}}$.
Sehingga $|\innerprod{u}{v}| \le \norm{u}\norm{v}$ terbukti ($\frac{1}{4} \le \frac{1}{\sqrt{15}}$).

\subsection*{c. Perlihatkan pula bahwa $\norm{u+v} \le \norm{u} + \norm{v}$.}
\textbf{Jawaban:}\\
Ini adalah pertidaksamaan segitiga.
$u+v = x+x^2$.
$\norm{u+v}^2 = \innerprod{x+x^2}{x+x^2} = \int_{0}^{1} (t+t^2)(t+t^2) dt = \int_{0}^{1} (t^2 + 2t^3 + t^4) dt$
$= \left[ \frac{t^3}{3} + 2\frac{t^4}{4} + \frac{t^5}{5} \right]_{0}^{1} = \left[ \frac{t^3}{3} + \frac{t^4}{2} + \frac{t^5}{5} \right]_{0}^{1}$
$= \frac{1}{3} + \frac{1}{2} + \frac{1}{5} = \frac{10+15+6}{30} = \frac{31}{30}$.
Maka $\norm{u+v} = \sqrt{\frac{31}{30}}$.

$\norm{u} + \norm{v} = \frac{1}{\sqrt{3}} + \frac{1}{\sqrt{5}} = \frac{\sqrt{5}+\sqrt{3}}{\sqrt{15}}$.
Kita perlu membandingkan $\sqrt{\frac{31}{30}}$ dengan $\frac{\sqrt{5}+\sqrt{3}}{\sqrt{15}}$.
Kuadratkan kedua sisi:
$\left(\sqrt{\frac{31}{30}}\right)^2 = \frac{31}{30} \approx 1.033$.
$\left(\frac{\sqrt{5}+\sqrt{3}}{\sqrt{15}}\right)^2 = \frac{(\sqrt{5}+\sqrt{3})^2}{15} = \frac{5 + 2\sqrt{15} + 3}{15} = \frac{8 + 2\sqrt{15}}{15}$.
$2\sqrt{15} \approx 2 \times 3.87 = 7.74$.
Maka $\frac{8 + 7.74}{15} = \frac{15.74}{15} \approx 1.049$.
Karena $1.033 \le 1.049$, maka $\norm{u+v} \le \norm{u} + \norm{v}$ terbukti.
(Lebih formal: $\frac{31}{30} \le \frac{8+2\sqrt{15}}{15} \Leftrightarrow 31 \cdot 15 \le 30(8+2\sqrt{15}) \Leftrightarrow 31 \le 2(8+2\sqrt{15}) \Leftrightarrow 31 \le 16+4\sqrt{15} \Leftrightarrow 15 \le 4\sqrt{15} \Leftrightarrow 225 \le 16 \cdot 15 = 240$. Pernyataan $225 \le 240$ benar.)

\section*{Problem}
\textbf{Perhatikan ruang $\Poly_2 = \{f=a+bx+cx^2 \mid a,b,c \in \R\}$ yang dilengkapi dengan hasil kali dalam $\innerprod{f}{g} = \int_{0}^{1} f(t)g(t)dt$.}

\subsection*{i. Periksa, apakah $u=1-x$ ortogonal terhadap $v=x$.}
\textbf{Jawaban:}\\
$\innerprod{u}{v} = \int_{0}^{1} (1-t)(t) dt = \int_{0}^{1} (t-t^2) dt = \left[ \frac{t^2}{2} - \frac{t^3}{3} \right]_{0}^{1}$
$= (\frac{1}{2} - \frac{1}{3}) - 0 = \frac{3-2}{6} = \frac{1}{6}$.
Karena $\innerprod{u}{v} = \frac{1}{6} \neq 0$, maka $u=1-x$ \textbf{tidak} ortogonal terhadap $v=x$.

\subsection*{ii. Carilah semua vektor $u \in V$ (seharusnya $u \in \Poly_2$) yang ortogonal terhadap $v=1$.}
\textbf{Jawaban:}\\
Misalkan $u(x) = a+bx+cx^2$. Kita ingin $\innerprod{u}{v} = 0$.
$\innerprod{a+bx+cx^2}{1} = \int_{0}^{1} (a+bt+ct^2)(1) dt = \int_{0}^{1} (a+bt+ct^2) dt$
$= \left[ at + \frac{bt^2}{2} + \frac{ct^3}{3} \right]_{0}^{1} = a + \frac{b}{2} + \frac{c}{3}$.
Agar ortogonal, $a + \frac{b}{2} + \frac{c}{3} = 0$.
Ini adalah persamaan bidang dalam ruang koefisien $(a,b,c)$.
Semua polinom $u(x) = a+bx+cx^2$ dimana $a + \frac{b}{2} + \frac{c}{3} = 0$ (atau $6a+3b+2c=0$) ortogonal terhadap $v=1$.

\subsection*{iii. Tentukan panjang vektor $w=1+x$.}
\textbf{Jawaban:}\\
$\norm{w}^2 = \innerprod{1+x}{1+x} = \int_{0}^{1} (1+t)(1+t) dt = \int_{0}^{1} (1+2t+t^2) dt$
$= \left[ t + \frac{2t^2}{2} + \frac{t^3}{3} \right]_{0}^{1} = \left[ t + t^2 + \frac{t^3}{3} \right]_{0}^{1}$
$= (1+1+\frac{1}{3}) - 0 = 2 + \frac{1}{3} = \frac{7}{3}$.
Maka $\norm{w} = \sqrt{\frac{7}{3}} = \frac{\sqrt{7}}{\sqrt{3}} = \frac{\sqrt{21}}{3}$.

\subsection*{iv. Ubahlah $B=\{1,x,x^2\}$ menjadi basis ortonormal $B'$ untuk $\Poly_2$.}
\textbf{Jawaban:}\\
Gunakan proses Gram-Schmidt. Misalkan $\vektor{v}_1=1, \vektor{v}_2=x, \vektor{v}_3=x^2$.
Langkah 1: $\vektor{w}_1 = \vektor{v}_1 = 1$.
$\norm{\vektor{w}_1}^2 = \innerprod{1}{1} = \int_{0}^{1} 1 \cdot 1 dt = [t]_0^1 = 1$. Maka $\norm{\vektor{w}_1}=1$.
$\vektor{q}_1 = \frac{\vektor{w}_1}{\norm{\vektor{w}_1}} = \frac{1}{1} = 1$.

Langkah 2: $\vektor{w}_2 = \vektor{v}_2 - \text{proj}_{\vektor{w}_1}\vektor{v}_2 = \vektor{v}_2 - \frac{\innerprod{\vektor{v}_2}{\vektor{w}_1}}{\norm{\vektor{w}_1}^2}\vektor{w}_1$.
$\innerprod{\vektor{v}_2}{\vektor{w}_1} = \innerprod{x}{1} = \int_{0}^{1} t \cdot 1 dt = [\frac{t^2}{2}]_0^1 = \frac{1}{2}$.
$\vektor{w}_2 = x - \frac{1/2}{1} \cdot 1 = x - \frac{1}{2}$.
$\norm{\vektor{w}_2}^2 = \innerprod{x-\frac{1}{2}}{x-\frac{1}{2}} = \int_{0}^{1} (t-\frac{1}{2})^2 dt = \int_{0}^{1} (t^2-t+\frac{1}{4}) dt$
$= [\frac{t^3}{3} - \frac{t^2}{2} + \frac{t}{4}]_0^1 = \frac{1}{3} - \frac{1}{2} + \frac{1}{4} = \frac{4-6+3}{12} = \frac{1}{12}$.
Maka $\norm{\vektor{w}_2} = \sqrt{\frac{1}{12}} = \frac{1}{2\sqrt{3}}$.
$\vektor{q}_2 = \frac{\vektor{w}_2}{\norm{\vektor{w}_2}} = \frac{x-1/2}{1/(2\sqrt{3})} = 2\sqrt{3}(x-\frac{1}{2}) = \sqrt{3}(2x-1)$.

Langkah 3: $\vektor{w}_3 = \vektor{v}_3 - \text{proj}_{\vektor{w}_1}\vektor{v}_3 - \text{proj}_{\vektor{w}_2}\vektor{v}_3 = \vektor{v}_3 - \frac{\innerprod{\vektor{v}_3}{\vektor{w}_1}}{\norm{\vektor{w}_1}^2}\vektor{w}_1 - \frac{\innerprod{\vektor{v}_3}{\vektor{w}_2}}{\norm{\vektor{w}_2}^2}\vektor{w}_2$.
$\innerprod{\vektor{v}_3}{\vektor{w}_1} = \innerprod{x^2}{1} = \int_{0}^{1} t^2 dt = [\frac{t^3}{3}]_0^1 = \frac{1}{3}$.
$\innerprod{\vektor{v}_3}{\vektor{w}_2} = \innerprod{x^2}{x-\frac{1}{2}} = \int_{0}^{1} t^2(t-\frac{1}{2}) dt = \int_{0}^{1} (t^3-\frac{1}{2}t^2) dt$
$= [\frac{t^4}{4} - \frac{1}{2}\frac{t^3}{3}]_0^1 = \frac{1}{4} - \frac{1}{6} = \frac{3-2}{12} = \frac{1}{12}$.
$\vektor{w}_3 = x^2 - \frac{1/3}{1}(1) - \frac{1/12}{1/12}(x-\frac{1}{2}) = x^2 - \frac{1}{3} - (x-\frac{1}{2})$
$= x^2 - x - \frac{1}{3} + \frac{1}{2} = x^2 - x + \frac{-2+3}{6} = x^2 - x + \frac{1}{6}$.
$\norm{\vektor{w}_3}^2 = \innerprod{x^2-x+\frac{1}{6}}{x^2-x+\frac{1}{6}} = \int_{0}^{1} (t^2-t+\frac{1}{6})^2 dt$
$= \int_{0}^{1} (t^4 - 2t^3 + \frac{4}{3}t^2 - \frac{1}{3}t + \frac{1}{36}) dt$ (Perhitungan ini cukup panjang)
$= [\frac{t^5}{5} - \frac{2t^4}{4} + \frac{4t^3}{9} - \frac{t^2}{6} + \frac{t}{36}]_0^1$
$= \frac{1}{5} - \frac{1}{2} + \frac{4}{9} - \frac{1}{6} + \frac{1}{36} = \frac{36 - 90 + 80 - 30 + 5}{180} = \frac{1}{180}$.
Maka $\norm{\vektor{w}_3} = \sqrt{\frac{1}{180}} = \frac{1}{6\sqrt{5}}$.
$\vektor{q}_3 = \frac{\vektor{w}_3}{\norm{\vektor{w}_3}} = \frac{x^2-x+1/6}{1/(6\sqrt{5})} = 6\sqrt{5}(x^2-x+\frac{1}{6}) = \sqrt{5}(6x^2-6x+1)$.

Basis ortonormal $B' = \{\vektor{q}_1, \vektor{q}_2, \vektor{q}_3\} = \{1, \sqrt{3}(2x-1), \sqrt{5}(6x^2-6x+1)\}$.

\subsection*{v. Tentukan proyeksi ortogonal $u=1-x$ pada subruang $W=\spanof{1,x}$.}
\textbf{Jawaban:}\\
Kita sudah memiliki basis ortogonal untuk $W$ dari langkah sebelumnya (sebelum dinormalisasi): $\vektor{w}_1=1$, $\vektor{w}_2=x-\frac{1}{2}$.
$\text{proj}_W u = \frac{\innerprod{u}{\vektor{w}_1}}{\norm{\vektor{w}_1}^2}\vektor{w}_1 + \frac{\innerprod{u}{\vektor{w}_2}}{\norm{\vektor{w}_2}^2}\vektor{w}_2$.
$u=1-x$.
$\innerprod{u}{\vektor{w}_1} = \innerprod{1-x}{1} = \int_{0}^{1} (1-t)(1) dt = [t-\frac{t^2}{2}]_0^1 = 1-\frac{1}{2} = \frac{1}{2}$.
$\norm{\vektor{w}_1}^2 = 1$.
$\innerprod{u}{\vektor{w}_2} = \innerprod{1-x}{x-\frac{1}{2}} = \int_{0}^{1} (1-t)(t-\frac{1}{2}) dt = \int_{0}^{1} (t-\frac{1}{2}-t^2+\frac{1}{2}t) dt$
$= \int_{0}^{1} (-t^2 + \frac{3}{2}t - \frac{1}{2}) dt = [-\frac{t^3}{3} + \frac{3}{4}t^2 - \frac{1}{2}t]_0^1$
$= -\frac{1}{3} + \frac{3}{4} - \frac{1}{2} = \frac{-4+9-6}{12} = -\frac{1}{12}$.
$\norm{\vektor{w}_2}^2 = \frac{1}{12}$.
$\text{proj}_W u = \frac{1/2}{1}(1) + \frac{-1/12}{1/12}(x-\frac{1}{2}) = \frac{1}{2} - 1(x-\frac{1}{2}) = \frac{1}{2} - x + \frac{1}{2} = 1-x$.
Proyeksi ortogonal $u=1-x$ pada $W=\spanof{1,x}$ adalah $1-x$ itu sendiri. Ini masuk akal karena $u=1-x$ sudah berada dalam $W$ (merupakan kombinasi linear dari $1$ dan $x$).

\end{document}
