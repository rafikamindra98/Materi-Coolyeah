\documentclass{article}
\usepackage{amsmath} % Untuk lingkungan matriks dan perintah matematika lainnya
\usepackage{amsfonts} % Untuk simbol
\usepackage{amssymb} % Untuk simbol \mathbb{R}
\usepackage{geometry} % Untuk mengatur margin
\geometry{a4paper, margin=1in}

\title{Aljabar Linear: Nilai dan Vektor Eigen}
\author{Rafi Kamindra 2201006}
\date{} % Kosongkan tanggal jika tidak ingin ditampilkan

\newcommand{\vektor}[1]{\mathbf{#1}} % Perintah untuk vektor
\newcommand{\R}{\mathbb{R}} % Simbol R
\newcommand{\matriks}[1]{\begin{pmatrix} #1 \end{pmatrix}} % Perintah untuk matriks

\begin{document}
\maketitle
\pagenumbering{gobble} % Menghilangkan nomor halaman jika tidak diinginkan untuk halaman judul

\section*{Problem}
\textbf{Carilah semua nilai dan vektor eigen yang berpadanan dari matriks-matriks:}

\subsection*{$A = \matriks{1 & 1 \\ 2 & 2}$}
\textbf{Jawaban:}\\
Persamaan karakteristik: $\det(A - \lambda I) = 0$.
\[ \begin{vmatrix} 1-\lambda & 1 \\ 2 & 2-\lambda \end{vmatrix} = (1-\lambda)(2-\lambda) - (1)(2) = 0 \]
\[ 2 - \lambda - 2\lambda + \lambda^2 - 2 = 0 \]
\[ \lambda^2 - 3\lambda = 0 \]
\[ \lambda(\lambda - 3) = 0 \]
Nilai eigen: $\lambda_1 = 0$, $\lambda_2 = 3$.

Untuk $\lambda_1 = 0$: $(A - 0I)\vektor{x} = \vektor{0}$
\[ \matriks{1 & 1 \\ 2 & 2} \matriks{x_1 \\ x_2} = \matriks{0 \\ 0} \]
$x_1 + x_2 = 0 \Rightarrow x_1 = -x_2$.
Misalkan $x_2 = t$, maka $x_1 = -t$.
Vektor eigen: $\vektor{x} = t\matriks{-1 \\ 1}$. Basis untuk ruang eigen: $\left\{ \matriks{-1 \\ 1} \right\}$.

Untuk $\lambda_2 = 3$: $(A - 3I)\vektor{x} = \vektor{0}$
\[ \matriks{1-3 & 1 \\ 2 & 2-3} \matriks{x_1 \\ x_2} = \matriks{-2 & 1 \\ 2 & -1} \matriks{x_1 \\ x_2} = \matriks{0 \\ 0} \]
$-2x_1 + x_2 = 0 \Rightarrow x_2 = 2x_1$.
Misalkan $x_1 = t$, maka $x_2 = 2t$.
Vektor eigen: $\vektor{x} = t\matriks{1 \\ 2}$. Basis untuk ruang eigen: $\left\{ \matriks{1 \\ 2} \right\}$.

\subsection*{$B = \matriks{1 & 2 \\ 4 & 3}$}
\textbf{Jawaban:}\\
Persamaan karakteristik: $\det(B - \lambda I) = 0$.
\[ \begin{vmatrix} 1-\lambda & 2 \\ 4 & 3-\lambda \end{vmatrix} = (1-\lambda)(3-\lambda) - (2)(4) = 0 \]
\[ 3 - \lambda - 3\lambda + \lambda^2 - 8 = 0 \]
\[ \lambda^2 - 4\lambda - 5 = 0 \]
\[ (\lambda - 5)(\lambda + 1) = 0 \]
Nilai eigen: $\lambda_1 = 5$, $\lambda_2 = -1$.

Untuk $\lambda_1 = 5$: $(B - 5I)\vektor{x} = \vektor{0}$
\[ \matriks{1-5 & 2 \\ 4 & 3-5} \matriks{x_1 \\ x_2} = \matriks{-4 & 2 \\ 4 & -2} \matriks{x_1 \\ x_2} = \matriks{0 \\ 0} \]
$-4x_1 + 2x_2 = 0 \Rightarrow 2x_2 = 4x_1 \Rightarrow x_2 = 2x_1$.
Misalkan $x_1 = t$, maka $x_2 = 2t$.
Vektor eigen: $\vektor{x} = t\matriks{1 \\ 2}$. Basis untuk ruang eigen: $\left\{ \matriks{1 \\ 2} \right\}$.

Untuk $\lambda_2 = -1$: $(B - (-1)I)\vektor{x} = (B+I)\vektor{x} = \vektor{0}$
\[ \matriks{1+1 & 2 \\ 4 & 3+1} \matriks{x_1 \\ x_2} = \matriks{2 & 2 \\ 4 & 4} \matriks{x_1 \\ x_2} = \matriks{0 \\ 0} \]
$2x_1 + 2x_2 = 0 \Rightarrow x_1 = -x_2$.
Misalkan $x_2 = t$, maka $x_1 = -t$.
Vektor eigen: $\vektor{x} = t\matriks{-1 \\ 1}$. Basis untuk ruang eigen: $\left\{ \matriks{-1 \\ 1} \right\}$.

\subsection*{$C = \matriks{1 & 2 \\ 3 & 2}$}
\textbf{Jawaban:}\\
Persamaan karakteristik: $\det(C - \lambda I) = 0$.
\[ \begin{vmatrix} 1-\lambda & 2 \\ 3 & 2-\lambda \end{vmatrix} = (1-\lambda)(2-\lambda) - (2)(3) = 0 \]
\[ 2 - \lambda - 2\lambda + \lambda^2 - 6 = 0 \]
\[ \lambda^2 - 3\lambda - 4 = 0 \]
\[ (\lambda - 4)(\lambda + 1) = 0 \]
Nilai eigen: $\lambda_1 = 4$, $\lambda_2 = -1$.

Untuk $\lambda_1 = 4$: $(C - 4I)\vektor{x} = \vektor{0}$
\[ \matriks{1-4 & 2 \\ 3 & 2-4} \matriks{x_1 \\ x_2} = \matriks{-3 & 2 \\ 3 & -2} \matriks{x_1 \\ x_2} = \matriks{0 \\ 0} \]
$-3x_1 + 2x_2 = 0 \Rightarrow 2x_2 = 3x_1 \Rightarrow x_2 = \frac{3}{2}x_1$.
Misalkan $x_1 = 2t$, maka $x_2 = 3t$.
Vektor eigen: $\vektor{x} = t\matriks{2 \\ 3}$. Basis untuk ruang eigen: $\left\{ \matriks{2 \\ 3} \right\}$.

Untuk $\lambda_2 = -1$: $(C - (-1)I)\vektor{x} = (C+I)\vektor{x} = \vektor{0}$
\[ \matriks{1+1 & 2 \\ 3 & 2+1} \matriks{x_1 \\ x_2} = \matriks{2 & 2 \\ 3 & 3} \matriks{x_1 \\ x_2} = \matriks{0 \\ 0} \]
$2x_1 + 2x_2 = 0 \Rightarrow x_1 = -x_2$.
Misalkan $x_2 = t$, maka $x_1 = -t$.
Vektor eigen: $\vektor{x} = t\matriks{-1 \\ 1}$. Basis untuk ruang eigen: $\left\{ \matriks{-1 \\ 1} \right\}$.

\subsection*{$D = \matriks{0 & 1 & 0 \\ 0 & 0 & 2 \\ 0 & 0 & 0}$}
\textbf{Jawaban:}\\
Karena $D$ adalah matriks segitiga atas, nilai eigen adalah elemen-elemen diagonalnya.
Nilai eigen: $\lambda_1 = 0, \lambda_2 = 0, \lambda_3 = 0$ (atau $\lambda=0$ dengan multiplisitas aljabar 3).

Untuk $\lambda = 0$: $(D - 0I)\vektor{x} = D\vektor{x} = \vektor{0}$
\[ \matriks{0 & 1 & 0 \\ 0 & 0 & 2 \\ 0 & 0 & 0} \matriks{x_1 \\ x_2 \\ x_3} = \matriks{0 \\ 0 \\ 0} \]
$0x_1 + 1x_2 + 0x_3 = 0 \Rightarrow x_2 = 0$.
$0x_1 + 0x_2 + 2x_3 = 0 \Rightarrow 2x_3 = 0 \Rightarrow x_3 = 0$.
$x_1$ adalah variabel bebas. Misalkan $x_1 = t$.
Vektor eigen: $\vektor{x} = t\matriks{1 \\ 0 \\ 0}$. Basis untuk ruang eigen: $\left\{ \matriks{1 \\ 0 \\ 0} \right\}$.
Multiplisitas geometris untuk $\lambda=0$ adalah 1.

\subsection*{$E = \matriks{4 & 0 & 1 \\ 2 & 3 & 2 \\ 1 & 0 & 4}$}
\textbf{Jawaban:}\\
Persamaan karakteristik: $\det(E - \lambda I) = 0$.
\[ \begin{vmatrix} 4-\lambda & 0 & 1 \\ 2 & 3-\lambda & 2 \\ 1 & 0 & 4-\lambda \end{vmatrix} = 0 \]
Ekspansi kofaktor sepanjang kolom kedua:
$(3-\lambda) \begin{vmatrix} 4-\lambda & 1 \\ 1 & 4-\lambda \end{vmatrix} = 0$
$(3-\lambda) [ (4-\lambda)^2 - 1 ] = 0$
$(3-\lambda) [ (4-\lambda-1)(4-\lambda+1) ] = 0$
$(3-\lambda) (3-\lambda)(5-\lambda) = 0$
$(3-\lambda)^2 (5-\lambda) = 0$.
Nilai eigen: $\lambda_1 = 3$ (multiplisitas 2), $\lambda_2 = 5$.

Untuk $\lambda_1 = 3$: $(E - 3I)\vektor{x} = \vektor{0}$
\[ \matriks{4-3 & 0 & 1 \\ 2 & 3-3 & 2 \\ 1 & 0 & 4-3} \matriks{x_1 \\ x_2 \\ x_3} = \matriks{1 & 0 & 1 \\ 2 & 0 & 2 \\ 1 & 0 & 1} \matriks{x_1 \\ x_2 \\ x_3} = \matriks{0 \\ 0 \\ 0} \]
$x_1 + x_3 = 0 \Rightarrow x_1 = -x_3$.
$2x_1 + 2x_3 = 0 \Rightarrow x_1 = -x_3$.
$x_2$ adalah variabel bebas. Misalkan $x_2 = s$, $x_3 = t$. Maka $x_1 = -t$.
Vektor eigen: $\vektor{x} = \matriks{-t \\ s \\ t} = s\matriks{0 \\ 1 \\ 0} + t\matriks{-1 \\ 0 \\ 1}$.
Basis untuk ruang eigen: $\left\{ \matriks{0 \\ 1 \\ 0}, \matriks{-1 \\ 0 \\ 1} \right\}$. Multiplisitas geometris adalah 2.

Untuk $\lambda_2 = 5$: $(E - 5I)\vektor{x} = \vektor{0}$
\[ \matriks{4-5 & 0 & 1 \\ 2 & 3-5 & 2 \\ 1 & 0 & 4-5} \matriks{x_1 \\ x_2 \\ x_3} = \matriks{-1 & 0 & 1 \\ 2 & -2 & 2 \\ 1 & 0 & -1} \matriks{x_1 \\ x_2 \\ x_3} = \matriks{0 \\ 0 \\ 0} \]
$-x_1 + x_3 = 0 \Rightarrow x_1 = x_3$.
$x_1 - x_3 = 0 \Rightarrow x_1 = x_3$.
$2x_1 - 2x_2 + 2x_3 = 0 \Rightarrow x_1 - x_2 + x_3 = 0$.
Substitusikan $x_1=x_3$: $x_1 - x_2 + x_1 = 0 \Rightarrow 2x_1 - x_2 = 0 \Rightarrow x_2 = 2x_1$.
Misalkan $x_1 = t$, maka $x_3 = t$, $x_2 = 2t$.
Vektor eigen: $\vektor{x} = t\matriks{1 \\ 2 \\ 1}$. Basis untuk ruang eigen: $\left\{ \matriks{1 \\ 2 \\ 1} \right\}$.

\subsection*{$F = \matriks{1 & 1 & 1 \\ 0 & 1 & 0 \\ 0 & 1 & 2}$}
\textbf{Jawaban:}\\
Karena $F$ adalah matriks segitiga atas, nilai eigen adalah elemen-elemen diagonalnya.
Nilai eigen: $\lambda_1 = 1$ (multiplisitas 2), $\lambda_2 = 2$.

Untuk $\lambda_1 = 1$: $(F - I)\vektor{x} = \vektor{0}$
\[ \matriks{1-1 & 1 & 1 \\ 0 & 1-1 & 0 \\ 0 & 1 & 2-1} \matriks{x_1 \\ x_2 \\ x_3} = \matriks{0 & 1 & 1 \\ 0 & 0 & 0 \\ 0 & 1 & 1} \matriks{x_1 \\ x_2 \\ x_3} = \matriks{0 \\ 0 \\ 0} \]
$x_2 + x_3 = 0 \Rightarrow x_2 = -x_3$.
$x_1$ adalah variabel bebas. Misalkan $x_1 = s$, $x_3 = t$. Maka $x_2 = -t$.
Vektor eigen: $\vektor{x} = \matriks{s \\ -t \\ t} = s\matriks{1 \\ 0 \\ 0} + t\matriks{0 \\ -1 \\ 1}$.
Basis untuk ruang eigen: $\left\{ \matriks{1 \\ 0 \\ 0}, \matriks{0 \\ -1 \\ 1} \right\}$. Multiplisitas geometris adalah 2.

Untuk $\lambda_2 = 2$: $(F - 2I)\vektor{x} = \vektor{0}$
\[ \matriks{1-2 & 1 & 1 \\ 0 & 1-2 & 0 \\ 0 & 1 & 2-2} \matriks{x_1 \\ x_2 \\ x_3} = \matriks{-1 & 1 & 1 \\ 0 & -1 & 0 \\ 0 & 1 & 0} \matriks{x_1 \\ x_2 \\ x_3} = \matriks{0 \\ 0 \\ 0} \]
$-x_2 = 0 \Rightarrow x_2 = 0$.
$x_2 = 0$.
$-x_1 + x_2 + x_3 = 0 \Rightarrow -x_1 + 0 + x_3 = 0 \Rightarrow x_1 = x_3$.
Misalkan $x_3 = t$, maka $x_1 = t$, $x_2 = 0$.
Vektor eigen: $\vektor{x} = t\matriks{1 \\ 0 \\ 1}$. Basis untuk ruang eigen: $\left\{ \matriks{1 \\ 0 \\ 1} \right\}$.

\subsection*{$G = \matriks{1 & 3 & 0 \\ 0 & -2 & 0 \\ 0 & 6 & 1}$}
\textbf{Jawaban:}\\
Karena $G$ adalah matriks segitiga bawah, nilai eigen adalah elemen-elemen diagonalnya.
Nilai eigen: $\lambda_1 = 1$ (multiplisitas 2), $\lambda_2 = -2$.

Untuk $\lambda_1 = 1$: $(G - I)\vektor{x} = \vektor{0}$
\[ \matriks{1-1 & 3 & 0 \\ 0 & -2-1 & 0 \\ 0 & 6 & 1-1} \matriks{x_1 \\ x_2 \\ x_3} = \matriks{0 & 3 & 0 \\ 0 & -3 & 0 \\ 0 & 6 & 0} \matriks{x_1 \\ x_2 \\ x_3} = \matriks{0 \\ 0 \\ 0} \]
$3x_2 = 0 \Rightarrow x_2 = 0$.
$-3x_2 = 0 \Rightarrow x_2 = 0$.
$6x_2 = 0 \Rightarrow x_2 = 0$.
$x_1$ dan $x_3$ adalah variabel bebas. Misalkan $x_1 = s$, $x_3 = t$.
Vektor eigen: $\vektor{x} = \matriks{s \\ 0 \\ t} = s\matriks{1 \\ 0 \\ 0} + t\matriks{0 \\ 0 \\ 1}$.
Basis untuk ruang eigen: $\left\{ \matriks{1 \\ 0 \\ 0}, \matriks{0 \\ 0 \\ 1} \right\}$. Multiplisitas geometris adalah 2.

Untuk $\lambda_2 = -2$: $(G - (-2)I)\vektor{x} = (G+2I)\vektor{x} = \vektor{0}$
\[ \matriks{1+2 & 3 & 0 \\ 0 & -2+2 & 0 \\ 0 & 6 & 1+2} \matriks{x_1 \\ x_2 \\ x_3} = \matriks{3 & 3 & 0 \\ 0 & 0 & 0 \\ 0 & 6 & 3} \matriks{x_1 \\ x_2 \\ x_3} = \matriks{0 \\ 0 \\ 0} \]
$3x_1 + 3x_2 = 0 \Rightarrow x_1 = -x_2$.
$6x_2 + 3x_3 = 0 \Rightarrow 3x_3 = -6x_2 \Rightarrow x_3 = -2x_2$.
Misalkan $x_2 = t$, maka $x_1 = -t$, $x_3 = -2t$.
Vektor eigen: $\vektor{x} = t\matriks{-1 \\ 1 \\ -2}$. Basis untuk ruang eigen: $\left\{ \matriks{-1 \\ 1 \\ -2} \right\}$.

\end{document}
