\documentclass{article}
\usepackage{amsmath} % Untuk lingkungan matriks dan perintah matematika lainnya
\usepackage{amsfonts} % Untuk simbol
\usepackage{amssymb} % Untuk simbol \mathbb{R}
\usepackage{geometry} % Untuk mengatur margin
\geometry{a4paper, margin=1in}

\title{Aljabar Linear: Transformasi Linear}
\author{Rafi Kamindra 2201006}
\date{} % Kosongkan tanggal jika tidak ingin ditampilkan

\newcommand{\vektor}[1]{\mathbf{#1}} % Perintah untuk vektor
\newcommand{\R}{\mathbb{R}} % Simbol R
\newcommand{\Poly}{\mathbb{P}} % Simbol P untuk polinom
\newcommand{\Transformasi}[1]{T(#1)} % Notasi Transformasi

\begin{document}
\maketitle
\pagenumbering{gobble} % Menghilangkan nomor halaman jika tidak diinginkan untuk halaman judul

\section*{Problem}

\subsection*{1. Diketahui transformasi linear $T: \R^2 \rightarrow \R^2$ dengan $T[(x,y)]=(1,2)$ untuk semua $(x,y) \in \R^2$. Periksa, apakah $T$ linear.}
\textbf{Jawaban:}\\
Untuk menjadi transformasi linear, $T$ harus memenuhi dua syarat:
\begin{enumerate}
    \item $T(\vektor{u} + \vektor{v}) = T(\vektor{u}) + T(\vektor{v})$ untuk semua $\vektor{u}, \vektor{v} \in \R^2$.
    \item $T(k\vektor{u}) = kT(\vektor{u})$ untuk semua $\vektor{u} \in \R^2$ dan skalar $k$.
\end{enumerate}
Mari kita periksa syarat pertama.
Misalkan $\vektor{u} = (x_1, y_1)$ dan $\vektor{v} = (x_2, y_2)$.
$T(\vektor{u} + \vektor{v}) = T((x_1+x_2, y_1+y_2))$.
Menurut definisi $T$, $T((x_1+x_2, y_1+y_2)) = (1,2)$.
Sekarang hitung $T(\vektor{u}) + T(\vektor{v})$.
$T(\vektor{u}) = T((x_1,y_1)) = (1,2)$.
$T(\vektor{v}) = T((x_2,y_2)) = (1,2)$.
$T(\vektor{u}) + T(\vektor{v}) = (1,2) + (1,2) = (2,4)$.
Karena $(1,2) \neq (2,4)$ (kecuali jika $1=2$ dan $2=4$, yang tidak mungkin), maka $T(\vektor{u} + \vektor{v}) \neq T(\vektor{u}) + T(\vektor{v})$.
Syarat pertama tidak terpenuhi.

Mari kita periksa syarat kedua (meskipun satu syarat gagal sudah cukup).
$T(k\vektor{u}) = T((kx_1, ky_1))$.
Menurut definisi $T$, $T((kx_1, ky_1)) = (1,2)$.
Sekarang hitung $kT(\vektor{u})$.
$kT(\vektor{u}) = k T((x_1,y_1)) = k(1,2) = (k, 2k)$.
Agar $T(k\vektor{u}) = kT(\vektor{u})$, maka $(1,2) = (k, 2k)$. Ini berarti $k=1$ dan $2k=2 \Rightarrow k=1$.
Syarat ini hanya terpenuhi untuk $k=1$, tidak untuk semua skalar $k$. Misalnya, jika $k=2$, $T(2\vektor{u})=(1,2)$ tetapi $2T(\vektor{u})=(2,4)$.
Karena kedua syarat tidak terpenuhi (cukup satu saja), maka $T$ bukan transformasi linear.
(Catatan: Transformasi linear harus memetakan vektor nol ke vektor nol. $T((0,0))=(1,2) \neq (0,0)$. Ini juga menunjukkan $T$ tidak linear).

\subsection*{2. Diketahui transformasi linear $T: \R^2 \rightarrow \R^2$ dimana $T[(1,1)]=(3,2)$ dan $T[(1,-1)]=(0,-3)$. Tentukan rumus $T[(a,b)]$.}
\textbf{Jawaban:}\\
Kita ingin menyatakan vektor $(a,b)$ sebagai kombinasi linear dari $(1,1)$ dan $(1,-1)$.
Misalkan $(a,b) = c_1(1,1) + c_2(1,-1) = (c_1+c_2, c_1-c_2)$.
Maka $c_1+c_2 = a$ dan $c_1-c_2 = b$.
Jumlahkan kedua persamaan: $(c_1+c_2) + (c_1-c_2) = a+b \Rightarrow 2c_1 = a+b \Rightarrow c_1 = \frac{a+b}{2}$.
Kurangkan persamaan kedua dari pertama: $(c_1+c_2) - (c_1-c_2) = a-b \Rightarrow 2c_2 = a-b \Rightarrow c_2 = \frac{a-b}{2}$.
Karena $T$ linear:
$T[(a,b)] = T[c_1(1,1) + c_2(1,-1)]$
$= c_1 T[(1,1)] + c_2 T[(1,-1)]$
$= \frac{a+b}{2} (3,2) + \frac{a-b}{2} (0,-3)$
$= (\frac{3(a+b)}{2} + \frac{0(a-b)}{2}, \frac{2(a+b)}{2} + \frac{-3(a-b)}{2})$
$= (\frac{3a+3b}{2}, \frac{2a+2b-3a+3b}{2})$
$= (\frac{3a+3b}{2}, \frac{-a+5b}{2})$.
Jadi, rumus $T[(a,b)] = (\frac{3a+3b}{2}, \frac{-a+5b}{2})$.

\subsection*{3. Diketahui transformasi linear $T: \R^2 \rightarrow \R^3$ dimana $T[(2,1)]=(3,2,0)$ dan $T[(0,-1)]=(2,1,-3)$. Tentukan peta dari $(1,0)$ dan prapeta dari $(1,1,1)$.}
\textbf{Jawaban:}\\
\textbf{Peta dari $(1,0)$}:
Kita ingin menyatakan $(1,0)$ sebagai kombinasi linear dari $(2,1)$ dan $(0,-1)$.
Misalkan $(1,0) = c_1(2,1) + c_2(0,-1) = (2c_1, c_1-c_2)$.
Maka $2c_1 = 1 \Rightarrow c_1 = 1/2$.
Dan $c_1-c_2 = 0 \Rightarrow c_2 = c_1 = 1/2$.
Jadi, $(1,0) = \frac{1}{2}(2,1) + \frac{1}{2}(0,-1)$.
Karena $T$ linear:
$T[(1,0)] = T[\frac{1}{2}(2,1) + \frac{1}{2}(0,-1)]$
$= \frac{1}{2} T[(2,1)] + \frac{1}{2} T[(0,-1)]$
$= \frac{1}{2} (3,2,0) + \frac{1}{2} (2,1,-3)$
$= (\frac{3}{2}, 1, 0) + (1, \frac{1}{2}, -\frac{3}{2})$
$= (\frac{3}{2}+1, 1+\frac{1}{2}, 0-\frac{3}{2})$
$= (\frac{5}{2}, \frac{3}{2}, -\frac{3}{2})$.
Jadi, peta dari $(1,0)$ adalah $(\frac{5}{2}, \frac{3}{2}, -\frac{3}{2})$.

\textbf{Prapeta dari $(1,1,1)$}:
Kita mencari vektor $(a,b)$ sehingga $T[(a,b)] = (1,1,1)$.
Misalkan $(a,b) = k_1(2,1) + k_2(0,-1) = (2k_1, k_1-k_2)$.
Maka $T[(a,b)] = k_1 T[(2,1)] + k_2 T[(0,-1)]$
$= k_1(3,2,0) + k_2(2,1,-3)$
$= (3k_1+2k_2, 2k_1+k_2, -3k_2)$.
Kita ingin $(3k_1+2k_2, 2k_1+k_2, -3k_2) = (1,1,1)$.
Ini menghasilkan sistem persamaan:
$3k_1 + 2k_2 = 1 \quad (1)$
$2k_1 + k_2 = 1 \quad (2)$
$-3k_2 = 1 \quad (3)$
Dari (3), $k_2 = -1/3$.
Substitusikan $k_2 = -1/3$ ke (2):
$2k_1 - 1/3 = 1 \Rightarrow 2k_1 = 1 + 1/3 = 4/3 \Rightarrow k_1 = 2/3$.
Periksa dengan (1):
$3(2/3) + 2(-1/3) = 2 - 2/3 = 6/3 - 2/3 = 4/3$.
Hasilnya $4/3 \neq 1$. Ini berarti ada kontradiksi.
Sistem persamaan untuk $k_1, k_2$ tidak konsisten.
Ini berarti $(1,1,1)$ tidak berada dalam jangkauan (range) dari $T$.
Jadi, tidak ada prapeta untuk $(1,1,1)$ di bawah transformasi $T$ ini.

Mari kita periksa kembali.
Vektor basis domain adalah $\vektor{v}_1=(2,1)$ dan $\vektor{v}_2=(0,-1)$.
$T(\vektor{v}_1) = (3,2,0)$ dan $T(\vektor{v}_2) = (2,1,-3)$.
Matriks transformasi $A$ terhadap basis standar (jika kita bisa menentukannya) akan memetakan $(a,b)$ ke $(1,1,1)$.
Misalkan $T[(a,b)] = A \begin{pmatrix} a \\ b \end{pmatrix}$.
Kita tahu $T[(1,0)] = (5/2, 3/2, -3/2)$.
Kita perlu $T[(0,1)]$.
$(0,1) = d_1(2,1) + d_2(0,-1) = (2d_1, d_1-d_2)$.
$2d_1=0 \Rightarrow d_1=0$.
$d_1-d_2=1 \Rightarrow 0-d_2=1 \Rightarrow d_2=-1$.
Jadi $(0,1) = 0(2,1) - 1(0,-1)$.
$T[(0,1)] = 0 \cdot T[(2,1)] - 1 \cdot T[(0,-1)] = -(2,1,-3) = (-2,-1,3)$.
Matriks standar $A = \begin{pmatrix} T[(1,0)] & T[(0,1)] \end{pmatrix} = \begin{pmatrix} 5/2 & -2 \\ 3/2 & -1 \\ -3/2 & 3 \end{pmatrix}$.
Kita mencari $(a,b)$ sehingga $A \begin{pmatrix} a \\ b \end{pmatrix} = \begin{pmatrix} 1 \\ 1 \\ 1 \end{pmatrix}$.
$\begin{pmatrix} 5/2 & -2 \\ 3/2 & -1 \\ -3/2 & 3 \end{pmatrix} \begin{pmatrix} a \\ b \end{pmatrix} = \begin{pmatrix} 1 \\ 1 \\ 1 \end{pmatrix}$.
Sistem persamaan:
$\frac{5}{2}a - 2b = 1 \quad (I)$
$\frac{3}{2}a - b = 1 \quad (II)$
$-\frac{3}{2}a + 3b = 1 \quad (III)$
Dari (II), $b = \frac{3}{2}a - 1$.
Substitusikan ke (I):
$\frac{5}{2}a - 2(\frac{3}{2}a - 1) = 1$
$\frac{5}{2}a - 3a + 2 = 1$
$\frac{5}{2}a - \frac{6}{2}a = 1 - 2$
$-\frac{1}{2}a = -1 \Rightarrow a = 2$.
Maka $b = \frac{3}{2}(2) - 1 = 3 - 1 = 2$.
Periksa dengan (III):
$-\frac{3}{2}(2) + 3(2) = -3 + 6 = 3$.
Hasilnya $3 \neq 1$. Kontradiksi.
Jadi, memang tidak ada prapeta untuk $(1,1,1)$.

\end{document}
