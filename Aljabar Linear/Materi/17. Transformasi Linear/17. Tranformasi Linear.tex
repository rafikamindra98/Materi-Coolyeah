\documentclass[12pt, a4paper]{article}
\usepackage{amsmath} % Untuk lingkungan matriks dan perintah matematika lainnya
\usepackage{amsfonts} % Untuk simbol matematika
\usepackage{amssymb} % Untuk simbol \mathbb{R}
\usepackage{geometry} % Untuk mengatur margin
\usepackage[indonesian]{babel} % Mengatur bahasa Indonesia
\usepackage{array} % Untuk kolom tabel yang lebih baik
\usepackage{booktabs} % Untuk garis tabel yang lebih baik
\usepackage{graphicx} % Untuk menyertakan gambar (jika diperlukan)
\usepackage{hyperref} % Untuk hyperlink (jika diperlukan)
\usepackage{amsthm} % Untuk lingkungan definisi dan teorema
\usepackage{enumitem} % Untuk kustomisasi label pada enumerate
\usepackage{nicematrix} % Untuk matriks dengan garis partisi
\usepackage{palatino} % Menggunakan font Palatino untuk tampilan yang lebih menarik
\usepackage{listings} % Untuk kode (jika diperlukan)
\usepackage{systeme} % Untuk sistem persamaan

\geometry{a4paper, margin=1in, headheight=15pt} % Mengatur margin halaman

% Definisi lingkungan baru
\theoremstyle{definition} % Gaya standar untuk definisi dan contoh
\newtheorem{definisi}{Definisi}[section]
\newtheorem{contoh}{Contoh}[section]
\newtheorem{catatan}{Catatan}[section]
\theoremstyle{plain} % Gaya standar untuk teorema
\newtheorem{teorema}{Teorema}[section]
\newtheorem{fakta}{Fakta}[section]
\newtheorem{proposisi}{Proposisi}[section]
\newtheorem{akibat}{Akibat}[section]

% Definisi perintah baru untuk kemudahan
\newcommand{\matriks}[1]{\begin{pmatrix} #1 \end{pmatrix}} % Perintah untuk matriks
\newcommand{\R}{\mathbb{R}} % Simbol R untuk bilangan real
\newcommand{\Poly}{\mathbb{P}} % Simbol P untuk polinom
\newcommand{\vektor}[1]{\mathbf{#1}} % Perintah untuk vektor tebal
\newcommand{\nol}{\mathbf{0}} % Vektor nol
\newcommand{\Transformasi}[1]{T\left(#1\right)} % Notasi Transformasi
\newcommand{\TransformasiKurung}[1]{T[#1]} % Notasi Transformasi dengan kurung siku
\newcommand{\kernel}{\text{ker}} % Kernel
\newcommand{\rank}{\text{rank}} % Rank
\newcommand{\Range}{\text{R}} % Jangkauan (Range)
\newcommand{\nulitas}{\text{nulitas}} % Nulitas

\title{Transformasi Linear}
\author{Rafi Kamindra}
\date{}

\begin{document}
\maketitle

\section{Definisi Transformasi Linear}
Dalam aljabar linear, transformasi linear adalah fungsi antara dua ruang vektor yang mempertahankan operasi penjumlahan vektor dan perkalian skalar.

\begin{definisi}
Misalkan $V$ dan $W$ masing-masing adalah ruang vektor atas lapangan skalar $\R$. Sebuah transformasi (pemetaan) $T: V \longrightarrow W$ dikatakan \textbf{linear} jika untuk semua vektor $\vektor{u}, \vektor{v} \in V$ dan semua skalar $\alpha \in \R$ berlaku kedua sifat berikut:
\begin{enumerate}[label=(\roman*)]
    \item \textbf{Aditivitas}: $\Transformasi{\vektor{u} + \vektor{v}} = \Transformasi{\vektor{u}} + \Transformasi{\vektor{v}}$
    \item \textbf{Homogenitas}: $\Transformasi{\alpha \vektor{u}} = \alpha \Transformasi{\vektor{u}}$
\end{enumerate}
Jika $\Transformasi{\vektor{v}} = \vektor{w}$, maka $\vektor{w}$ disebut \textbf{peta} (image) dari $\vektor{v}$ oleh $T$, dan $\vektor{v}$ disebut \textbf{prapeta} (preimage) dari $\vektor{w}$.
\end{definisi}
Kedua sifat di atas dapat digabungkan menjadi satu syarat: $T(\alpha\vektor{u} + \beta\vektor{v}) = \alpha T(\vektor{u}) + \beta T(\vektor{v})$ untuk semua $\vektor{u},\vektor{v} \in V$ dan semua skalar $\alpha, \beta \in \R$.

\begin{fakta}
Jika $T: V \longrightarrow W$ adalah transformasi linear, maka $T$ memetakan vektor nol di $V$ ke vektor nol di $W$, yaitu $\Transformasi{\nol_V} = \nol_W$.
\end{fakta}
\begin{proof}
Kita dapat menulis $\nol_V = 0\vektor{u}$ untuk suatu $\vektor{u} \in V$. Dengan menggunakan sifat homogenitas (ii):
\[ \Transformasi{\nol_V} = \Transformasi{0\vektor{u}} = 0 \Transformasi{\vektor{u}} \]
Karena $0$ dikalikan vektor apapun di $W$ menghasilkan vektor nol di $W$, maka $0 \Transformasi{\vektor{u}} = \nol_W$.
Jadi, $\Transformasi{\nol_V} = \nol_W$.
\end{proof}
Fakta ini sering digunakan sebagai tes awal: jika $\Transformasi{\nol_V} \neq \nol_W$, maka $T$ pasti tidak linear. Namun, jika $\Transformasi{\nol_V} = \nol_W$, belum tentu $T$ linear; kedua aksioma utama tetap harus diperiksa.

\section{Contoh-Contoh Transformasi Linear dan Bukan Linear}

\begin{contoh}[Pemetaan Nol dan Pemetaan Identitas]
\begin{enumerate}
    \item \textbf{Pemetaan Nol}: Misalkan $V$ dan $W$ adalah ruang vektor. Definisikan pemetaan $T_0: V \longrightarrow W$ dengan $\TransformasiKurung{\vektor{x}} = \nol_W$ untuk setiap $\vektor{x} \in V$.
    Kita periksa kelinearannya:
    \begin{itemize}
        \item $T_0[\vektor{u}+\vektor{v}] = \nol_W$. Juga, $T_0[\vektor{u}] + T_0[\vektor{v}] = \nol_W + \nol_W = \nol_W$. Jadi, $T_0[\vektor{u}+\vektor{v}] = T_0[\vektor{u}] + T_0[\vektor{v}]$.
        \item $T_0[\alpha\vektor{u}] = \nol_W$. Juga, $\alpha T_0[\vektor{u}] = \alpha \nol_W = \nol_W$. Jadi, $T_0[\alpha\vektor{u}] = \alpha T_0[\vektor{u}]$.
    \end{itemize}
    Karena kedua sifat terpenuhi, pemetaan nol $T_0$ adalah transformasi linear.

    \item \textbf{Pemetaan Identitas}: Misalkan $V$ suatu ruang vektor. Definisikan pemetaan $I: V \longrightarrow V$ dengan $\TransformasiKurung{\vektor{x}} = \vektor{x}$ untuk setiap $\vektor{x} \in V$.
    \begin{itemize}
        \item $I[\vektor{u}+\vektor{v}] = \vektor{u}+\vektor{v}$. Juga, $I[\vektor{u}] + I[\vektor{v}] = \vektor{u} + \vektor{v}$. Jadi, $I[\vektor{u}+\vektor{v}] = I[\vektor{u}] + I[\vektor{v}]$.
        \item $I[\alpha\vektor{u}] = \alpha\vektor{u}$. Juga, $\alpha I[\vektor{u}] = \alpha\vektor{u}$. Jadi, $I[\alpha\vektor{u}] = \alpha I[\vektor{u}]$.
    \end{itemize}
    Karena kedua sifat terpenuhi, pemetaan identitas $I$ adalah transformasi linear.
\end{enumerate}
\end{contoh}

\begin{contoh}[Transformasi Tidak Linear]
Definisikan pemetaan $T: \R^2 \longrightarrow \R^2$ dengan $T[(x,y)] = (x+y, y+2)$ untuk $(x,y) \in \R^2$.
Periksa apakah $T$ linear.
\textbf{Solusi}:
Kita periksa apakah $T$ memetakan vektor nol ke vektor nol.
$T[(0,0)] = (0+0, 0+2) = (0,2)$.
Karena $T[(0,0)] = (0,2) \neq (0,0)$ (vektor nol di $\R^2$), maka $T$ \textbf{bukan} transformasi linear.
\end{contoh}

\begin{contoh}[Transformasi Linear dari $\R^2$ ke $\R^2$]
Didefinisikan pemetaan $T: \R^2 \longrightarrow \R^2$ dengan $T[(x,y)] = (x+2y, 3x-y)$.
\begin{itemize}
    \item Tentukan peta dari $\vektor{u}=(1,2)$ dan prapeta dari $(1,1)$.
    \item Periksa, apakah $T$ linear.
\end{itemize}
\textbf{Solusi}:
\begin{itemize}
    \item Peta dari $\vektor{u}=(1,2)$ adalah $T[(1,2)] = (1+2(2), 3(1)-2) = (1+4, 3-2) = (5,1)$.
    \item Prapeta dari $(1,1)$ adalah $(x,y) \in \R^2$ yang memenuhi $T[(x,y)] = (1,1)$.
    $(x+2y, 3x-y) = (1,1)$. Ini menghasilkan sistem:
    \begin{align*} x + 2y &= 1 \\ 3x - y &= 1 \end{align*}
    Dari persamaan kedua, $y = 3x-1$. Substitusi ke persamaan pertama:
    $x + 2(3x-1) = 1 \Rightarrow x + 6x - 2 = 1 \Rightarrow 7x = 3 \Rightarrow x = 3/7$.
    Maka $y = 3(3/7) - 1 = 9/7 - 7/7 = 2/7$.
    Jadi, prapeta dari $(1,1)$ adalah $(3/7, 2/7)$.

    \item Periksa kelinearan $T$:
    Ambil $\vektor{u}=(x_1,y_1)$, $\vektor{v}=(x_2,y_2) \in \R^2$, dan $\alpha \in \R$.
    \begin{enumerate}[label=(\roman*)]
        \item Aditivitas:
        \begin{align*}
        T[\vektor{u}+\vektor{v}] &= T[(x_1+x_2, y_1+y_2)] \\
        &= ((x_1+x_2) + 2(y_1+y_2), 3(x_1+x_2) - (y_1+y_2)) \\
        &= (x_1+2y_1 + x_2+2y_2, 3x_1-y_1 + 3x_2-y_2) \\
        &= (x_1+2y_1, 3x_1-y_1) + (x_2+2y_2, 3x_2-y_2) \\
        &= T[\vektor{u}] + T[\vektor{v}].
        \end{align*}
        \item Homogenitas:
        \begin{align*}
        T[\alpha\vektor{u}] &= T[(\alpha x_1, \alpha y_1)] \\
        &= (\alpha x_1 + 2\alpha y_1, 3\alpha x_1 - \alpha y_1) \\
        &= \alpha(x_1+2y_1, 3x_1-y_1) \\
        &= \alpha T[\vektor{u}].
        \end{align*}
    \end{enumerate}
    Karena kedua sifat terpenuhi, $T$ adalah transformasi linear.
\end{itemize}
\end{contoh}

\begin{contoh}[Transformasi Matriks]
Didefinisikan pemetaan $T: \R^3 \longrightarrow \R^2$ dengan $T[\vektor{u}] = A\vektor{u}$ dimana $A = \matriks{1 & -2 & 1 \\ 2 & 0 & 3}$.
Peta dari $\vektor{u}=(1,2,3)^T$ adalah $A\vektor{u} = \matriks{1 & -2 & 1 \\ 2 & 0 & 3} \matriks{1 \\ 2 \\ 3} = \matriks{1-4+3 \\ 2+0+9} = \matriks{0 \\ 11}$.
Transformasi yang didefinisikan oleh perkalian matriks selalu linear.
Ambil $\vektor{u}, \vektor{v} \in \R^3$ dan $\alpha \in \R$.
\begin{enumerate}[label=(\roman*)]
    \item $T[\vektor{u}+\vektor{v}] = A(\vektor{u}+\vektor{v}) = A\vektor{u} + A\vektor{v}$ (sifat distributif perkalian matriks) $= T[\vektor{u}] + T[\vektor{v}]$.
    \item $T[\alpha\vektor{u}] = A(\alpha\vektor{u}) = \alpha(A\vektor{u})$ (sifat asosiatif skalar pada perkalian matriks) $= \alpha T[\vektor{u}]$.
\end{enumerate}
Jadi, $T$ adalah transformasi linear. Secara umum, setiap transformasi matriks $T_A: \R^n \to \R^m$ yang didefinisikan oleh $T_A(\vektor{x}) = A\vektor{x}$ (dimana $A$ adalah matriks $m \times n$) adalah transformasi linear.
\end{contoh}

\begin{contoh}[Transformasi pada Ruang Polinom]
Didefinisikan pemetaan $T: \Poly_2 \longrightarrow \Poly_2$ dengan $T[a+bx+cx^2] = a+(b+c)x+cx^2$.
Ambil $\vektor{u} = a+bx+cx^2$, $\vektor{v} = d+ex+fx^2 \in \Poly_2$, dan $\alpha \in \R$.
\begin{enumerate}[label=(\roman*)]
    \item $T[\vektor{u}+\vektor{v}] = T[(a+d)+(b+e)x+(c+f)x^2]$
    $= (a+d) + ((b+e)+(c+f))x + (c+f)x^2$
    $= (a+d) + (b+c+e+f)x + (c+f)x^2$.
    $T[\vektor{u}] + T[\vektor{v}] = (a+(b+c)x+cx^2) + (d+(e+f)x+fx^2)$
    $= (a+d) + (b+c+e+f)x + (c+f)x^2$.
    Jadi, $T[\vektor{u}+\vektor{v}] = T[\vektor{u}] + T[\vektor{v}]$.
    \item $T[\alpha\vektor{u}] = T[\alpha a + \alpha b x + \alpha c x^2]$
    $= (\alpha a) + (\alpha b + \alpha c)x + (\alpha c)x^2$
    $= \alpha (a + (b+c)x + cx^2) = \alpha T[\vektor{u}]$.
\end{enumerate}
Karena kedua sifat terpenuhi, $T$ adalah transformasi linear.
\end{contoh}

\section{Cara Mendefinisikan Transformasi Linear}
Misalkan $V$ dan $W$ masing-masing adalah ruang vektor. Ada dua cara utama untuk mendefinisikan transformasi linear $T: V \longrightarrow W$:
\begin{enumerate}
    \item \textbf{Menyatakan peta setiap vektor di $V$ secara eksplisit} (dengan rumus umum).
    Contoh: $T[(x,y)] = (x+2y, 3x-y)$.
    \item \textbf{Menentukan peta dari vektor-vektor basis untuk $V$}.
    Jika $B=\{\vektor{v}_1, \dots, \vektor{v}_n\}$ adalah basis untuk $V$, dan kita mengetahui $T[\vektor{v}_1], \dots, T[\vektor{v}_n]$, maka peta dari sebarang vektor $\vektor{x} \in V$ dapat ditentukan.
    Karena $\vektor{x} = c_1\vektor{v}_1 + \dots + c_n\vektor{v}_n$ untuk skalar unik $c_i$, maka karena kelinearan $T$:
    \[ T[\vektor{x}] = T[c_1\vektor{v}_1 + \dots + c_n\vektor{v}_n] = c_1T[\vektor{v}_1] + \dots + c_n T[\vektor{v}_n] \]
\end{enumerate}

\begin{contoh}
Definisikan transformasi linear $T: \R^2 \longrightarrow \R^2$ dimana $T[(1,0)]=(1,2)$ dan $T[(0,1)]=(2,-3)$.
\begin{itemize}
    \item Apa hasil pemetaan $(3,4)$ oleh $T$?
    \item Secara umum, apa rumus $T[(a,b)]$?
\end{itemize}
\textbf{Solusi}:
Basis standar untuk $\R^2$ adalah $B=\{\vektor{e}_1=(1,0), \vektor{e}_2=(0,1)\}$.
Sebarang vektor $(a,b) \in \R^2$ dapat ditulis sebagai $(a,b) = a(1,0) + b(0,1) = a\vektor{e}_1 + b\vektor{e}_2$.
Maka,
\begin{align*}
T[(a,b)] &= T[a\vektor{e}_1 + b\vektor{e}_2] \\
&= aT[\vektor{e}_1] + bT[\vektor{e}_2] \quad (\text{karena } T \text{ linear}) \\
&= a(1,2) + b(2,-3) \\
&= (a,2a) + (2b,-3b) \\
&= (a+2b, 2a-3b)
\end{align*}
Ini adalah rumus umum untuk $T[(a,b)]$.
Untuk $(3,4)$: $a=3, b=4$.
$T[(3,4)] = (3+2(4), 2(3)-3(4)) = (3+8, 6-12) = (11, -6)$.
\end{contoh}

\end{document}
