\documentclass[12pt, a4paper]{article}
\usepackage{amsmath} % Untuk lingkungan matriks dan perintah matematika lainnya
\usepackage{amsfonts} % Untuk simbol matematika
\usepackage{amssymb} % Untuk simbol \mathbb{R}
\usepackage{geometry} % Untuk mengatur margin
\usepackage[indonesian]{babel} % Mengatur bahasa Indonesia
\usepackage{array} % Untuk kolom tabel yang lebih baik
\usepackage{booktabs} % Untuk garis tabel yang lebih baik
\usepackage{graphicx} % Untuk menyertakan gambar (jika diperlukan)
\usepackage{hyperref} % Untuk hyperlink (jika diperlukan)
\usepackage{amsthm} % Untuk lingkungan definisi dan teorema
\usepackage{nicematrix} % Untuk matriks dengan garis partisi
\usepackage{palatino} % Menggunakan font Palatino untuk tampilan yang lebih menarik

\geometry{a4paper, margin=1in, headheight=15pt} % Mengatur margin halaman

% Definisi lingkungan baru
\theoremstyle{definition} % Gaya standar untuk definisi dan contoh
\newtheorem{definisi}{Definisi}[section]
\newtheorem{contoh}{Contoh}[section]
\newtheorem{catatan}{Catatan}[section]
\theoremstyle{plain} % Gaya standar untuk teorema
\newtheorem{teorema}{Teorema}[section]
\newtheorem{fakta}{Fakta}[section]
\newtheorem{proposisi}{Proposisi}[section]
\newtheorem{akibat}{Akibat}[section]

% Definisi perintah baru untuk kemudahan
\newcommand{\matriks}[1]{\begin{pmatrix} #1 \end{pmatrix}} % Perintah untuk matriks
\newcommand{\R}{\mathbb{R}} % Simbol R untuk bilangan real
\newcommand{\vektor}[1]{\mathbf{#1}} % Perintah untuk vektor tebal
\newcommand{\nol}{\mathbf{0}} % Vektor nol
\newcommand{\spanof}[1]{\text{span}\{#1\}} % Perintah untuk span
\newcommand{\rank}{\text{rank}} % Perintah untuk rank
\newcommand{\rowspace}[1]{\text{R}_{#1}} % Ruang baris
\newcommand{\colspace}[1]{\text{C}_{#1}} % Ruang kolom

\title{Ruang Baris dan Ruang Kolom Matriks}
\author{Rafi Kamindra 2201006}
\date{}

\begin{document}
\maketitle

\section{Pendahuluan}
Setiap matriks memiliki dua subruang fundamental yang terkait dengannya: ruang baris dan ruang kolom. Memahami subruang-subruang ini penting karena mereka mengungkapkan informasi mendalam tentang matriks dan transformasi linear yang diwakilinya, termasuk rank matriks dan konsistensi sistem persamaan linear.

\section{Ruang Baris (Row Space)}
\begin{definisi}
Perhatikan matriks $A$ berukuran $m \times n$:
\[ A = \matriks{
a_{11} & a_{12} & \cdots & a_{1n} \\
a_{21} & a_{22} & \cdots & a_{2n} \\
\vdots & \vdots & \ddots & \vdots \\
a_{m1} & a_{m2} & \cdots & a_{mn}
} \]
Vektor-vektor baris dari $A$ adalah:
\begin{align*}
\vektor{r}_1 &= (a_{11}, a_{12}, \dots, a_{1n}) \in \R^n \\
\vektor{r}_2 &= (a_{21}, a_{22}, \dots, a_{2n}) \in \R^n \\
&\vdots \\
\vektor{r}_m &= (a_{m1}, a_{m2}, \dots, a_{mn}) \in \R^n
\end{align*}
\textbf{Ruang baris} dari matriks $A$, dinotasikan $\rowspace{A}$ (atau Row$(A)$), adalah subruang dari $\R^n$ yang direntang oleh vektor-vektor baris $A$.
\[ \rowspace{A} = \spanof{\vektor{r}_1, \vektor{r}_2, \dots, \vektor{r}_m} = \{ \alpha_1\vektor{r}_1 + \alpha_2\vektor{r}_2 + \dots + \alpha_m\vektor{r}_m \mid \alpha_i \in \R \} \]
\end{definisi}

\begin{contoh}
Perhatikan matriks $A = \matriks{1 & 2 & -1 \\ 3 & 0 & 4}$.
Vektor-vektor barisnya adalah $\vektor{r}_1 = (1,2,-1)$ dan $\vektor{r}_2 = (3,0,4)$.
Ruang baris dari $A$ adalah $\rowspace{A} = \spanof{(1,2,-1), (3,0,4)} = \{ \alpha(1,2,-1) + \beta(3,0,4) \mid \alpha, \beta \in \R \}$.
$\rowspace{A}$ adalah subruang dari $\R^3$.
Dimensi dari $\rowspace{A}$ adalah rank dari matriks $A$. Dalam contoh ini, kedua vektor baris linear independen (bukan kelipatan skalar satu sama lain), sehingga $\dim(\rowspace{A}) = 2$.
\end{contoh}

\subsection{Operasi Baris Elementer (OBE) dan Ruang Baris}
Sebuah fakta fundamental adalah bahwa Operasi Baris Elementer tidak mengubah ruang baris suatu matriks.
\begin{fakta}
\begin{enumerate}
    \item \textbf{Pertukaran Baris}: Jika matriks $B$ diperoleh dari matriks $A$ dengan menukar dua baris, maka $\rowspace{A} = \rowspace{B}$. (Himpunan vektor barisnya tetap sama, hanya urutannya berubah, sehingga span-nya juga sama).
    \item \textbf{Perkalian Baris dengan Skalar tak Nol}: Jika matriks $B$ diperoleh dari $A$ dengan mengalikan satu baris dari $A$ dengan skalar tak nol $k$, maka $\rowspace{A} = \rowspace{B}$.
    (Misalkan $\vektor{r}_i$ adalah baris ke-$i$ dari $A$. Di $B$, baris ini menjadi $k\vektor{r}_i$. Setiap kombinasi linear dari baris-baris $A$ dapat ditulis sebagai kombinasi linear dari baris-baris $B$, dan sebaliknya).
    \item \textbf{Penjumlahan Kelipatan Baris ke Baris Lain}: Jika matriks $B$ diperoleh dari $A$ dengan menambahkan kelipatan $k$ dari baris $\vektor{r}_i$ ke baris $\vektor{r}_j$ (yaitu $R_j \to R_j + kR_i$), maka $\rowspace{A} = \rowspace{B}$.
    (Vektor-vektor baris baru di $B$ adalah $\{\vektor{r}_1, \dots, \vektor{r}_j+k\vektor{r}_i, \dots, \vektor{r}_m\}$. Setiap kombinasi linear dari baris $A$ dapat ditulis sebagai kombinasi linear dari baris $B$, dan sebaliknya).
\end{enumerate}
\end{fakta}

\begin{teorema}
Operasi Baris Elementer pada suatu matriks tidak mengubah ruang baris matriks tersebut. Dengan kata lain, jika matriks $A$ ekuivalen baris dengan matriks $B$ ($A \sim B$), maka $\rowspace{A} = \rowspace{B}$.
\end{teorema}
Teorema ini sangat penting karena memungkinkan kita untuk mencari basis ruang baris dengan cara mereduksi matriks ke bentuk yang lebih sederhana (bentuk eselon baris).

\subsection{Basis untuk Ruang Baris}
\begin{teorema}
Misalkan $A$ suatu matriks dan $A'$ adalah bentuk eselon baris dari $A$. Maka, baris-baris tak nol pada $A'$ membentuk basis untuk ruang baris $A$ (yaitu, $\rowspace{A}$).
\end{teorema}
\textbf{Konsekuensi}: Dimensi ruang baris $A$ (yaitu $\rank(A)$) adalah jumlah baris tak nol dalam bentuk eselon baris $A'$.

\begin{contoh}
Tentukan basis ruang baris dari matriks $A = \matriks{1 & 2 & -1 & 0 \\ 3 & 7 & 4 & 2 \\ 4 & 9 & 3 & 2}$.
\textbf{Solusi}:
Lakukan OBE pada $A$:
\[ A = \matriks{1 & 2 & -1 & 0 \\ 3 & 7 & 4 & 2 \\ 4 & 9 & 3 & 2} \xrightarrow{R_2 \to R_2 - 3R_1, R_3 \to R_3 - 4R_1} \matriks{1 & 2 & -1 & 0 \\ 0 & 1 & 7 & 2 \\ 0 & 1 & 7 & 2} \]
\[ \xrightarrow{R_3 \to R_3 - R_2} \matriks{1 & 2 & -1 & 0 \\ 0 & 1 & 7 & 2 \\ 0 & 0 & 0 & 0} = A' \]
Bentuk eselon baris $A'$ memiliki dua baris tak nol: $(1,2,-1,0)$ dan $(0,1,7,2)$.
Jadi, basis untuk ruang baris $\rowspace{A}$ adalah $B_{\rowspace{A}} = \{(1,2,-1,0), (0,1,7,2)\}$.
Dimensi $\rowspace{A} = \rank(A) = 2$.
\end{contoh}

\section{Ruang Kolom (Column Space)}
\begin{definisi}
Perhatikan matriks $A$ berukuran $m \times n$:
\[ A = \matriks{
a_{11} & a_{12} & \cdots & a_{1n} \\
a_{21} & a_{22} & \cdots & a_{2n} \\
\vdots & \vdots & \ddots & \vdots \\
a_{m1} & a_{m2} & \cdots & a_{mn}
} \]
Vektor-vektor kolom dari $A$ adalah:
\[ \vektor{c}_1 = \matriks{a_{11} \\ a_{21} \\ \vdots \\ a_{m1}} \in \R^m, \quad \vektor{c}_2 = \matriks{a_{12} \\ a_{22} \\ \vdots \\ a_{m2}} \in \R^m, \quad \dots, \quad \vektor{c}_n = \matriks{a_{1n} \\ a_{2n} \\ \vdots \\ a_{mn}} \in \R^m \]
\textbf{Ruang kolom} dari matriks $A$, dinotasikan $\colspace{A}$ (atau Col$(A)$), adalah subruang dari $\R^m$ yang direntang oleh vektor-vektor kolom $A$.
\[ \colspace{A} = \spanof{\vektor{c}_1, \vektor{c}_2, \dots, \vektor{c}_n} = \{ x_1\vektor{c}_1 + x_2\vektor{c}_2 + \dots + x_n\vektor{c}_n \mid x_i \in \R \} \]
\end{definisi}

\begin{contoh}
Perhatikan matriks $A = \matriks{1 & 2 & -1 \\ 3 & 0 & 4}$.
Vektor-vektor kolomnya adalah $\vektor{c}_1 = \matriks{1 \\ 3}$, $\vektor{c}_2 = \matriks{2 \\ 0}$, $\vektor{c}_3 = \matriks{-1 \\ 4}$.
Ruang kolom dari $A$ adalah $\colspace{A} = \spanof{\matriks{1 \\ 3}, \matriks{2 \\ 0}, \matriks{-1 \\ 4}}$.
$\colspace{A}$ adalah subruang dari $\R^2$.
Untuk menentukan dimensinya, kita bisa memeriksa kebebasan linear dari ketiga vektor ini. Karena berada di $\R^2$, paling banyak 2 vektor yang bisa bebas linear.
Matriks $\matriks{1 & 2 & -1 \\ 3 & 0 & 4} \xrightarrow{R_2 \to R_2-3R_1} \matriks{1 & 2 & -1 \\ 0 & -6 & 7}$. Ranknya 2.
Jadi, $\dim(\colspace{A})=2$.
\end{contoh}

\subsection{Ruang Kolom dan Konsistensi SPL}
\begin{fakta}
Sebuah vektor $\vektor{b} \in \R^m$ berada dalam ruang kolom $\colspace{A}$ dari matriks $A \in M_{m \times n}$ jika dan hanya jika sistem persamaan linear $A\vektor{x} = \vektor{b}$ mempunyai solusi (konsisten).
\end{fakta}
\begin{proof}
Persamaan $A\vektor{x} = \vektor{b}$ dapat ditulis sebagai:
\[ x_1\vektor{c}_1 + x_2\vektor{c}_2 + \dots + x_n\vektor{c}_n = \vektor{b} \]
dimana $\vektor{c}_j$ adalah kolom ke-$j$ dari $A$.
Sistem ini memiliki solusi jika dan hanya jika $\vektor{b}$ dapat dinyatakan sebagai kombinasi linear dari kolom-kolom $A$, yang berarti $\vektor{b} \in \spanof{\vektor{c}_1, \dots, \vektor{c}_n} = \colspace{A}$.
\end{proof}

\subsection{Basis untuk Ruang Kolom}
\textbf{Penting}: Operasi Baris Elementer \textit{dapat mengubah} ruang kolom dari sebuah matriks. Jadi, kita tidak bisa langsung menggunakan kolom-kolom dari bentuk eselon baris sebagai basis ruang kolom matriks asli.

\begin{teorema}
Misalkan $A$ suatu matriks dan $A'$ adalah bentuk eselon baris dari $A$. Kolom-kolom pada matriks \textbf{asli} $A$ yang bersesuaian dengan kolom-kolom pada $A'$ yang memuat 1 utama (pivot) membentuk basis untuk ruang kolom $A$ (yaitu, $\colspace{A}$).
\end{teorema}
\textbf{Konsekuensi}: Dimensi ruang kolom $A$ juga sama dengan $\rank(A)$.

\begin{teorema}[Teorema Rank]
Untuk setiap matriks $A$, dimensi ruang barisnya sama dengan dimensi ruang kolomnya. Dimensi ini disebut \textbf{rank} dari $A$.
\[ \dim(\rowspace{A}) = \dim(\colspace{A}) = \rank(A) \]
\end{teorema}

\begin{contoh}
Tentukan basis ruang kolom dari matriks $A = \matriks{1 & 2 & -1 & 0 \\ 3 & 7 & -4 & -1 \\ -1 & 2 & -3 & -4}$.
\textbf{Solusi}:
Lakukan OBE pada $A$:
\[ A = \matriks{1 & 2 & -1 & 0 \\ 3 & 7 & -4 & -1 \\ -1 & 2 & -3 & -4} \xrightarrow{R_2 \to R_2 - 3R_1, R_3 \to R_3 + R_1} \matriks{1 & 2 & -1 & 0 \\ 0 & 1 & -1 & -1 \\ 0 & 4 & -4 & -4} \]
\[ \xrightarrow{R_3 \to R_3 - 4R_2} \matriks{1 & 2 & -1 & 0 \\ 0 & 1 & -1 & -1 \\ 0 & 0 & 0 & 0} = A' \]
Bentuk eselon baris $A'$ memiliki 1 utama (pivot) pada kolom pertama dan kolom kedua.
Maka, kolom pertama dan kolom kedua dari matriks \textbf{asli} $A$ membentuk basis untuk ruang kolom $\colspace{A}$.
Basis $\colspace{A} = \left\{ \matriks{1 \\ 3 \\ -1}, \matriks{2 \\ 7 \\ 2} \right\}$.
Dimensi $\colspace{A} = \rank(A) = 2$.
\end{contoh}

\section{Ringkasan Hasil Utama}
Misalkan $A$ suatu matriks dan $A'$ adalah bentuk eselon baris dari $A$.
\begin{enumerate}
    \item Baris-baris tak nol pada $A'$ merupakan basis untuk ruang baris $\rowspace{A}$.
    \item Kolom-kolom pada $A$ (matriks asli) yang bersesuaian dengan kolom-kolom pada $A'$ yang memuat 1 utama merupakan basis untuk ruang kolom $\colspace{A}$.
    \item $\dim(\rowspace{A}) = \dim(\colspace{A}) = \rank(A) =$ banyaknya 1 utama di $A'$.
\end{enumerate}

\end{document}
