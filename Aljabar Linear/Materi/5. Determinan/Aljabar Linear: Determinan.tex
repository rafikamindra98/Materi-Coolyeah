\documentclass{article}
\usepackage{amsmath} % Untuk lingkungan matriks dan perintah matematika lainnya
\usepackage{amsfonts} % Untuk simbol
\usepackage{amssymb} % Untuk simbol \mathbb{R}
\usepackage{geometry} % Untuk mengatur margin
\geometry{a4paper, margin=1in}

\title{Aljabar Linear: Determinan}
\author{Rafi Kamindra 2201006}
\date{} % Kosongkan tanggal jika tidak ingin ditampilkan

\begin{document}
\maketitle
\pagenumbering{gobble} % Menghilangkan nomor halaman jika tidak diinginkan untuk halaman judul

\section*{Hitung:}

\subsection*{1.
\[ \begin{vmatrix} 1 & 3 & -2 & 1 \\ -1 & 2 & 3 & -1 \\ 2 & 3 & 4 & 1 \\ 3 & 2 & 0 & 1 \end{vmatrix} \]}
\textbf{Jawaban:}\\
Misalkan $D_1 = \begin{vmatrix} 1 & 3 & -2 & 1 \\ -1 & 2 & 3 & -1 \\ 2 & 3 & 4 & 1 \\ 3 & 2 & 0 & 1 \end{vmatrix}$.
Kita lakukan ekspansi kofaktor sepanjang baris ke-4 (karena ada elemen 0):
\begin{align*} D_1 &= 3(-1)^{4+1} \begin{vmatrix} 3 & -2 & 1 \\ 2 & 3 & -1 \\ 3 & 4 & 1 \end{vmatrix} + 2(-1)^{4+2} \begin{vmatrix} 1 & -2 & 1 \\ -1 & 3 & -1 \\ 2 & 4 & 1 \end{vmatrix} \\ & \quad + 0(-1)^{4+3} \begin{vmatrix} \dots \end{vmatrix} + 1(-1)^{4+4} \begin{vmatrix} 1 & 3 & -2 \\ -1 & 2 & 3 \\ 2 & 3 & 4 \end{vmatrix} \end{align*}
Hitung determinan 3x3:
\begin{align*} \begin{vmatrix} 3 & -2 & 1 \\ 2 & 3 & -1 \\ 3 & 4 & 1 \end{vmatrix} &= 3(3 - (-4)) - (-2)(2 - (-3)) + 1(8 - 9) \\ &= 3(7) + 2(5) + 1(-1) = 21 + 10 - 1 = 30 \end{align*}
\begin{align*} \begin{vmatrix} 1 & -2 & 1 \\ -1 & 3 & -1 \\ 2 & 4 & 1 \end{vmatrix} &= 1(3 - (-4)) - (-2)(-1 - (-2)) + 1(-4 - 6) \\ &= 1(7) + 2(1) + 1(-10) = 7 + 2 - 10 = -1 \end{align*}
\begin{align*} \begin{vmatrix} 1 & 3 & -2 \\ -1 & 2 & 3 \\ 2 & 3 & 4 \end{vmatrix} &= 1(8 - 9) - 3(-4 - 6) + (-2)(-3 - 4) \\ &= 1(-1) - 3(-10) - 2(-7) = -1 + 30 + 14 = 43 \end{align*}
Maka,
\begin{align*} D_1 &= 3(-1)(30) + 2(1)(-1) + 0 + 1(1)(43) \\ &= -90 - 2 + 43 \\ &= -49 \end{align*}

\subsection*{2.
\[ \begin{vmatrix} 2 & 4 & 6 & 8 \\ 1 & 2 & 3 & -1 \\ 2 & -2 & 2 & -2 \\ -3 & 0 & 3 & 0 \end{vmatrix} \]}
\textbf{Jawaban:}\\
Misalkan $D_2 = \begin{vmatrix} 2 & 4 & 6 & 8 \\ 1 & 2 & 3 & -1 \\ 2 & -2 & 2 & -2 \\ -3 & 0 & 3 & 0 \end{vmatrix}$.
Faktorkan konstanta dari baris:
$R_1 \rightarrow \frac{1}{2}R_1$: $D_2 = 2 \begin{vmatrix} 1 & 2 & 3 & 4 \\ 1 & 2 & 3 & -1 \\ 2 & -2 & 2 & -2 \\ -3 & 0 & 3 & 0 \end{vmatrix}$.
$R_3 \rightarrow \frac{1}{2}R_3$: $D_2 = 2 \cdot 2 \begin{vmatrix} 1 & 2 & 3 & 4 \\ 1 & 2 & 3 & -1 \\ 1 & -1 & 1 & -1 \\ -3 & 0 & 3 & 0 \end{vmatrix}$.
$R_4 \rightarrow \frac{1}{3}R_4$: $D_2 = 2 \cdot 2 \cdot 3 \begin{vmatrix} 1 & 2 & 3 & 4 \\ 1 & 2 & 3 & -1 \\ 1 & -1 & 1 & -1 \\ -1 & 0 & 1 & 0 \end{vmatrix} = 12 \begin{vmatrix} 1 & 2 & 3 & 4 \\ 1 & 2 & 3 & -1 \\ 1 & -1 & 1 & -1 \\ -1 & 0 & 1 & 0 \end{vmatrix}$.
Misalkan $M = \begin{vmatrix} 1 & 2 & 3 & 4 \\ 1 & 2 & 3 & -1 \\ 1 & -1 & 1 & -1 \\ -1 & 0 & 1 & 0 \end{vmatrix}$.
Lakukan operasi $R_2 \rightarrow R_2 - R_1$:
$M = \begin{vmatrix} 1 & 2 & 3 & 4 \\ 0 & 0 & 0 & -5 \\ 1 & -1 & 1 & -1 \\ -1 & 0 & 1 & 0 \end{vmatrix}$.
Ekspansi kofaktor sepanjang baris ke-2:
$M = (-5)(-1)^{2+4} \begin{vmatrix} 1 & 2 & 3 \\ 1 & -1 & 1 \\ -1 & 0 & 1 \end{vmatrix}$.
\begin{align*} \begin{vmatrix} 1 & 2 & 3 \\ 1 & -1 & 1 \\ -1 & 0 & 1 \end{vmatrix} &= 1(-1 - 0) - 2(1 - (-1)) + 3(0 - 1) \\ &= 1(-1) - 2(2) + 3(-1) \\ &= -1 - 4 - 3 = -8 \end{align*}
Maka $M = (-5)(1)(-8) = 40$.
Jadi, $D_2 = 12 \cdot M = 12 \cdot 40 = 480$.

\subsection*{3.
\[ \begin{vmatrix} 0.25 & 0.5 & 0.25 \\ 2 & 1 & 3 \\ -3 & 2 & 1 \end{vmatrix} \]}
\textbf{Jawaban:}\\
Misalkan $D_3 = \begin{vmatrix} 0.25 & 0.5 & 0.25 \\ 2 & 1 & 3 \\ -3 & 2 & 1 \end{vmatrix}$.
Faktorkan 0.25 dari baris pertama:
$D_3 = 0.25 \begin{vmatrix} 1 & 2 & 1 \\ 2 & 1 & 3 \\ -3 & 2 & 1 \end{vmatrix}$.
\begin{align*} \begin{vmatrix} 1 & 2 & 1 \\ 2 & 1 & 3 \\ -3 & 2 & 1 \end{vmatrix} &= 1(1 - 6) - 2(2 - (-9)) + 1(4 - (-3)) \\ &= 1(-5) - 2(2+9) + 1(4+3) \\ &= -5 - 2(11) + 1(7) \\ &= -5 - 22 + 7 = -20 \end{align*}
Maka $D_3 = 0.25 \cdot (-20) = -5$.

\subsection*{4. $\begin{vmatrix} 1 & a & a^2 \\ 1 & b & b^2 \\ 1 & c & c^2 \end{vmatrix}$, $a, b, c \in \mathbb{R}$}
\textbf{Jawaban:}\\
Ini adalah determinan Vandermonde.
$D_4 = \begin{vmatrix} 1 & a & a^2 \\ 1 & b & b^2 \\ 1 & c & c^2 \end{vmatrix}$.
Lakukan operasi baris:
$R_2 \rightarrow R_2 - R_1$:
$R_3 \rightarrow R_3 - R_1$:
\[ D_4 = \begin{vmatrix} 1 & a & a^2 \\ 0 & b-a & b^2-a^2 \\ 0 & c-a & c^2-a^2 \end{vmatrix} \]
Ekspansi kofaktor sepanjang kolom pertama:
\[ D_4 = 1 \cdot \begin{vmatrix} b-a & b^2-a^2 \\ c-a & c^2-a^2 \end{vmatrix} = \begin{vmatrix} b-a & (b-a)(b+a) \\ c-a & (c-a)(c+a) \end{vmatrix} \]
Faktorkan $(b-a)$ dari baris pertama dan $(c-a)$ dari baris kedua:
\[ D_4 = (b-a)(c-a) \begin{vmatrix} 1 & b+a \\ 1 & c+a \end{vmatrix} \]
\[ D_4 = (b-a)(c-a)((c+a) - (b+a)) = (b-a)(c-a)(c+a-b-a) = (b-a)(c-a)(c-b) \]
Jadi, $D_4 = (b-a)(c-a)(c-b)$.

\subsection*{5. $\begin{vmatrix} -1 & -3 & 0 \\ -1 & x & x \\ x & 1 & -2 \end{vmatrix}$, $x \in \mathbb{R}$}
\textbf{Jawaban:}\\
Misalkan $D_5 = \begin{vmatrix} -1 & -3 & 0 \\ -1 & x & x \\ x & 1 & -2 \end{vmatrix}$.
Ekspansi kofaktor sepanjang baris pertama:
\begin{align*} D_5 &= -1 \begin{vmatrix} x & x \\ 1 & -2 \end{vmatrix} - (-3) \begin{vmatrix} -1 & x \\ x & -2 \end{vmatrix} + 0 \begin{vmatrix} -1 & x \\ x & 1 \end{vmatrix} \\ &= -1(-2x - x) + 3(2 - x^2) + 0 \\ &= -1(-3x) + 6 - 3x^2 \\ &= 3x + 6 - 3x^2 \end{align*}
Jadi, $D_5 = -3x^2 + 3x + 6$.

\subsection*{6.
\[ \begin{vmatrix} 0 & 0 & 0 & 2 \\ 0 & 0 & 3 & 7 \\ 0 & -2 & 2 & 1 \\ 3 & 1 & 4 & -5 \end{vmatrix} \]}
\textbf{Jawaban:}\\
Misalkan $D_6 = \begin{vmatrix} 0 & 0 & 0 & 2 \\ 0 & 0 & 3 & 7 \\ 0 & -2 & 2 & 1 \\ 3 & 1 & 4 & -5 \end{vmatrix}$.
Kita dapat melakukan serangkaian pertukaran baris untuk mengubahnya menjadi matriks segitiga atas.
$R_1 \leftrightarrow R_4$: Determinan dikalikan $(-1)$.
\[ -\begin{vmatrix} 3 & 1 & 4 & -5 \\ 0 & 0 & 3 & 7 \\ 0 & -2 & 2 & 1 \\ 0 & 0 & 0 & 2 \end{vmatrix} \]
$R_2 \leftrightarrow R_3$: Determinan dikalikan $(-1)$ lagi.
\[ (-1)(-1)\begin{vmatrix} 3 & 1 & 4 & -5 \\ 0 & -2 & 2 & 1 \\ 0 & 0 & 3 & 7 \\ 0 & 0 & 0 & 2 \end{vmatrix} = \begin{vmatrix} 3 & 1 & 4 & -5 \\ 0 & -2 & 2 & 1 \\ 0 & 0 & 3 & 7 \\ 0 & 0 & 0 & 2 \end{vmatrix} \]
Matriks ini adalah matriks segitiga atas. Determinannya adalah hasil kali elemen-elemen diagonalnya.
$D_6 = 3 \cdot (-2) \cdot 3 \cdot 2 = -36$.

\end{document}
