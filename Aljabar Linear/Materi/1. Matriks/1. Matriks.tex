\documentclass[12pt, a4paper]{article}
\usepackage{amsmath} % Untuk lingkungan matriks dan perintah matematika lainnya
\usepackage{amsfonts} % Untuk simbol matematika
\usepackage{amssymb} % Untuk simbol \mathbb{R}
\usepackage{geometry} % Untuk mengatur margin
\usepackage[indonesian]{babel} % Mengatur bahasa Indonesia
\usepackage{array} % Untuk kolom tabel yang lebih baik
\usepackage{booktabs} % Untuk garis tabel yang lebih baik
\usepackage{graphicx} % Untuk menyertakan gambar (jika diperlukan)
\usepackage{hyperref} % Untuk hyperlink (jika diperlukan)

\geometry{a4paper, margin=1in} % Mengatur margin halaman

% Definisi perintah baru untuk kemudahan
\newcommand{\matriks}[1]{\begin{pmatrix} #1 \end{pmatrix}} % Perintah untuk matriks
\newcommand{\R}{\mathbb{R}} % Simbol R untuk bilangan real
\newcommand{\M}[1]{\mathcal{M}_{#1}} % Simbol M untuk himpunan matriks

\title{Operasi-Operasi pada Matriks}
\author{Rafi Kamindra 2201006}
\date{}

\begin{document}
\maketitle

\section{Definisi dan Notasi Matriks}
Matriks adalah susunan bilangan dalam bentuk persegi panjang yang diatur dalam baris dan kolom. Bilangan-bilangan dalam matriks disebut sebagai entri atau elemen matriks.

\subsection{Ukuran Matriks}
Sebuah matriks real $A = [a_{ij}]$ dikatakan berukuran $p \times q$ jika matriks $A$ terdiri dari $p$ buah baris dan $q$ buah kolom. Ini dapat diilustrasikan sebagai berikut:
\[ A = \matriks{
a_{11} & a_{12} & \cdots & a_{1q} \\
a_{21} & a_{22} & \cdots & a_{2q} \\
\vdots & \vdots & \ddots & \vdots \\
a_{p1} & a_{p2} & \cdots & a_{pq}
} \]
Di sini, $a_{ij}$ adalah entri pada baris ke-$i$ dan kolom ke-$j$.

\subsection{Matriks Persegi}
Jika jumlah baris sama dengan jumlah kolom, yaitu $p=q$, maka matriks $A$ disebut \textbf{matriks persegi} orde $p$ (atau $q$). Entri-entri $a_{11}, a_{22}, \dots, a_{pp}$ disebut sebagai \textbf{entri-entri diagonal utama} dari matriks $A$.

\subsection{Notasi Himpunan Matriks}
Himpunan semua matriks berukuran $p \times q$ dengan entri-entri real dinotasikan sebagai $\M{p \times q}(\R)$ atau cukup $M_{p \times q}$ jika konteksnya jelas bahwa entrinya real. Jika $p=q$, notasi yang digunakan bisa $M_p(\R)$ atau $M_p$.

\subsection{Contoh Matriks}
\begin{itemize}
    \item Contoh matriks $2 \times 3$:
    \[ A = \matriks{1 & -2 & 0 \\ -3 & 1 & 4} \in M_{2 \times 3} \]
    \item Contoh matriks persegi $3 \times 3$:
    \[ B = \matriks{1 & -2 & 2 \\ -2 & 0 & 1 \\ 3 & -1 & 1} \in M_3 \]
    Entri-entri diagonal utama matriks $B$ adalah $b_{11}=1$, $b_{22}=0$, dan $b_{33}=1$.
\end{itemize}

\section{Beberapa Jenis Matriks Khusus}
\begin{description}
    \item[Matriks Nol ($0$)] Matriks yang semua entrinya adalah nol. Ini berlaku untuk matriks persegi maupun bukan. Contoh: $0_{2 \times 3} = \matriks{0 & 0 & 0 \\ 0 & 0 & 0}$.
    \item[Matriks Identitas ($I_n$)] Matriks persegi $n \times n$ yang semua entri diagonal utamanya adalah 1, dan entri lainnya adalah 0. Dinotasikan sebagai $I_n = [a_{ij}]$ dimana $a_{ii}=1$ dan $a_{ij}=0$ untuk $i \neq j$. Contoh: $I_3 = \matriks{1 & 0 & 0 \\ 0 & 1 & 0 \\ 0 & 0 & 1}$.
    \item[Matriks Diagonal ($D$)] Matriks persegi yang semua entri non-diagonal utamanya adalah nol. Dinotasikan sebagai $D = [a_{ij}]$ dimana $a_{ij}=0$ untuk $i \neq j$. Contoh: $D = \matriks{2 & 0 & 0 \\ 0 & -5 & 0 \\ 0 & 0 & 1}$.
    \item[Matriks Segitiga Atas] Matriks persegi yang semua entri di bawah diagonal utamanya adalah nol. Dinotasikan sebagai $U = [a_{ij}]$ dimana $a_{ij}=0$ untuk $i > j$. Contoh: $U = \matriks{1 & 2 & 3 \\ 0 & 4 & 5 \\ 0 & 0 & 6}$.
    \item[Matriks Segitiga Bawah] Matriks persegi yang semua entri di atas diagonal utamanya adalah nol. Dinotasikan sebagai $L = [a_{ij}]$ dimana $a_{ij}=0$ untuk $i < j$. Contoh: $L = \matriks{1 & 0 & 0 \\ 2 & 4 & 0 \\ 3 & 5 & 6}$.
\end{description}

\section{Operasi-Operasi Dasar pada Matriks}
Operasi-operasi dasar pada matriks meliputi:
\begin{enumerate}
    \item Penjumlahan Matriks
    \item Perkalian Matriks dengan Skalar
    \item Perkalian Matriks dengan Matriks
    \item Transpose Matriks
\end{enumerate}

\subsection{Penjumlahan Matriks}
Dua buah matriks $A=[a_{ij}]$ dan $B=[b_{ij}]$ dapat dijumlahkan jika dan hanya jika keduanya memiliki ukuran yang sama (jumlah baris sama dan jumlah kolom sama). Jika $A, B \in M_{p \times q}$, maka hasil penjumlahannya adalah matriks $C = A+B \in M_{p \times q}$ dengan entri $c_{ij} = a_{ij} + b_{ij}$.
\subsubsection{Contoh Penjumlahan Matriks}
Misalkan $A = \matriks{1 & 3 & 2 \\ -1 & 2 & 0 \\ -2 & 1 & 4}$ dan $B = \matriks{3 & 0 & 2 \\ 1 & 2 & 3 \\ 1 & -1 & 1}$.
Maka,
\[ A+B = \matriks{1+3 & 3+0 & 2+2 \\ -1+1 & 2+2 & 0+3 \\ -2+1 & 1+(-1) & 4+1} = \matriks{4 & 3 & 4 \\ 0 & 4 & 3 \\ -1 & 0 & 5} \]

\subsection{Perkalian Matriks dengan Skalar}
Jika $A=[a_{ij}]$ adalah sebuah matriks berukuran $p \times q$ dan $k$ adalah sebuah skalar (bilangan real), maka perkalian skalar $kA$ adalah matriks berukuran $p \times q$ yang entrinya $k \cdot a_{ij}$.
\subsubsection{Contoh Perkalian Skalar}
Misalkan $A = \matriks{1 & 3 & 2 \\ -1 & 2 & 0 \\ -2 & 1 & 4}$ dan $k=3$.
Maka,
\[ kA = 3A = 3\matriks{1 & 3 & 2 \\ -1 & 2 & 0 \\ -2 & 1 & 4} = \matriks{3 \cdot 1 & 3 \cdot 3 & 3 \cdot 2 \\ 3 \cdot (-1) & 3 \cdot 2 & 3 \cdot 0 \\ 3 \cdot (-2) & 3 \cdot 1 & 3 \cdot 4} = \matriks{3 & 9 & 6 \\ -3 & 6 & 0 \\ -6 & 3 & 12} \]

\subsection{Sifat-sifat Aljabar Penjumlahan dan Perkalian Skalar}
Misalkan $A, B, C \in M_{m \times n}$ adalah matriks-matriks berukuran sama, dan $k, l \in \R$ adalah skalar. Maka berlaku:
\begin{enumerate}
    \item $A+B = B+A$ (Hukum Komutatif untuk Penjumlahan)
    \item $(A+B)+C = A+(B+C)$ (Hukum Asosiatif untuk Penjumlahan)
    \item $A+0 = A$ (Elemen Identitas untuk Penjumlahan, dimana $0$ adalah matriks nol berukuran sama dengan $A$)
    \item $A+(-A) = 0$ (Invers Aditif, dimana $-A = (-1)A$)
    \item $k(A+B) = kA+kB$ (Hukum Distributif Skalar terhadap Penjumlahan Matriks)
    \item $(k+l)A = kA+lA$ (Hukum Distributif Penjumlahan Skalar terhadap Matriks)
    \item $k(lA) = (kl)A$ (Hukum Asosiatif untuk Perkalian Skalar)
    \item $1A = A$ (Elemen Identitas untuk Perkalian Skalar)
\end{enumerate}

\subsection{Perkalian Matriks dengan Matriks}
Misalkan $A=[a_{ij}]$ adalah matriks berukuran $m \times n$ dan $B=[b_{jk}]$ adalah matriks berukuran $n \times p$. Hasil perkalian $A$ dengan $B$, dinotasikan $AB$, adalah matriks $C=[c_{ik}]$ berukuran $m \times p$, dimana entri $c_{ik}$ pada baris ke-$i$ dan kolom ke-$k$ dihitung sebagai berikut:
\[ c_{ik} = \sum_{j=1}^{n} a_{ij}b_{jk} = a_{i1}b_{1k} + a_{i2}b_{2k} + \cdots + a_{in}b_{nk} \]
Artinya, entri $c_{ik}$ adalah hasil kali titik (dot product) dari vektor baris ke-$i$ dari matriks $A$ dengan vektor kolom ke-$k$ dari matriks $B$.
\textbf{Syarat Perkalian Matriks}: Jumlah kolom pada matriks pertama ($A$) harus sama dengan jumlah baris pada matriks kedua ($B$).

\subsubsection{Contoh Perkalian Matriks}
Misalkan $A = \matriks{1 & 2 \\ 3 & 4}$ dan $B = \matriks{5 & 6 \\ 7 & 8}$.
Maka $AB = \matriks{1 & 2 \\ 3 & 4} \matriks{5 & 6 \\ 7 & 8} = \matriks{c_{11} & c_{12} \\ c_{21} & c_{22}}$.
$c_{11} = (1)(5) + (2)(7) = 5 + 14 = 19$.
$c_{12} = (1)(6) + (2)(8) = 6 + 16 = 22$.
$c_{21} = (3)(5) + (4)(7) = 15 + 28 = 43$.
$c_{22} = (3)(6) + (4)(8) = 18 + 32 = 50$.
Jadi, $AB = \matriks{19 & 22 \\ 43 & 50}$.

\subsubsection{Sifat-sifat Perkalian Matriks}
Misalkan $A, B, C$ adalah matriks-matriks dengan ukuran yang sesuai sehingga operasi-operasi berikut terdefinisi, dan $k$ adalah skalar.
\begin{enumerate}
    \item $A(BC) = (AB)C$ (Hukum Asosiatif untuk Perkalian Matriks)
    \item $A(B+C) = AB+AC$ (Hukum Distributif Kiri)
    \item $(A+B)C = AC+BC$ (Hukum Distributif Kanan)
    \item $k(AB) = (kA)B = A(kB)$
    \item $I_m A = A I_n = A$ (jika $A \in M_{m \times n}$, $I_m$ dan $I_n$ matriks identitas yang sesuai)
\end{enumerate}
\textbf{Penting}: Secara umum, $AB \neq BA$. Perkalian matriks tidak komutatif.

\subsection{Pangkat Matriks}
Jika $A$ adalah matriks persegi $n \times n$, maka pangkat non-negatif dari $A$ didefinisikan sebagai:
\begin{itemize}
    \item $A^0 = I_n$ (matriks identitas orde $n$)
    \item $A^p = \underbrace{A A \cdots A}_{p \text{ faktor}}$ (untuk bilangan bulat positif $p$)
\end{itemize}
Jika $A$ invertibel, maka pangkat negatif didefinisikan sebagai $A^{-p} = (A^{-1})^p$.

\subsubsection{Contoh Pangkat Matriks Diagonal}
Misalkan $D = \matriks{d_1 & 0 & \cdots & 0 \\ 0 & d_2 & \cdots & 0 \\ \vdots & \vdots & \ddots & \vdots \\ 0 & 0 & \cdots & d_n}$ adalah matriks diagonal.
Maka untuk bilangan bulat positif $p$,
\[ D^p = \matriks{d_1^p & 0 & \cdots & 0 \\ 0 & d_2^p & \cdots & 0 \\ \vdots & \vdots & \ddots & \vdots \\ 0 & 0 & \cdots & d_n^p} \]

\subsection{Transpose Matriks}
Misalkan $A=[a_{ij}]$ adalah matriks berukuran $m \times n$. Matriks transpose dari $A$, dinotasikan $A^T$ (atau $A'$), adalah matriks berukuran $n \times m$ yang diperoleh dengan menukar baris dan kolom dari $A$. Jadi, entri $(A^T)_{ij} = a_{ji}$.
Baris ke-$i$ dari $A$ menjadi kolom ke-$i$ dari $A^T$.
Kolom ke-$j$ dari $A$ menjadi baris ke-$j$ dari $A^T$.

\subsubsection{Contoh Transpose Matriks}
Jika $A = \matriks{1 & 2 & 3 \\ 4 & 5 & 6}$, maka $A^T = \matriks{1 & 4 \\ 2 & 5 \\ 3 & 6}$.

\subsubsection{Sifat-sifat Transpose Matriks}
Misalkan $A, B$ adalah matriks-matriks dengan ukuran yang sesuai dan $k$ adalah skalar.
\begin{enumerate}
    \item $(A^T)^T = A$
    \item $(A+B)^T = A^T+B^T$
    \item $(kA)^T = kA^T$
    \item $(AB)^T = B^T A^T$ (Urutan dibalik)
\end{enumerate}

\end{document}
