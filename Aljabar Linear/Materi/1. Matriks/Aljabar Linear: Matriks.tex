\documentclass{article}
\usepackage{amsmath} % Untuk lingkungan matriks dan perintah matematika lainnya
\usepackage{amsfonts} % Untuk simbol M_2
\usepackage{geometry} % Untuk mengatur margin
\geometry{a4paper, margin=1in}

\title{Aljabar Linear: Matriks}
\author{Rafi Kamindra 2201006}
\date{} % Kosongkan tanggal jika tidak ingin ditampilkan

\begin{document}
\maketitle
\pagenumbering{gobble} % Menghilangkan nomor halaman jika tidak diinginkan untuk halaman judul

\section*{Latihan}

\subsection*{1. Misalkan}
$A = \begin{pmatrix} 2 & 4 & -2 \\ 1 & 1 & -3 \\ 3 & 2 & 1 \end{pmatrix}$, 
$B = \begin{pmatrix} 1 & 2 \\ -1 & 1 \\ 3 & 2 \end{pmatrix}$, 
$C = \begin{pmatrix} 0 & 1 & 2 \\ 1 & 0 & 3 \\ 1 & -2 & 2 \end{pmatrix}$

\subsubsection*{a. Hitung $A^2 + 2A$. Apakah hasilnya sama dengan $A(A+2I)$?}
\textbf{Jawaban:}

Pertama, kita hitung $A^2$:
\begin{align*} A^2 = A \cdot A &= \begin{pmatrix} 2 & 4 & -2 \\ 1 & 1 & -3 \\ 3 & 2 & 1 \end{pmatrix} \begin{pmatrix} 2 & 4 & -2 \\ 1 & 1 & -3 \\ 3 & 2 & 1 \end{pmatrix} \\ &= \begin{pmatrix} (2)(2)+(4)(1)+(-2)(3) & (2)(4)+(4)(1)+(-2)(2) & (2)(-2)+(4)(-3)+(-2)(1) \\ (1)(2)+(1)(1)+(-3)(3) & (1)(4)+(1)(1)+(-3)(2) & (1)(-2)+(1)(-3)+(-3)(1) \\ (3)(2)+(2)(1)+(1)(3) & (3)(4)+(2)(1)+(1)(2) & (3)(-2)+(2)(-3)+(1)(1) \end{pmatrix} \\ &= \begin{pmatrix} 4+4-6 & 8+4-4 & -4-12-2 \\ 2+1-9 & 4+1-6 & -2-3-3 \\ 6+2+3 & 12+2+2 & -6-6+1 \end{pmatrix} = \begin{pmatrix} 2 & 8 & -18 \\ -6 & -1 & -8 \\ 11 & 16 & -11 \end{pmatrix}\end{align*}

Kemudian, kita hitung $2A$:
\[ 2A = 2 \begin{pmatrix} 2 & 4 & -2 \\ 1 & 1 & -3 \\ 3 & 2 & 1 \end{pmatrix} = \begin{pmatrix} 4 & 8 & -4 \\ 2 & 2 & -6 \\ 6 & 4 & 2 \end{pmatrix} \]

Maka, $A^2 + 2A$:
\begin{align*} A^2 + 2A &= \begin{pmatrix} 2 & 8 & -18 \\ -6 & -1 & -8 \\ 11 & 16 & -11 \end{pmatrix} + \begin{pmatrix} 4 & 8 & -4 \\ 2 & 2 & -6 \\ 6 & 4 & 2 \end{pmatrix} \\ &= \begin{pmatrix} 2+4 & 8+8 & -18-4 \\ -6+2 & -1+2 & -8-6 \\ 11+6 & 16+4 & -11+2 \end{pmatrix} \\ &= \begin{pmatrix} 6 & 16 & -22 \\ -4 & 1 & -14 \\ 17 & 20 & -9 \end{pmatrix}\end{align*}

Sekarang, kita hitung $A(A+2I)$.
Pertama, $2I$, dimana $I$ adalah matriks identitas ordo 3x3:
\[ 2I = 2 \begin{pmatrix} 1 & 0 & 0 \\ 0 & 1 & 0 \\ 0 & 0 & 1 \end{pmatrix} = \begin{pmatrix} 2 & 0 & 0 \\ 0 & 2 & 0 \\ 0 & 0 & 2 \end{pmatrix} \]

Kemudian, $A+2I$:
\begin{align*} A+2I &= \begin{pmatrix} 2 & 4 & -2 \\ 1 & 1 & -3 \\ 3 & 2 & 1 \end{pmatrix} + \begin{pmatrix} 2 & 0 & 0 \\ 0 & 2 & 0 \\ 0 & 0 & 2 \end{pmatrix} \\ &= \begin{pmatrix} 2+2 & 4+0 & -2+0 \\ 1+0 & 1+2 & -3+0 \\ 3+0 & 2+0 & 1+2 \end{pmatrix} \\ &= \begin{pmatrix} 4 & 4 & -2 \\ 1 & 3 & -3 \\ 3 & 2 & 3 \end{pmatrix}\end{align*}

Terakhir, $A(A+2I)$:
\begin{align*} A(A+2I) &= \begin{pmatrix} 2 & 4 & -2 \\ 1 & 1 & -3 \\ 3 & 2 & 1 \end{pmatrix} \begin{pmatrix} 4 & 4 & -2 \\ 1 & 3 & -3 \\ 3 & 2 & 3 \end{pmatrix} \\ &= \begin{pmatrix} (2)(4)+(4)(1)+(-2)(3) & (2)(4)+(4)(3)+(-2)(2) & (2)(-2)+(4)(-3)+(-2)(3) \\ (1)(4)+(1)(1)+(-3)(3) & (1)(4)+(1)(3)+(-3)(2) & (1)(-2)+(1)(-3)+(-3)(3) \\ (3)(4)+(2)(1)+(1)(3) & (3)(4)+(2)(3)+(1)(2) & (3)(-2)+(2)(-3)+(1)(3) \end{pmatrix} \\ &= \begin{pmatrix} 8+4-6 & 8+12-4 & -4-12-6 \\ 4+1-9 & 4+3-6 & -2-3-9 \\ 12+2+3 & 12+6+2 & -6-6+3 \end{pmatrix} = \begin{pmatrix} 6 & 16 & -22 \\ -4 & 1 & -14 \\ 17 & 20 & -9 \end{pmatrix}\end{align*}

\textbf{Kesimpulan:} Ya, hasil $A^2 + 2A$ sama dengan $A(A+2I)$.

\subsubsection*{b. Hitung $C^T A^T$.}
\textbf{Jawaban:}
Kita tahu bahwa $(AC)^T = C^T A^T$.
$C^T = \begin{pmatrix} 0 & 1 & 1 \\ 1 & 0 & -2 \\ 2 & 3 & 2 \end{pmatrix}$, 
$A^T = \begin{pmatrix} 2 & 1 & 3 \\ 4 & 1 & 2 \\ -2 & -3 & 1 \end{pmatrix}$
\begin{align*} C^T A^T &= \begin{pmatrix} 0 & 1 & 1 \\ 1 & 0 & -2 \\ 2 & 3 & 2 \end{pmatrix} \begin{pmatrix} 2 & 1 & 3 \\ 4 & 1 & 2 \\ -2 & -3 & 1 \end{pmatrix} \\ &= \begin{pmatrix} (0)(2)+(1)(4)+(1)(-2) & (0)(1)+(1)(1)+(1)(-3) & (0)(3)+(1)(2)+(1)(1) \\ (1)(2)+(0)(4)+(-2)(-2) & (1)(1)+(0)(1)+(-2)(-3) & (1)(3)+(0)(2)+(-2)(1) \\ (2)(2)+(3)(4)+(2)(-2) & (2)(1)+(3)(1)+(2)(-3) & (2)(3)+(3)(2)+(2)(1) \end{pmatrix} \\ &= \begin{pmatrix} 0+4-2 & 0+1-3 & 0+2+1 \\ 2+0+4 & 1+0+6 & 3+0-2 \\ 4+12-4 & 2+3-6 & 6+6+2 \end{pmatrix} = \begin{pmatrix} 2 & -2 & 3 \\ 6 & 7 & 1 \\ 12 & -1 & 14 \end{pmatrix}\end{align*}

\subsubsection*{c. Baris kedua dari $AC$.}
\textbf{Jawaban:}
\[ AC = \begin{pmatrix} 2 & 4 & -2 \\ 1 & 1 & -3 \\ 3 & 2 & 1 \end{pmatrix} \begin{pmatrix} 0 & 1 & 2 \\ 1 & 0 & 3 \\ 1 & -2 & 2 \end{pmatrix} \]
Untuk baris kedua dari $AC$, kita kalikan baris kedua matriks $A$ dengan setiap kolom matriks $C$:
Baris kedua $A = \begin{pmatrix} 1 & 1 & -3 \end{pmatrix}$
\begin{itemize}
    \item Elemen (2,1) dari $AC = (1)(0) + (1)(1) + (-3)(1) = 0 + 1 - 3 = -2$
    \item Elemen (2,2) dari $AC = (1)(1) + (1)(0) + (-3)(-2) = 1 + 0 + 6 = 7$
    \item Elemen (2,3) dari $AC = (1)(2) + (1)(3) + (-3)(2) = 2 + 3 - 6 = -1$
\end{itemize}
Jadi, baris kedua dari $AC$ adalah $\begin{pmatrix} -2 & 7 & -1 \end{pmatrix}$.

\subsubsection*{d. Kolom ketiga dari $CA$.}
\textbf{Jawaban:}
\[ CA = \begin{pmatrix} 0 & 1 & 2 \\ 1 & 0 & 3 \\ 1 & -2 & 2 \end{pmatrix} \begin{pmatrix} 2 & 4 & -2 \\ 1 & 1 & -3 \\ 3 & 2 & 1 \end{pmatrix} \]
Untuk kolom ketiga dari $CA$, kita kalikan setiap baris matriks $C$ dengan kolom ketiga matriks $A$:
Kolom ketiga $A = \begin{pmatrix} -2 \\ -3 \\ 1 \end{pmatrix}$
\begin{itemize}
    \item Elemen (1,3) dari $CA = (0)(-2) + (1)(-3) + (2)(1) = 0 - 3 + 2 = -1$
    \item Elemen (2,3) dari $CA = (1)(-2) + (0)(-3) + (3)(1) = -2 + 0 + 3 = 1$
    \item Elemen (3,3) dari $CA = (1)(-2) + (-2)(-3) + (2)(1) = -2 + 6 + 2 = 6$
\end{itemize}
Jadi, kolom ketiga dari $CA$ adalah $\begin{pmatrix} -1 \\ 1 \\ 6 \end{pmatrix}$.

\subsubsection*{e. $(AB)^2$.}
\textbf{Jawaban:}
Pertama, hitung $AB$:
\begin{align*} AB &= \begin{pmatrix} 2 & 4 & -2 \\ 1 & 1 & -3 \\ 3 & 2 & 1 \end{pmatrix} \begin{pmatrix} 1 & 2 \\ -1 & 1 \\ 3 & 2 \end{pmatrix} \\ &= \begin{pmatrix} (2)(1)+(4)(-1)+(-2)(3) & (2)(2)+(4)(1)+(-2)(2) \\ (1)(1)+(1)(-1)+(-3)(3) & (1)(2)+(1)(1)+(-3)(2) \\ (3)(1)+(2)(-1)+(1)(3) & (3)(2)+(2)(1)+(1)(2) \end{pmatrix} \\ &= \begin{pmatrix} 2-4-6 & 4+4-4 \\ 1-1-9 & 2+1-6 \\ 3-2+3 & 6+2+2 \end{pmatrix} = \begin{pmatrix} -8 & 4 \\ -9 & -3 \\ 4 & 10 \end{pmatrix}\end{align*}
Karena $AB$ adalah matriks 3x2, maka $(AB)^2$ tidak terdefinisi karena jumlah kolom $AB$ (yaitu 2) tidak sama dengan jumlah baris $AB$ (yaitu 3). Perkalian matriks hanya bisa dilakukan jika jumlah kolom matriks pertama sama dengan jumlah baris matriks kedua.

\subsection*{2. Carilah matriks-matriks $X, Y \in M_2$ sehingga $XY \neq YX$.}
($M_2$ berarti matriks berordo 2x2)

\textbf{Jawaban:}
Ini adalah contoh bahwa perkalian matriks umumnya tidak komutatif.
Misalkan $X = \begin{pmatrix} 1 & 2 \\ 3 & 4 \end{pmatrix}$ dan $Y = \begin{pmatrix} 0 & 1 \\ 1 & 0 \end{pmatrix}$.
\[ XY = \begin{pmatrix} 1 & 2 \\ 3 & 4 \end{pmatrix} \begin{pmatrix} 0 & 1 \\ 1 & 0 \end{pmatrix} = \begin{pmatrix} (1)(0)+(2)(1) & (1)(1)+(2)(0) \\ (3)(0)+(4)(1) & (3)(1)+(4)(0) \end{pmatrix} = \begin{pmatrix} 2 & 1 \\ 4 & 3 \end{pmatrix} \]
\[ YX = \begin{pmatrix} 0 & 1 \\ 1 & 0 \end{pmatrix} \begin{pmatrix} 1 & 2 \\ 3 & 4 \end{pmatrix} = \begin{pmatrix} (0)(1)+(1)(3) & (0)(2)+(1)(4) \\ (1)(1)+(0)(3) & (1)(2)+(0)(4) \end{pmatrix} = \begin{pmatrix} 3 & 4 \\ 1 & 2 \end{pmatrix} \]
Karena $\begin{pmatrix} 2 & 1 \\ 4 & 3 \end{pmatrix} \neq \begin{pmatrix} 3 & 4 \\ 1 & 2 \end{pmatrix}$, maka $XY \neq YX$.
Jadi, $X = \begin{pmatrix} 1 & 2 \\ 3 & 4 \end{pmatrix}$ dan $Y = \begin{pmatrix} 0 & 1 \\ 1 & 0 \end{pmatrix}$ adalah contoh yang memenuhi.

\subsection*{3. Jika $D, E \in M_2$ dan $DE=0$ apakah $D$ dan $E$ harus nol?}
\textbf{Jawaban:}
Tidak harus. Kita bisa menemukan matriks $D$ dan $E$ yang bukan matriks nol, tetapi hasil kalinya adalah matriks nol.
Misalkan $D = \begin{pmatrix} 1 & 0 \\ 0 & 0 \end{pmatrix}$ dan $E = \begin{pmatrix} 0 & 0 \\ 0 & 1 \end{pmatrix}$.
$D \neq 0$ dan $E \neq 0$.
\[ DE = \begin{pmatrix} 1 & 0 \\ 0 & 0 \end{pmatrix} \begin{pmatrix} 0 & 0 \\ 0 & 1 \end{pmatrix} = \begin{pmatrix} (1)(0)+(0)(0) & (1)(0)+(0)(1) \\ (0)(0)+(0)(0) & (0)(0)+(0)(1) \end{pmatrix} = \begin{pmatrix} 0 & 0 \\ 0 & 0 \end{pmatrix} = 0. \]
Contoh lain:
Misalkan $D = \begin{pmatrix} 1 & 1 \\ 1 & 1 \end{pmatrix}$ dan $E = \begin{pmatrix} 1 & -1 \\ -1 & 1 \end{pmatrix}$.
$D \neq 0$ dan $E \neq 0$.
\[ DE = \begin{pmatrix} 1 & 1 \\ 1 & 1 \end{pmatrix} \begin{pmatrix} 1 & -1 \\ -1 & 1 \end{pmatrix} = \begin{pmatrix} (1)(1)+(1)(-1) & (1)(-1)+(1)(1) \\ (1)(1)+(1)(-1) & (1)(-1)+(1)(1) \end{pmatrix} = \begin{pmatrix} 1-1 & -1+1 \\ 1-1 & -1+1 \end{pmatrix} = \begin{pmatrix} 0 & 0 \\ 0 & 0 \end{pmatrix} = 0. \]
Jadi, $D$ dan $E$ tidak harus nol.

\subsection*{4. Diketahui}
$A = \begin{pmatrix} 1 & 0 & 2 \\ 1 & 2 & 3 \\ -1 & 1 & 1 \end{pmatrix}$, 
$B = \begin{pmatrix} 1 \\ -1 \\ 3 \end{pmatrix}$, 
$C = \begin{pmatrix} 2 & 4 & 3 \end{pmatrix}$

\subsubsection*{a. Matriks $U$ sehingga $AU = B$.}
\textbf{Jawaban:}
Jika $AU = B$, maka $U = A^{-1}B$.
Pertama, kita cari invers dari matriks $A$, yaitu $A^{-1}$.
Kita gunakan metode adjoin: $A^{-1} = \frac{1}{\det(A)} \text{adj}(A)$.

Hitung determinan $A$:
\begin{align*} \det(A) &= 1 \begin{vmatrix} 2 & 3 \\ 1 & 1 \end{vmatrix} - 0 \begin{vmatrix} 1 & 3 \\ -1 & 1 \end{vmatrix} + 2 \begin{vmatrix} 1 & 2 \\ -1 & 1 \end{vmatrix} \\ &= 1((2)(1) - (3)(1)) - 0 + 2((1)(1) - (2)(-1)) \\ &= 1(2 - 3) + 2(1 + 2) \\ &= 1(-1) + 2(3) = -1 + 6 = 5 \end{align*}
Karena $\det(A) \neq 0$, maka $A$ memiliki invers.
Sekarang cari matriks kofaktor $K(A)$:
\begin{itemize}
    \item $K_{11} = (-1)^{1+1} \begin{vmatrix} 2 & 3 \\ 1 & 1 \end{vmatrix} = (1)(2-3) = -1$
    \item $K_{12} = (-1)^{1+2} \begin{vmatrix} 1 & 3 \\ -1 & 1 \end{vmatrix} = (-1)(1-(-3)) = (-1)(4) = -4$
    \item $K_{13} = (-1)^{1+3} \begin{vmatrix} 1 & 2 \\ -1 & 1 \end{vmatrix} = (1)(1-(-2)) = (1)(3) = 3$
    \item $K_{21} = (-1)^{2+1} \begin{vmatrix} 0 & 2 \\ 1 & 1 \end{vmatrix} = (-1)(0-2) = (-1)(-2) = 2$
    \item $K_{22} = (-1)^{2+2} \begin{vmatrix} 1 & 2 \\ -1 & 1 \end{vmatrix} = (1)(1-(-2)) = (1)(3) = 3$
    \item $K_{23} = (-1)^{2+3} \begin{vmatrix} 1 & 0 \\ -1 & 1 \end{vmatrix} = (-1)(1-0) = (-1)(1) = -1$
    \item $K_{31} = (-1)^{3+1} \begin{vmatrix} 0 & 2 \\ 2 & 3 \end{vmatrix} = (1)(0-4) = -4$
    \item $K_{32} = (-1)^{3+2} \begin{vmatrix} 1 & 2 \\ 1 & 3 \end{vmatrix} = (-1)(3-2) = (-1)(1) = -1$
    \item $K_{33} = (-1)^{3+3} \begin{vmatrix} 1 & 0 \\ 1 & 2 \end{vmatrix} = (1)(2-0) = 2$
\end{itemize}
Matriks Kofaktor $K(A) = \begin{pmatrix} -1 & -4 & 3 \\ 2 & 3 & -1 \\ -4 & -1 & 2 \end{pmatrix}$
Adjoin $A$, $\text{adj}(A) = K(A)^T = \begin{pmatrix} -1 & 2 & -4 \\ -4 & 3 & -1 \\ 3 & -1 & 2 \end{pmatrix}$

Maka, $A^{-1} = \frac{1}{5} \begin{pmatrix} -1 & 2 & -4 \\ -4 & 3 & -1 \\ 3 & -1 & 2 \end{pmatrix} = \begin{pmatrix} -1/5 & 2/5 & -4/5 \\ -4/5 & 3/5 & -1/5 \\ 3/5 & -1/5 & 2/5 \end{pmatrix}$

Sekarang hitung $U = A^{-1}B$:
\begin{align*} U &= \frac{1}{5} \begin{pmatrix} -1 & 2 & -4 \\ -4 & 3 & -1 \\ 3 & -1 & 2 \end{pmatrix} \begin{pmatrix} 1 \\ -1 \\ 3 \end{pmatrix} \\ &= \frac{1}{5} \begin{pmatrix} (-1)(1)+(2)(-1)+(-4)(3) \\ (-4)(1)+(3)(-1)+(-1)(3) \\ (3)(1)+(-1)(-1)+(2)(3) \end{pmatrix} \\ &= \frac{1}{5} \begin{pmatrix} -1-2-12 \\ -4-3-3 \\ 3+1+6 \end{pmatrix} = \frac{1}{5} \begin{pmatrix} -15 \\ -10 \\ 10 \end{pmatrix} = \begin{pmatrix} -15/5 \\ -10/5 \\ 10/5 \end{pmatrix} = \begin{pmatrix} -3 \\ -2 \\ 2 \end{pmatrix}\end{align*}
Jadi, matriks $U = \begin{pmatrix} -3 \\ -2 \\ 2 \end{pmatrix}$.

\subsubsection*{b. $CAB$}
\textbf{Jawaban:}
Kita sudah punya $A$, $B$, dan $C$.
$C = \begin{pmatrix} 2 & 4 & 3 \end{pmatrix}$ (matriks 1x3)
$A = \begin{pmatrix} 1 & 0 & 2 \\ 1 & 2 & 3 \\ -1 & 1 & 1 \end{pmatrix}$ (matriks 3x3)
$B = \begin{pmatrix} 1 \\ -1 \\ 3 \end{pmatrix}$ (matriks 3x1)

Hitung $CA$ terlebih dahulu:
\begin{align*} CA &= \begin{pmatrix} 2 & 4 & 3 \end{pmatrix} \begin{pmatrix} 1 & 0 & 2 \\ 1 & 2 & 3 \\ -1 & 1 & 1 \end{pmatrix} \\ &= \begin{pmatrix} (2)(1)+(4)(1)+(3)(-1) & (2)(0)+(4)(2)+(3)(1) & (2)(2)+(4)(3)+(3)(1) \end{pmatrix} \\ &= \begin{pmatrix} 2+4-3 & 0+8+3 & 4+12+3 \end{pmatrix} = \begin{pmatrix} 3 & 11 & 19 \end{pmatrix}\end{align*}
(Hasilnya matriks 1x3)

Sekarang hitung $(CA)B$:
\begin{align*} CAB &= \begin{pmatrix} 3 & 11 & 19 \end{pmatrix} \begin{pmatrix} 1 \\ -1 \\ 3 \end{pmatrix} \\ &= \begin{pmatrix} (3)(1)+(11)(-1)+(19)(3) \end{pmatrix} \\ &= \begin{pmatrix} 3 - 11 + 57 \end{pmatrix} = \begin{pmatrix} 49 \end{pmatrix}\end{align*}
Jadi, $CAB = [49]$.

\subsection*{5. Carilah matriks-matriks tak nol $X, Y, Z$ dimana $Y \neq Z$ tetapi $XY = XZ$.}
\textbf{Jawaban:}
Ini menunjukkan bahwa hukum kanselasi tidak selalu berlaku untuk perkalian matriks (jika $XY=XZ$, tidak berarti $Y=Z$, kecuali jika $X$ memiliki invers).
Kita perlu mencari matriks $X$ yang singular (determinannya nol) agar ini bisa terjadi.

Misalkan $X = \begin{pmatrix} 1 & 1 \\ 1 & 1 \end{pmatrix}$. Matriks ini singular karena $\det(X) = (1)(1) - (1)(1) = 0$.
Misalkan $Y = \begin{pmatrix} 1 \\ 0 \end{pmatrix}$ dan $Z = \begin{pmatrix} 0 \\ 1 \end{pmatrix}$. Jelas $Y \neq Z$.
Semua $X, Y, Z$ adalah matriks tak nol.
Agar perkalian $XY$ dan $XZ$ terdefinisi, $Y$ dan $Z$ harus memiliki jumlah baris yang sama dengan jumlah kolom $X$. Karena $X$ adalah 2x2, maka $Y$ dan $Z$ harus matriks 2xk. Kita pilih $k=1$ agar lebih sederhana.
\[ XY = \begin{pmatrix} 1 & 1 \\ 1 & 1 \end{pmatrix} \begin{pmatrix} 1 \\ 0 \end{pmatrix} = \begin{pmatrix} (1)(1)+(1)(0) \\ (1)(1)+(1)(0) \end{pmatrix} = \begin{pmatrix} 1 \\ 1 \end{pmatrix} \]
\[ XZ = \begin{pmatrix} 1 & 1 \\ 1 & 1 \end{pmatrix} \begin{pmatrix} 0 \\ 1 \end{pmatrix} = \begin{pmatrix} (1)(0)+(1)(1) \\ (1)(0)+(1)(1) \end{pmatrix} = \begin{pmatrix} 1 \\ 1 \end{pmatrix} \]
Karena $XY = \begin{pmatrix} 1 \\ 1 \end{pmatrix}$ dan $XZ = \begin{pmatrix} 1 \\ 1 \end{pmatrix}$, maka $XY = XZ$.
Padahal $Y = \begin{pmatrix} 1 \\ 0 \end{pmatrix} \neq Z = \begin{pmatrix} 0 \\ 1 \end{pmatrix}$.

Jadi, contoh matriks yang memenuhi adalah:
$X = \begin{pmatrix} 1 & 1 \\ 1 & 1 \end{pmatrix}$, 
$Y = \begin{pmatrix} 1 \\ 0 \end{pmatrix}$, 
$Z = \begin{pmatrix} 0 \\ 1 \end{pmatrix}$

Contoh lain dengan $Y$ dan $Z$ berordo 2x2:
Misalkan $X = \begin{pmatrix} 1 & 0 \\ 0 & 0 \end{pmatrix}$ (singular).
Misalkan $Y = \begin{pmatrix} 1 & 2 \\ 3 & 4 \end{pmatrix}$
Misalkan $Z = \begin{pmatrix} 1 & 2 \\ 5 & 6 \end{pmatrix}$ (berbeda dari Y hanya pada baris kedua)
$Y \neq Z$.
\[ XY = \begin{pmatrix} 1 & 0 \\ 0 & 0 \end{pmatrix} \begin{pmatrix} 1 & 2 \\ 3 & 4 \end{pmatrix} = \begin{pmatrix} (1)(1)+(0)(3) & (1)(2)+(0)(4) \\ (0)(1)+(0)(3) & (0)(2)+(0)(4) \end{pmatrix} = \begin{pmatrix} 1 & 2 \\ 0 & 0 \end{pmatrix} \]
\[ XZ = \begin{pmatrix} 1 & 0 \\ 0 & 0 \end{pmatrix} \begin{pmatrix} 1 & 2 \\ 5 & 6 \end{pmatrix} = \begin{pmatrix} (1)(1)+(0)(5) & (1)(2)+(0)(6) \\ (0)(1)+(0)(5) & (0)(2)+(0)(6) \end{pmatrix} = \begin{pmatrix} 1 & 2 \\ 0 & 0 \end{pmatrix} \]
Jadi, $XY = XZ$ meskipun $Y \neq Z$.

\end{document}
