\documentclass[12pt, a4paper]{article}
\usepackage{amsmath} % Untuk lingkungan matriks dan perintah matematika lainnya
\usepackage{amsfonts} % Untuk simbol matematika
\usepackage{amssymb} % Untuk simbol \mathbb{R}
\usepackage{geometry} % Untuk mengatur margin
\usepackage[indonesian]{babel} % Mengatur bahasa Indonesia
\usepackage{array} % Untuk kolom tabel yang lebih baik
\usepackage{booktabs} % Untuk garis tabel yang lebih baik
\usepackage{graphicx} % Untuk menyertakan gambar (jika diperlukan)
\usepackage{hyperref} % Untuk hyperlink (jika diperlukan)
\usepackage{amsthm} % Untuk lingkungan definisi dan teorema
\usepackage{nicematrix} % Untuk matriks dengan garis partisi
\usepackage{palatino} % Menggunakan font Palatino untuk tampilan yang lebih menarik

\geometry{a4paper, margin=1in, headheight=15pt} % Mengatur margin halaman

% Definisi lingkungan baru
\theoremstyle{definition} % Gaya standar untuk definisi dan contoh
\newtheorem{definisi}{Definisi}[section]
\newtheorem{contoh}{Contoh}[section]
\theoremstyle{plain} % Gaya standar untuk teorema
\newtheorem{teorema}{Teorema}[section]
\newtheorem{fakta}{Fakta}[section]
\newtheorem{proposisi}{Proposisi}[section]

% Definisi perintah baru untuk kemudahan
\newcommand{\matriks}[1]{\begin{pmatrix} #1 \end{pmatrix}} % Perintah untuk matriks
\newcommand{\R}{\mathbb{R}} % Simbol R untuk bilangan real
\newcommand{\Poly}{\mathbb{P}} % Simbol P untuk polinom
\newcommand{\vektor}[1]{\mathbf{#1}} % Perintah untuk vektor tebal
\newcommand{\koordinat}[2]{[\vektor{#1}]_{#2}} % Notasi koordinat vektor
\newcommand{\nol}{\mathbf{0}} % Vektor nol

\title{Koordinat dan Matriks Transisi}
\author{Rafi Kamindra 2201006}
\date{}

\begin{document}
\maketitle

\section{Koordinat Vektor terhadap Suatu Basis}
Dalam ruang vektor, sebuah basis menyediakan kerangka acuan untuk merepresentasikan setiap vektor secara unik. Representasi ini dikenal sebagai koordinat vektor terhadap basis tersebut.

\begin{definisi}
Misalkan $V$ suatu ruang vektor dan $B = \{\vektor{v}_1, \vektor{v}_2, \dots, \vektor{v}_n\}$ adalah suatu \textbf{basis terurut} untuk $V$. Untuk setiap vektor $\vektor{x} \in V$, terdapat skalar-skalar unik $k_1, k_2, \dots, k_n \in \R$ sedemikian sehingga
\[ \vektor{x} = k_1\vektor{v}_1 + k_2\vektor{v}_2 + \dots + k_n\vektor{v}_n \]
Vektor kolom
\[ \koordinat{x}{B} = \matriks{k_1 \\ k_2 \\ \vdots \\ k_n} \]
disebut \textbf{vektor koordinat} dari $\vektor{x}$ terhadap basis $B$. Skalar-skalar $k_1, \dots, k_n$ disebut \textbf{koordinat} $\vektor{x}$ relatif terhadap $B$.
\end{definisi}

\begin{contoh}[Koordinat terhadap Basis Baku]
Perhatikan basis baku (standar) untuk $\R^3$, yaitu $B = \{\vektor{e}_1=(1,0,0), \vektor{e}_2=(0,1,0), \vektor{e}_3=(0,0,1)\}$.
Koordinat dari vektor $\vektor{x}=(1,2,3)$ terhadap basis $B$ adalah
\[ \koordinat{x}{B} = \matriks{1 \\ 2 \\ 3} \]
karena $\vektor{x} = 1\vektor{e}_1 + 2\vektor{e}_2 + 3\vektor{e}_3 = 1(1,0,0) + 2(0,1,0) + 3(0,0,1)$.
\end{contoh}

\begin{contoh}[Koordinat terhadap Basis Non-Baku]
Tentukan koordinat vektor $\vektor{y}=(4,5,1)$ terhadap basis $B' = \{\vektor{u}_1=(1,0,0), \vektor{u}_2=(2,3,0), \vektor{u}_3=(1,2,1)\}$ untuk $\R^3$.
\textbf{Solusi}:
Kita mencari skalar $\alpha, \beta, \gamma$ yang memenuhi persamaan:
\[ \alpha\vektor{u}_1 + \beta\vektor{u}_2 + \gamma\vektor{u}_3 = \vektor{y} \]
\[ \alpha(1,0,0) + \beta(2,3,0) + \gamma(1,2,1) = (4,5,1) \]
Ini menghasilkan sistem persamaan linear:
\begin{align*}
\alpha + 2\beta + \gamma &= 4 \\
3\beta + 2\gamma &= 5 \\
\gamma &= 1
\end{align*}
Dengan substitusi balik:
Dari persamaan ketiga, $\gamma = 1$.
Substitusi $\gamma=1$ ke persamaan kedua: $3\beta + 2(1) = 5 \Rightarrow 3\beta = 3 \Rightarrow \beta = 1$.
Substitusi $\beta=1$ dan $\gamma=1$ ke persamaan pertama: $\alpha + 2(1) + 1 = 4 \Rightarrow \alpha + 3 = 4 \Rightarrow \alpha = 1$.
Jadi, $\alpha=1, \beta=1, \gamma=1$. Sehingga, vektor koordinat $\vektor{y}$ terhadap basis $B'$ adalah:
\[ \koordinat{y}{B'} = \matriks{1 \\ 1 \\ 1} \]
\end{contoh}

\subsection{Ketunggalan Koordinat}
\begin{teorema}
Misalkan $V$ suatu ruang vektor dengan basis terurut $B = \{\vektor{v}_1, \vektor{v}_2, \dots, \vektor{v}_n\}$. Maka koordinat setiap vektor $\vektor{x} \in V$ terhadap basis $B$ bersifat tunggal.
\end{teorema}
\begin{proof}
Andaikan koordinat $\vektor{x}$ terhadap basis $B$ tidak tunggal. Misalkan terdapat dua representasi:
\[ [\vektor{x}]_B = \matriks{a_1 \\ \vdots \\ a_n} \quad \text{dan} \quad [\vektor{x}]_B = \matriks{b_1 \\ \vdots \\ b_n} \]
Ini berarti
\[ \vektor{x} = a_1\vektor{v}_1 + a_2\vektor{v}_2 + \dots + a_n\vektor{v}_n \]
dan juga
\[ \vektor{x} = b_1\vektor{v}_1 + b_2\vektor{v}_2 + \dots + b_n\vektor{v}_n \]
Dengan menyamakan kedua ekspresi untuk $\vektor{x}$, kita peroleh:
\[ a_1\vektor{v}_1 + \dots + a_n\vektor{v}_n = b_1\vektor{v}_1 + \dots + b_n\vektor{v}_n \]
\[ (a_1-b_1)\vektor{v}_1 + (a_2-b_2)\vektor{v}_2 + \dots + (a_n-b_n)\vektor{v}_n = \nol \]
Karena $B = \{\vektor{v}_1, \dots, \vektor{v}_n\}$ adalah basis, maka himpunan ini bebas linear. Oleh karena itu, satu-satunya cara agar kombinasi linear di atas menghasilkan vektor nol adalah jika semua koefisiennya nol:
\[ a_i - b_i = 0 \quad \text{untuk setiap } i=1, \dots, n \]
Ini berarti $a_i = b_i$ untuk setiap $i$. Jadi, representasi koordinat adalah tunggal.
\end{proof}

\begin{contoh}[Koordinat Vektor Basis]
Misalkan $V$ suatu ruang vektor dengan basis terurut $B = \{\vektor{u}_1, \vektor{u}_2, \dots, \vektor{u}_n\}$.
Perhatikan bahwa untuk setiap vektor basis $\vektor{u}_j \in B$:
\[ \vektor{u}_j = 0\vektor{u}_1 + \dots + 0\vektor{u}_{j-1} + 1\vektor{u}_j + 0\vektor{u}_{j+1} + \dots + 0\vektor{u}_n \]
Jadi, koordinat $\vektor{u}_j$ terhadap basis $B$ adalah vektor kolom dengan entri 1 pada posisi ke-$j$ dan 0 pada posisi lainnya. Sebagai contoh:
\[ [\vektor{u}_1]_B = \matriks{1 \\ 0 \\ \vdots \\ 0}, \quad [\vektor{u}_2]_B = \matriks{0 \\ 1 \\ \vdots \\ 0}, \quad \dots, \quad [\vektor{u}_n]_B = \matriks{0 \\ 0 \\ \vdots \\ 1} \]
Ini adalah vektor-vektor basis standar di $\R^n$.
\end{contoh}

\section{Matriks Transisi}
Ketika kita bekerja dengan ruang vektor, terkadang kita perlu mengubah representasi koordinat suatu vektor dari satu basis ke basis lainnya. Proses ini difasilitasi oleh matriks transisi.

\begin{definisi}
Misalkan $V$ suatu ruang vektor, dan $B = \{\vektor{u}_1, \vektor{u}_2, \dots, \vektor{u}_n\}$ serta $B' = \{\vektor{v}_1, \vektor{v}_2, \dots, \vektor{v}_n\}$ masing-masing adalah basis (terurut) untuk $V$. \textbf{Matriks transisi} $P$ dari basis $B$ ke basis $B'$, dinotasikan $P_{B \to B'}$, adalah matriks $n \times n$ yang kolom-kolomnya adalah vektor-vektor koordinat dari vektor-vektor basis $B$ (basis lama) relatif terhadap basis $B'$ (basis baru):
\[ P_{B \to B'} = \begin{bmatrix} | & | & & | \\ [\vektor{u}_1]_{B'} & [\vektor{u}_2]_{B'} & \cdots & [\vektor{u}_n]_{B'} \\ | & | & & | \end{bmatrix} \]
Lebih lanjut, untuk setiap vektor $\vektor{x} \in V$ berlaku hubungan:
\[ [\vektor{x}]_{B'} = P_{B \to B'} [\vektor{x}]_B \]
Persamaan ini menyatakan bahwa vektor koordinat baru $[\vektor{x}]_{B'}$ dapat diperoleh dengan mengalikan matriks transisi $P_{B \to B'}$ dengan vektor koordinat lama $[\vektor{x}]_B$.
\end{definisi}

\begin{contoh}
Tentukan matriks transisi $P_{B \to B'}$ dari basis $B = \{\vektor{u}_1=(1,0), \vektor{u}_2=(0,1)\}$ (basis standar di $\R^2$) ke basis $B' = \{\vektor{v}_1=(1,1), \vektor{v}_2=(1,2)\}$.
\textbf{Solusi}:
Kolom-kolom matriks $P_{B \to B'}$ adalah $[\vektor{u}_1]_{B'}$ dan $[\vektor{u}_2]_{B'}$.
Kita perlu menyatakan $\vektor{u}_1$ dan $\vektor{u}_2$ sebagai kombinasi linear dari $\vektor{v}_1$ dan $\vektor{v}_2$.

Untuk $\vektor{u}_1 = (1,0)$:
$(1,0) = \alpha\vektor{v}_1 + \beta\vektor{v}_2 = \alpha(1,1) + \beta(1,2) = (\alpha+\beta, \alpha+2\beta)$.
Sistem persamaannya:
\begin{align*} \alpha + \beta &= 1 \\ \alpha + 2\beta &= 0 \end{align*}
Mengurangkan persamaan pertama dari kedua: $\beta = -1$.
Substitusi $\beta=-1$ ke persamaan pertama: $\alpha - 1 = 1 \Rightarrow \alpha = 2$.
Jadi, $[\vektor{u}_1]_{B'} = \matriks{2 \\ -1}$.

Untuk $\vektor{u}_2 = (0,1)$:
$(0,1) = \gamma\vektor{v}_1 + \delta\vektor{v}_2 = \gamma(1,1) + \delta(1,2) = (\gamma+\delta, \gamma+2\delta)$.
Sistem persamaannya:
\begin{align*} \gamma + \delta &= 0 \\ \gamma + 2\delta &= 1 \end{align*}
Mengurangkan persamaan pertama dari kedua: $\delta = 1$.
Substitusi $\delta=1$ ke persamaan pertama: $\gamma + 1 = 0 \Rightarrow \gamma = -1$.
Jadi, $[\vektor{u}_2]_{B'} = \matriks{-1 \\ 1}$.

Oleh karena itu, matriks transisi dari $B$ ke $B'$ adalah:
\[ P_{B \to B'} = \matriks{[\vektor{u}_1]_{B'} & [\vektor{u}_2]_{B'}} = \matriks{2 & -1 \\ -1 & 1} \]
\textbf{Lanjutan Contoh}:
Gunakan matriks transisi ini untuk memperoleh koordinat vektor $\vektor{y}=(3,4)$ terhadap basis $B'$.
Pertama, koordinat $\vektor{y}$ terhadap basis standar $B$ adalah $[\vektor{y}]_B = \matriks{3 \\ 4}$.
Maka,
\[ [\vektor{y}]_{B'} = P_{B \to B'} [\vektor{y}]_B = \matriks{2 & -1 \\ -1 & 1} \matriks{3 \\ 4} = \matriks{2(3) - 1(4) \\ -1(3) + 1(4)} = \matriks{6-4 \\ -3+4} = \matriks{2 \\ 1} \]
Jadi, $(3,4) = 2\vektor{v}_1 + 1\vektor{v}_2 = 2(1,1) + 1(1,2) = (2,2) + (1,2) = (3,4)$. Hasilnya konsisten.

\textbf{Metode Alternatif (Menggunakan Matriks Augmented)}:
Untuk mencari $[\vektor{u}_1]_{B'}$ dan $[\vektor{u}_2]_{B'}$ secara bersamaan, kita dapat menyelesaikan sistem $[ \vektor{v}_1 \vektor{v}_2 | \vektor{u}_1 \vektor{u}_2 ]$:
$\left[ \begin{NiceArray}{cc|cc} 1 & 1 & 1 & 0 \\ 1 & 2 & 0 & 1 \end{NiceArray} \right] \xrightarrow{R_2 \to R_2 - R_1} \left[ \begin{NiceArray}{cc|cc} 1 & 1 & 1 & 0 \\ 0 & 1 & -1 & 1 \end{NiceArray} \right]$
$\xrightarrow{R_1 \to R_1 - R_2} \left[ \begin{NiceArray}{cc|cc} 1 & 0 & 2 & -1 \\ 0 & 1 & -1 & 1 \end{NiceArray} \right]$.
Bagian kanan matriks ini adalah $P_{B \to B'} = \matriks{2 & -1 \\ -1 & 1}$. Ini adalah matriks yang kolom-kolomnya adalah $[\vektor{u}_1]_{B'}$ dan $[\vektor{u}_2]_{B'}$. Perhatikan bahwa notasi $P_{B'B}$ pada slide (halaman 26) mungkin mengacu pada $P_{B \to B'}$ jika $B$ adalah basis lama dan $B'$ adalah basis baru.
\end{contoh}

\subsection{Keterbalikan Matriks Transisi}
\begin{teorema}
Misalkan $V$ suatu ruang vektor, dan $P_{B \to B'}$ adalah matriks transisi dari basis $B$ ke basis $B'$. Maka berlaku:
\begin{enumerate}
    \item Matriks $P_{B \to B'}$ selalu mempunyai invers (invertibel).
    \item Matriks $(P_{B \to B'})^{-1}$ merupakan matriks transisi dari basis $B'$ ke basis $B$, yaitu $(P_{B \to B'})^{-1} = P_{B' \to B}$.
\end{enumerate}
\end{teorema}
Ini berarti jika $[\vektor{x}]_{B'} = P_{B \to B'} [\vektor{x}]_B$, maka $[\vektor{x}]_B = (P_{B \to B'})^{-1} [\vektor{x}]_{B'} = P_{B' \to B} [\vektor{x}]_{B'}$.

\begin{contoh}
Dari contoh sebelumnya, $P_{B \to B'} = \matriks{2 & -1 \\ -1 & 1}$.
Kita cari matriks transisi dari $B' = \{\vektor{v}_1=(1,1), \vektor{v}_2=(1,2)\}$ ke $B = \{\vektor{u}_1=(1,0), \vektor{u}_2=(0,1)\}$.
Kolom-kolom $P_{B' \to B}$ adalah $[\vektor{v}_1]_B$ dan $[\vektor{v}_2]_B$.
$\vektor{v}_1 = (1,1) = 1(1,0) + 1(0,1) = 1\vektor{u}_1 + 1\vektor{u}_2 \Rightarrow [\vektor{v}_1]_B = \matriks{1 \\ 1}$.
$\vektor{v}_2 = (1,2) = 1(1,0) + 2(0,1) = 1\vektor{u}_1 + 2\vektor{u}_2 \Rightarrow [\vektor{v}_2]_B = \matriks{1 \\ 2}$.
Jadi, $P_{B' \to B} = \matriks{1 & 1 \\ 1 & 2}$.
Sekarang kita periksa apakah $(P_{B \to B'})^{-1} = P_{B' \to B}$.
Invers dari $P_{B \to B'} = \matriks{2 & -1 \\ -1 & 1}$ adalah:
\[ (P_{B \to B'})^{-1} = \frac{1}{(2)(1) - (-1)(-1)} \matriks{1 & 1 \\ 1 & 2} = \frac{1}{2-1} \matriks{1 & 1 \\ 1 & 2} = \matriks{1 & 1 \\ 1 & 2} \]
Hasil ini sama dengan $P_{B' \to B}$.
\end{contoh}

\end{document}
