\documentclass{article}
\usepackage{amsmath} % Untuk lingkungan matriks dan perintah matematika lainnya
\usepackage{amsfonts} % Untuk simbol
\usepackage{amssymb} % Untuk simbol
\usepackage{geometry} % Untuk mengatur margin
\usepackage{systeme} % Untuk sistem persamaan linear
\geometry{a4paper, margin=1in}

\title{Aljabar Linear: SPL Homogen}
\author{Rafi Kamindra 2201006}
\date{} % Kosongkan tanggal jika tidak ingin ditampilkan

\begin{document}
\maketitle
\pagenumbering{gobble} % Menghilangkan nomor halaman jika tidak diinginkan untuk halaman judul

\section*{Untuk no 1-4, tentukan basis solusi SPL homogen}

\subsection*{1.
\sysdelim..
\systeme{
2x_1 + 2x_2 + x_3 = 0,
-x_1 - 3x_2 + x_3 = 0,
x_1 - x_2 + 2x_3 = 0
}}
\textbf{Jawaban:}\\
Kita bentuk matriks koefisien dan lakukan Operasi Baris Elementer (OBE):
\[ A = \begin{pmatrix} 2 & 2 & 1 \\ -1 & -3 & 1 \\ 1 & -1 & 2 \end{pmatrix} \]
$R_1 \leftrightarrow R_3$:
\[ \begin{pmatrix} 1 & -1 & 2 \\ -1 & -3 & 1 \\ 2 & 2 & 1 \end{pmatrix} \]
$R_2 \rightarrow R_2 + R_1$:
\[ \begin{pmatrix} 1 & -1 & 2 \\ 0 & -4 & 3 \\ 2 & 2 & 1 \end{pmatrix} \]
$R_3 \rightarrow R_3 - 2R_1$:
\[ \begin{pmatrix} 1 & -1 & 2 \\ 0 & -4 & 3 \\ 0 & 4 & -3 \end{pmatrix} \]
$R_3 \rightarrow R_3 + R_2$:
\[ \begin{pmatrix} 1 & -1 & 2 \\ 0 & -4 & 3 \\ 0 & 0 & 0 \end{pmatrix} \]
Dari baris kedua: $-4x_2 + 3x_3 = 0 \Rightarrow 4x_2 = 3x_3 \Rightarrow x_2 = \frac{3}{4}x_3$.
Dari baris pertama: $x_1 - x_2 + 2x_3 = 0 \Rightarrow x_1 = x_2 - 2x_3 = \frac{3}{4}x_3 - 2x_3 = \frac{3-8}{4}x_3 = -\frac{5}{4}x_3$.
Misalkan $x_3 = t$, maka $x_1 = -\frac{5}{4}t$, $x_2 = \frac{3}{4}t$, $x_3 = t$.
Solusi: $\begin{pmatrix} x_1 \\ x_2 \\ x_3 \end{pmatrix} = t \begin{pmatrix} -5/4 \\ 3/4 \\ 1 \end{pmatrix}$.
Basis solusi adalah $\left\{ \begin{pmatrix} -5/4 \\ 3/4 \\ 1 \end{pmatrix} \right\}$ atau bisa dikalikan dengan 4 menjadi $\left\{ \begin{pmatrix} -5 \\ 3 \\ 4 \end{pmatrix} \right\}$.

\subsection*{2.
\sysdelim..
\systeme{
-x_1 + x_2 - 2x_3 - 2x_4 = 0,
x_1 + x_2 + 2x_3 + x_4 = 0,
2x_1 - x_2 + 2x_3 + 3x_4 = 0
}}
\textbf{Jawaban:}\\
Matriks koefisien:
\[ A = \begin{pmatrix} -1 & 1 & -2 & -2 \\ 1 & 1 & 2 & 1 \\ 2 & -1 & 2 & 3 \end{pmatrix} \]
$R_1 \rightarrow -R_1$:
\[ \begin{pmatrix} 1 & -1 & 2 & 2 \\ 1 & 1 & 2 & 1 \\ 2 & -1 & 2 & 3 \end{pmatrix} \]
$R_2 \rightarrow R_2 - R_1$:
\[ \begin{pmatrix} 1 & -1 & 2 & 2 \\ 0 & 2 & 0 & -1 \\ 2 & -1 & 2 & 3 \end{pmatrix} \]
$R_3 \rightarrow R_3 - 2R_1$:
\[ \begin{pmatrix} 1 & -1 & 2 & 2 \\ 0 & 2 & 0 & -1 \\ 0 & 1 & -2 & -1 \end{pmatrix} \]
$R_2 \leftrightarrow R_3$:
\[ \begin{pmatrix} 1 & -1 & 2 & 2 \\ 0 & 1 & -2 & -1 \\ 0 & 2 & 0 & -1 \end{pmatrix} \]
$R_3 \rightarrow R_3 - 2R_2$:
\[ \begin{pmatrix} 1 & -1 & 2 & 2 \\ 0 & 1 & -2 & -1 \\ 0 & 0 & 4 & 1 \end{pmatrix} \]
Dari baris ketiga: $4x_3 + x_4 = 0 \Rightarrow x_4 = -4x_3$.
Dari baris kedua: $x_2 - 2x_3 - x_4 = 0 \Rightarrow x_2 = 2x_3 + x_4 = 2x_3 - 4x_3 = -2x_3$.
Dari baris pertama: $x_1 - x_2 + 2x_3 + 2x_4 = 0 \Rightarrow x_1 = x_2 - 2x_3 - 2x_4 = -2x_3 - 2x_3 - 2(-4x_3) = -4x_3 + 8x_3 = 4x_3$.
Misalkan $x_3 = t$, maka $x_1 = 4t$, $x_2 = -2t$, $x_3 = t$, $x_4 = -4t$.
Solusi: $\begin{pmatrix} x_1 \\ x_2 \\ x_3 \\ x_4 \end{pmatrix} = t \begin{pmatrix} 4 \\ -2 \\ 1 \\ -4 \end{pmatrix}$.
Basis solusi adalah $\left\{ \begin{pmatrix} 4 \\ -2 \\ 1 \\ -4 \end{pmatrix} \right\}$.

\subsection*{3.
\sysdelim..
\systeme{
x_1 - x_2 + 2x_3 + x_4 = 0
}}
\textbf{Jawaban:}\\
Persamaan tunggal dengan 4 variabel. Ada $4-1=3$ variabel bebas.
$x_1 = x_2 - 2x_3 - x_4$.
Misalkan $x_2 = s$, $x_3 = t$, $x_4 = u$.
Maka $x_1 = s - 2t - u$.
Solusi: $\begin{pmatrix} x_1 \\ x_2 \\ x_3 \\ x_4 \end{pmatrix} = \begin{pmatrix} s - 2t - u \\ s \\ t \\ u \end{pmatrix} = s \begin{pmatrix} 1 \\ 1 \\ 0 \\ 0 \end{pmatrix} + t \begin{pmatrix} -2 \\ 0 \\ 1 \\ 0 \end{pmatrix} + u \begin{pmatrix} -1 \\ 0 \\ 0 \\ 1 \end{pmatrix}$.
Basis solusi adalah $\left\{ \begin{pmatrix} 1 \\ 1 \\ 0 \\ 0 \end{pmatrix}, \begin{pmatrix} -2 \\ 0 \\ 1 \\ 0 \end{pmatrix}, \begin{pmatrix} -1 \\ 0 \\ 0 \\ 1 \end{pmatrix} \right\}$.

\subsection*{4.
\sysdelim..
\systeme{
x_1 - x_2 + 2x_3 + x_4 - 3x_5 = 0,
x_1 + x_2 + x_3 + 3x_4 + 2x_5 = 0,
x_1 - 2x_2 - x_3 + 2x_5 = 0
}}
\textbf{Jawaban:}\\
Matriks koefisien:
\[ A = \begin{pmatrix} 1 & -1 & 2 & 1 & -3 \\ 1 & 1 & 1 & 3 & 2 \\ 1 & -2 & -1 & 0 & 2 \end{pmatrix} \]
(Perhatikan $x_4$ tidak ada di persamaan ketiga, jadi koefisiennya 0)
\[ A = \begin{pmatrix} 1 & -1 & 2 & 1 & -3 \\ 1 & 1 & 1 & 3 & 2 \\ 1 & -2 & -1 & 0 & 2 \end{pmatrix} \]
$R_2 \rightarrow R_2 - R_1$:
\[ \begin{pmatrix} 1 & -1 & 2 & 1 & -3 \\ 0 & 2 & -1 & 2 & 5 \\ 1 & -2 & -1 & 0 & 2 \end{pmatrix} \]
$R_3 \rightarrow R_3 - R_1$:
\[ \begin{pmatrix} 1 & -1 & 2 & 1 & -3 \\ 0 & 2 & -1 & 2 & 5 \\ 0 & -1 & -3 & -1 & 5 \end{pmatrix} \]
$R_2 \leftrightarrow R_3$ (untuk mendapatkan 1 utama lebih mudah):
\[ \begin{pmatrix} 1 & -1 & 2 & 1 & -3 \\ 0 & -1 & -3 & -1 & 5 \\ 0 & 2 & -1 & 2 & 5 \end{pmatrix} \]
$R_2 \rightarrow -R_2$:
\[ \begin{pmatrix} 1 & -1 & 2 & 1 & -3 \\ 0 & 1 & 3 & 1 & -5 \\ 0 & 2 & -1 & 2 & 5 \end{pmatrix} \]
$R_3 \rightarrow R_3 - 2R_2$:
\[ \begin{pmatrix} 1 & -1 & 2 & 1 & -3 \\ 0 & 1 & 3 & 1 & -5 \\ 0 & 0 & -1-6 & 2-2 & 5+10 \end{pmatrix} = \begin{pmatrix} 1 & -1 & 2 & 1 & -3 \\ 0 & 1 & 3 & 1 & -5 \\ 0 & 0 & -7 & 0 & 15 \end{pmatrix} \]
Dari baris ketiga: $-7x_3 + 15x_5 = 0 \Rightarrow 7x_3 = 15x_5 \Rightarrow x_3 = \frac{15}{7}x_5$.
Dari baris kedua: $x_2 + 3x_3 + x_4 - 5x_5 = 0 \Rightarrow x_2 = -3x_3 - x_4 + 5x_5 = -3(\frac{15}{7}x_5) - x_4 + 5x_5 = -\frac{45}{7}x_5 - x_4 + \frac{35}{7}x_5 = -x_4 - \frac{10}{7}x_5$.
Dari baris pertama: $x_1 - x_2 + 2x_3 + x_4 - 3x_5 = 0 \Rightarrow x_1 = x_2 - 2x_3 - x_4 + 3x_5$
$x_1 = (-x_4 - \frac{10}{7}x_5) - 2(\frac{15}{7}x_5) - x_4 + 3x_5 = -x_4 - \frac{10}{7}x_5 - \frac{30}{7}x_5 - x_4 + \frac{21}{7}x_5 = -2x_4 - \frac{19}{7}x_5$.
Variabel bebas adalah $x_4$ dan $x_5$. Misalkan $x_4 = s$ dan $x_5 = t$.
$x_1 = -2s - \frac{19}{7}t$
$x_2 = -s - \frac{10}{7}t$
$x_3 = \frac{15}{7}t$
$x_4 = s$
$x_5 = t$
Solusi: $\begin{pmatrix} x_1 \\ x_2 \\ x_3 \\ x_4 \\ x_5 \end{pmatrix} = s \begin{pmatrix} -2 \\ -1 \\ 0 \\ 1 \\ 0 \end{pmatrix} + t \begin{pmatrix} -19/7 \\ -10/7 \\ 15/7 \\ 0 \\ 1 \end{pmatrix}$.
Basis solusi adalah $\left\{ \begin{pmatrix} -2 \\ -1 \\ 0 \\ 1 \\ 0 \end{pmatrix}, \begin{pmatrix} -19/7 \\ -10/7 \\ 15/7 \\ 0 \\ 1 \end{pmatrix} \right\}$ atau bisa dikalikan vektor kedua dengan 7 menjadi $\left\{ \begin{pmatrix} -2 \\ -1 \\ 0 \\ 1 \\ 0 \end{pmatrix}, \begin{pmatrix} -19 \\ -10 \\ 15 \\ 0 \\ 7 \end{pmatrix} \right\}$.

\subsection*{5. Tentukan semua bilangan real $k$, supaya SPL berikut mempunyai solusi banyak}
{
\sysdelim..
\systeme{
kx_1 - 2x_2 + x_3 = 0,
x_1 + 2x_2 + 3x_3 = 0,
-x_1 - x_2 + x_3 = 0
}}\\
\textbf{Jawaban:}\\
Agar SPL homogen mempunyai solusi banyak (solusi non-trivial), determinan matriks koefisiennya harus nol.
\[ A = \begin{pmatrix} k & -2 & 1 \\ 1 & 2 & 3 \\ -1 & -1 & 1 \end{pmatrix} \]
\begin{align*} \det(A) &= k \begin{vmatrix} 2 & 3 \\ -1 & 1 \end{vmatrix} - (-2) \begin{vmatrix} 1 & 3 \\ -1 & 1 \end{vmatrix} + 1 \begin{vmatrix} 1 & 2 \\ -1 & -1 \end{vmatrix} \\ &= k(2 - (-3)) + 2(1 - (-3)) + 1(-1 - (-2)) \\ &= k(2+3) + 2(1+3) + 1(-1+2) \\ &= 5k + 2(4) + 1(1) \\ &= 5k + 8 + 1 \\ &= 5k + 9 \end{align*}
Agar solusi banyak, $\det(A) = 0$:
$5k + 9 = 0 \Rightarrow 5k = -9 \Rightarrow k = -\frac{9}{5}$.
Jadi, SPL mempunyai solusi banyak jika $k = -9/5$.

\subsection*{6. Tentukan nilai-nilai $a, b, c$ sehingga $[1 \quad a \quad b \quad c]^T$ merupakan solusi SPL
\sysdelim..
\systeme{
x_1 - x_2 + 2x_3 + 2x_4 = 0,
-x_1 + x_2 - 2x_3 - x_4 = 0,
x_1 - 2x_2 + x_3 = 0
}}
\textbf{Jawaban:}\\
Solusi yang diberikan adalah $x_1 = 1, x_2 = a, x_3 = b, x_4 = c$. Substitusikan ke dalam SPL: \\
\begin{align*}
    1 - a + 2b + 2c &= 0 \quad (*) \\
    -1 + a - 2b - c &= 0 \quad (**) \\
    1 - 2a + b &= 0 \quad (***)
\end{align*}

Dari $(***)$, $b = 2a - 1$.
Substitusikan $b$ ke $(*)$:\\
\begin{align*}
    1 - a + 2(2a-1) + 2c &= 0 \\
    1 - a + 4a - 2 + 2c &= 0 \\
    3a - 1 + 2c &= 0 \quad (****)
\end{align*}

Substitusikan $b$ ke $(**)$:\\
\begin{align*}
    -1 + a - 2(2a-1) - c &= 0 \\
    -1 + a - 4a + 2 - c &= 0 \\
    -3a + 1 - c &= 0 \\
    c &= 1 - 3a \quad (*****)
\end{align*}

Substitusikan $c$ dari $(*****)$ ke $(****)$:\\
\begin{align*}
    3a - 1 + 2(1-3a) &= 0 \\
    3a - 1 + 2 - 6a &= 0 \\
    -3a + 1 &= 0 \\
    -3a &= -1 \\
    a &= \frac{1}{3}
\end{align*}

Sekarang cari $b$ dan $c$:\\
\begin{align*}
    b &= 2a - 1 = 2(\frac{1}{3}) - 1 = \frac{2}{3} - 1 = -\frac{1}{3} \\
    c &= 1 - 3a = 1 - 3(\frac{1}{3}) = 1 - 1 = 0
\end{align*}

Jadi, nilai-nilai yang memenuhi adalah $a = 1/3$, $b = -1/3$, dan $c = 0$.
Solusinya adalah $[1 \quad 1/3 \quad -1/3 \quad 0]^T$.

\end{document}
