\documentclass{article}
\usepackage{amsmath} % Untuk lingkungan matriks dan perintah matematika lainnya
\usepackage{amsfonts} % Untuk simbol
\usepackage{amssymb} % Untuk simbol \mathbb{R}
\usepackage{geometry} % Untuk mengatur margin
\geometry{a4paper, margin=1in}

\title{Aljabar Linear: Diagonalisasi Matriks}
\author{Rafi Kamindra 2201006}
\date{} % Kosongkan tanggal jika tidak ingin ditampilkan

\newcommand{\vektor}[1]{\mathbf{#1}} % Perintah untuk vektor
\newcommand{\R}{\mathbb{R}} % Simbol R
\newcommand{\matriks}[1]{\begin{pmatrix} #1 \end{pmatrix}} % Perintah untuk matriks

\begin{document}
\maketitle
\pagenumbering{gobble} % Menghilangkan nomor halaman jika tidak diinginkan untuk halaman judul

\section*{Problem}
\textbf{Periksa, apakah matriks-matriks berikut terdiagonalkan. Jika ya, berikan matriks pendiagonalnya.}

\subsection*{$A = \matriks{1 & 1 & 1 \\ 0 & 1 & 0 \\ 0 & 1 & 2}$}
\textbf{Jawaban:}\\
Nilai eigen (karena matriks segitiga atas): $\lambda_1 = 1$ (multiplisitas aljabar 2), $\lambda_2 = 2$ (multiplisitas aljabar 1).

Untuk $\lambda_1 = 1$: $(A - I)\vektor{x} = \vektor{0}$
\[ \matriks{0 & 1 & 1 \\ 0 & 0 & 0 \\ 0 & 1 & 1} \matriks{x_1 \\ x_2 \\ x_3} = \matriks{0 \\ 0 \\ 0} \]
$x_2 + x_3 = 0 \Rightarrow x_2 = -x_3$. $x_1$ adalah variabel bebas.
Misalkan $x_1 = s$, $x_3 = t$. Maka $x_2 = -t$.
Vektor eigen: $\vektor{x} = \matriks{s \\ -t \\ t} = s\matriks{1 \\ 0 \\ 0} + t\matriks{0 \\ -1 \\ 1}$.
Basis untuk ruang eigen $E_1$: $B_{E_1} = \left\{ \matriks{1 \\ 0 \\ 0}, \matriks{0 \\ -1 \\ 1} \right\}$. Dimensi $E_1 = 2$.
Multiplisitas geometris untuk $\lambda_1=1$ adalah 2, sama dengan multiplisitas aljabarnya.

Untuk $\lambda_2 = 2$: $(A - 2I)\vektor{x} = \vektor{0}$
\[ \matriks{-1 & 1 & 1 \\ 0 & -1 & 0 \\ 0 & 1 & 0} \matriks{x_1 \\ x_2 \\ x_3} = \matriks{0 \\ 0 \\ 0} \]
Dari baris kedua (atau ketiga): $-x_2 = 0 \Rightarrow x_2 = 0$.
Dari baris pertama: $-x_1 + x_2 + x_3 = 0 \Rightarrow -x_1 + 0 + x_3 = 0 \Rightarrow x_1 = x_3$.
Misalkan $x_3 = t$. Maka $x_1 = t$.
Vektor eigen: $\vektor{x} = t\matriks{1 \\ 0 \\ 1}$.
Basis untuk ruang eigen $E_2$: $B_{E_2} = \left\{ \matriks{1 \\ 0 \\ 1} \right\}$. Dimensi $E_2 = 1$.
Multiplisitas geometris untuk $\lambda_2=2$ adalah 1, sama dengan multiplisitas aljabarnya.

Karena untuk setiap nilai eigen, multiplisitas geometris sama dengan multiplisitas aljabarnya, dan jumlah dimensi ruang eigen ($2+1=3$) sama dengan ukuran matriks, maka matriks $A$ \textbf{terdiagonalkan}.
Matriks pendiagonal $P$ dibentuk oleh vektor-vektor basis ruang eigen:
$P = \matriks{1 & 0 & 1 \\ 0 & -1 & 0 \\ 0 & 1 & 1}$.
Matriks diagonalnya adalah $D = \matriks{1 & 0 & 0 \\ 0 & 1 & 0 \\ 0 & 0 & 2}$. (Urutan nilai eigen di $D$ sesuai urutan kolom vektor eigen di $P$).

\subsection*{$B = \matriks{1 & 3 & 0 \\ 0 & -2 & 0 \\ 0 & 6 & 1}$}
\textbf{Jawaban:}\\
Nilai eigen (karena matriks segitiga bawah): $\lambda_1 = 1$ (multiplisitas aljabar 2), $\lambda_2 = -2$ (multiplisitas aljabar 1).

Untuk $\lambda_1 = 1$: $(B - I)\vektor{x} = \vektor{0}$
\[ \matriks{0 & 3 & 0 \\ 0 & -3 & 0 \\ 0 & 6 & 0} \matriks{x_1 \\ x_2 \\ x_3} = \matriks{0 \\ 0 \\ 0} \]
Dari semua baris: $3x_2 = 0 \Rightarrow x_2 = 0$.
$x_1$ dan $x_3$ adalah variabel bebas. Misalkan $x_1 = s$, $x_3 = t$.
Vektor eigen: $\vektor{x} = \matriks{s \\ 0 \\ t} = s\matriks{1 \\ 0 \\ 0} + t\matriks{0 \\ 0 \\ 1}$.
Basis untuk ruang eigen $E_1$: $B_{E_1} = \left\{ \matriks{1 \\ 0 \\ 0}, \matriks{0 \\ 0 \\ 1} \right\}$. Dimensi $E_1 = 2$.
Multiplisitas geometris untuk $\lambda_1=1$ adalah 2, sama dengan multiplisitas aljabarnya.

Untuk $\lambda_2 = -2$: $(B - (-2)I)\vektor{x} = (B+2I)\vektor{x} = \vektor{0}$
\[ \matriks{3 & 3 & 0 \\ 0 & 0 & 0 \\ 0 & 6 & 3} \matriks{x_1 \\ x_2 \\ x_3} = \matriks{0 \\ 0 \\ 0} \]
Dari baris pertama: $3x_1 + 3x_2 = 0 \Rightarrow x_1 = -x_2$.
Dari baris ketiga: $6x_2 + 3x_3 = 0 \Rightarrow 3x_3 = -6x_2 \Rightarrow x_3 = -2x_2$.
Misalkan $x_2 = t$. Maka $x_1 = -t$, $x_3 = -2t$.
Vektor eigen: $\vektor{x} = t\matriks{-1 \\ 1 \\ -2}$.
Basis untuk ruang eigen $E_{-2}$: $B_{E_{-2}} = \left\{ \matriks{-1 \\ 1 \\ -2} \right\}$. Dimensi $E_{-2} = 1$.
Multiplisitas geometris untuk $\lambda_2=-2$ adalah 1, sama dengan multiplisitas aljabarnya.

Karena untuk setiap nilai eigen, multiplisitas geometris sama dengan multiplisitas aljabarnya, dan jumlah dimensi ruang eigen ($2+1=3$) sama dengan ukuran matriks, maka matriks $B$ \textbf{terdiagonalkan}.
Matriks pendiagonal $P$:
$P = \matriks{1 & 0 & -1 \\ 0 & 0 & 1 \\ 0 & 1 & -2}$.
Matriks diagonalnya adalah $D = \matriks{1 & 0 & 0 \\ 0 & 1 & 0 \\ 0 & 0 & -2}$.

\subsection*{$C = \matriks{1 & 2 & 2 \\ 0 & 3 & 0 \\ 0 & 2 & 3}$}
\textbf{Jawaban:}\\
Nilai eigen: $\lambda_1 = 1$ (multiplisitas aljabar 1), $\lambda_2 = 3$ (multiplisitas aljabar 2).

Untuk $\lambda_1 = 1$: Basis ruang eigen $E_1 = \left\{ \matriks{1 \\ 0 \\ 0} \right\}$. Dimensi $E_1 = 1$.
Multiplisitas geometris untuk $\lambda_1=1$ adalah 1, sama dengan multiplisitas aljabarnya.

Untuk $\lambda_2 = 3$: $(C - 3I)\vektor{x} = \vektor{0}$
\[ \matriks{-2 & 2 & 2 \\ 0 & 0 & 0 \\ 0 & 2 & 0} \matriks{x_1 \\ x_2 \\ x_3} = \matriks{0 \\ 0 \\ 0} \]
Dari baris ketiga: $2x_2 = 0 \Rightarrow x_2 = 0$.
Dari baris pertama: $-2x_1 + 2x_2 + 2x_3 = 0 \Rightarrow -2x_1 + 0 + 2x_3 = 0 \Rightarrow x_1 = x_3$.
Misalkan $x_3 = t$. Maka $x_1 = t$.
Vektor eigen: $\vektor{x} = t\matriks{1 \\ 0 \\ 1}$.
Basis untuk ruang eigen $E_3$: $B_{E_3} = \left\{ \matriks{1 \\ 0 \\ 1} \right\}$. Dimensi $E_3 = 1$.
Multiplisitas geometris untuk $\lambda_2=3$ adalah 1, sedangkan multiplisitas aljabarnya adalah 2.

Karena untuk nilai eigen $\lambda=3$, multiplisitas geometris (1) tidak sama dengan multiplisitas aljabar (2), maka matriks $C$ \textbf{tidak terdiagonalkan}.

\subsection*{$D = \matriks{3 & 3 & 3 \\ 3 & 3 & 3 \\ 3 & 3 & 3}$}
\textbf{Jawaban:}\\
Persamaan karakteristik: $(9-\lambda)\lambda^2 = 0$.
Nilai eigen: $\lambda_1 = 9$ (multiplisitas aljabar 1), $\lambda_2 = 0$ (multiplisitas aljabar 2).

Untuk $\lambda_1 = 9$: Basis ruang eigen $E_9 = \left\{ \matriks{1 \\ 1 \\ 1} \right\}$. Dimensi $E_9 = 1$.
Multiplisitas geometris untuk $\lambda_1=9$ adalah 1, sama dengan multiplisitas aljabarnya.

Untuk $\lambda_2 = 0$: Basis ruang eigen $E_0 = \left\{ \matriks{-1 \\ 1 \\ 0}, \matriks{-1 \\ 0 \\ 1} \right\}$. Dimensi $E_0 = 2$.
Multiplisitas geometris untuk $\lambda_2=0$ adalah 2, sama dengan multiplisitas aljabarnya.

Karena untuk setiap nilai eigen, multiplisitas geometris sama dengan multiplisitas aljabarnya, dan jumlah dimensi ruang eigen ($1+2=3$) sama dengan ukuran matriks, maka matriks $D$ \textbf{terdiagonalkan}.
Matriks pendiagonal $P$:
$P = \matriks{1 & -1 & -1 \\ 1 & 1 & 0 \\ 1 & 0 & 1}$.
Matriks diagonalnya adalah $D_{diag} = \matriks{9 & 0 & 0 \\ 0 & 0 & 0 \\ 0 & 0 & 0}$.

\subsection*{$E = \matriks{2 & -1 & 0 \\ -1 & 2 & 1 \\ 1 & 1 & 1}$}
\textbf{Jawaban:}\\
Nilai eigen: $\lambda_1 = 2, \lambda_2 = 0, \lambda_3 = 3$. Semua nilai eigen berbeda, sehingga multiplisitas aljabar masing-masing adalah 1.
Jika semua nilai eigen berbeda, matriks pasti terdiagonalkan, dan multiplisitas geometris untuk setiap nilai eigen adalah 1.

Untuk $\lambda_1 = 2$: Basis $E_2 = \left\{ \matriks{1 \\ 0 \\ 1} \right\}$. Dimensi $E_2 = 1$.
Untuk $\lambda_2 = 0$: Basis $E_0 = \left\{ \matriks{-1 \\ -2 \\ 3} \right\}$. Dimensi $E_0 = 1$.
Untuk $\lambda_3 = 3$: Basis $E_3 = \left\{ \matriks{-1 \\ 1 \\ 0} \right\}$. Dimensi $E_3 = 1$.

Karena matriks $3 \times 3$ ini memiliki 3 vektor eigen yang linear independen (karena berasal dari nilai eigen yang berbeda), maka matriks $E$ \textbf{terdiagonalkan}.
Matriks pendiagonal $P$:
$P = \matriks{1 & -1 & -1 \\ 0 & -2 & 1 \\ 1 & 3 & 0}$.
Matriks diagonalnya adalah $D_{diag} = \matriks{2 & 0 & 0 \\ 0 & 0 & 0 \\ 0 & 0 & 3}$.

\subsection*{$F = \matriks{1 & 3 & 1 \\ -1 & 2 & 1 \\ 1 & 8 & 3}$}
\textbf{Jawaban:}\\
Nilai eigen: $\lambda_1 = 0, \lambda_2 = 1, \lambda_3 = 5$. Semua nilai eigen berbeda, sehingga multiplisitas aljabar masing-masing adalah 1.
Jika semua nilai eigen berbeda, matriks pasti terdiagonalkan, dan multiplisitas geometris untuk setiap nilai eigen adalah 1.

Untuk $\lambda_1 = 0$: Basis $E_0 = \left\{ \matriks{1 \\ -2 \\ 5} \right\}$. Dimensi $E_0 = 1$.
Untuk $\lambda_2 = 1$: Basis $E_1 = \left\{ \matriks{-2 \\ 1 \\ -3} \right\}$. Dimensi $E_1 = 1$.
Untuk $\lambda_3 = 5$: Basis $E_5 = \left\{ \matriks{2 \\ 1 \\ 5} \right\}$. Dimensi $E_5 = 1$.

Karena matriks $3 \times 3$ ini memiliki 3 vektor eigen yang linear independen, maka matriks $F$ \textbf{terdiagonalkan}.
Matriks pendiagonal $P$:
$P = \matriks{1 & -2 & 2 \\ -2 & 1 & 1 \\ 5 & -3 & 5}$.
Matriks diagonalnya adalah $D_{diag} = \matriks{0 & 0 & 0 \\ 0 & 1 & 0 \\ 0 & 0 & 5}$.

\end{document}
