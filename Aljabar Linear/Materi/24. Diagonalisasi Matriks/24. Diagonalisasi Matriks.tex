\documentclass[12pt, a4paper]{article}
\usepackage{amsmath} % Untuk lingkungan matriks dan perintah matematika lainnya
\usepackage{amsfonts} % Untuk simbol matematika
\usepackage{amssymb} % Untuk simbol \mathbb{R}
\usepackage{geometry} % Untuk mengatur margin
\usepackage[indonesian]{babel} % Mengatur bahasa Indonesia
\usepackage{array} % Untuk kolom tabel yang lebih baik
\usepackage{booktabs} % Untuk garis tabel yang lebih baik
\usepackage{graphicx} % Untuk menyertakan gambar (jika diperlukan)
\usepackage{tikz} % Untuk menggambar diagram
\usetikzlibrary{arrows.meta, positioning}
\usepackage[hidelinks]{hyperref} % Untuk hyperlink tanpa kotak
\usepackage{amsthm} % Untuk lingkungan definisi dan teorema
\usepackage{nicematrix} % Untuk matriks dengan garis partisi, jika diperlukan
\usepackage{palatino} % Menggunakan font Palatino untuk tampilan yang lebih menarik
\usepackage{setspace} % Untuk mengatur spasi antar baris jika perlu
\usepackage{titlesec} % Untuk kustomisasi judul seksi
\usepackage{enumitem} % Untuk kustomisasi daftar

\geometry{a4paper, margin=1in, headheight=15pt, footskip=30pt} % Mengatur margin halaman

% Kustomisasi Judul Seksi
\titleformat{\section}{\Large\bfseries\sffamily\color{blue!70!black}}{\thesection}{1em}{}
\titleformat{\subsection}{\large\bfseries\sffamily\color{blue!60!black}}{\thesubsection}{1em}{}
\titleformat{\subsubsection}{\normalsize\bfseries\sffamily\color{blue!50!black}}{\thesubsubsection}{1em}{}

% Definisi lingkungan baru
\theoremstyle{definition} % Gaya standar untuk definisi dan contoh
\newtheorem{definisi}{Definisi}[section]
\newtheorem{contoh}{Contoh}[section]
\newtheorem{catatan}{Catatan Penting}[section]
\theoremstyle{plain} % Gaya standar untuk teorema
\newtheorem{teorema}{Teorema}[section]
\newtheorem{proposisi}{Proposisi}[section]
\newtheorem{akibat}{Akibat}[section]

% Definisi perintah baru untuk kemudahan
\newcommand{\matriks}[1]{\begin{pmatrix} #1 \end{pmatrix}} % Perintah untuk matriks
\newcommand{\R}{\mathbb{R}} % Simbol R untuk bilangan real
\newcommand{\vektor}[1]{\mathbf{#1}} % Perintah untuk vektor tebal
\newcommand{\nol}{\mathbf{0}} % Vektor nol
\newcommand{\dettext}[1]{\operatorname{det}(#1)} % Notasi determinan det(A)
\newcommand{\I}{\mathbf{I}} % Matriks identitas
\newcommand{\dimV}{\operatorname{dim}} % Dimensi

\title{\textbf{Diagonalisasi Matriks}}
\author{Rafi Kamindra 2201006}
\date{}

\begin{document}
\maketitle

\section{Pengantar Diagonalisasi}
Diagonalisasi adalah proses mengubah suatu matriks persegi menjadi matriks diagonal. Proses ini sangat penting dalam aljabar linear karena matriks diagonal memiliki sifat-sifat komputasi yang sangat sederhana. Misalnya, menghitung pangkat dari matriks diagonal jauh lebih mudah daripada menghitung pangkat dari matriks non-diagonal. Banyak masalah dalam sains dan teknik, seperti solusi sistem persamaan diferensial linear dan analisis rantai Markov, menjadi lebih mudah diselesaikan jika matriks yang terlibat dapat didiagonalkan.

\begin{definisi}[Matriks Similar]
Dua matriks persegi $A$ dan $B$ berukuran $n \times n$ dikatakan \textbf{similar} jika terdapat sebuah matriks invertibel (dapat dibalik) $P$ berukuran $n \times n$ sedemikian sehingga:
\[ B = P^{-1}AP \]
Transformasi dari $A$ ke $B$ ini disebut \textbf{transformasi similaritas}.
\end{definisi}

\begin{definisi}[Matriks Terdiagonalkan]
Sebuah matriks persegi $A$ berukuran $n \times n$ dikatakan \textbf{terdiagonalkan} (diagonalizable) jika $A$ similar dengan sebuah matriks diagonal $D$. Artinya, terdapat sebuah matriks invertibel $P$ sedemikian sehingga:
\[ P^{-1}AP = D = \matriks{\lambda_1 & 0 & \cdots & 0 \\ 0 & \lambda_2 & \cdots & 0 \\ \vdots & \vdots & \ddots & \vdots \\ 0 & 0 & \cdots & \lambda_n} \]
Matriks $P$ disebut \textbf{matriks pendiagonal} (diagonalizing matrix) untuk $A$.
\end{definisi}
Pertanyaan kunci yang akan kita jawab adalah:
\begin{enumerate}
    \item Kondisi apa yang harus dipenuhi oleh matriks $A$ agar dapat didiagonalkan?
    \item Jika $A$ dapat didiagonalkan, bagaimana cara menemukan matriks $P$ dan $D$?
\end{enumerate}

\section{Kondisi untuk Diagonalisasi}
Hubungan antara diagonalisasi dengan nilai eigen dan vektor eigen sangat erat. Misalkan $A$ terdiagonalkan, maka ada $P$ dan $D$ sehingga $AP = PD$.
Misalkan $P = \begin{bmatrix} \vektor{p}_1 & \vektor{p}_2 & \cdots & \vektor{p}_n \end{bmatrix}$ (dimana $\vektor{p}_i$ adalah vektor kolom ke-$i$ dari $P$) dan $D$ adalah matriks diagonal dengan entri diagonal $\lambda_1, \lambda_2, \dots, \lambda_n$.
Maka,
\[ A \begin{bmatrix} \vektor{p}_1 & \vektor{p}_2 & \cdots & \vektor{p}_n \end{bmatrix} = \begin{bmatrix} \vektor{p}_1 & \vektor{p}_2 & \cdots & \vektor{p}_n \end{bmatrix} \matriks{\lambda_1 & 0 & \cdots & 0 \\ 0 & \lambda_2 & \cdots & 0 \\ \vdots & \vdots & \ddots & \vdots \\ 0 & 0 & \cdots & \lambda_n} \]
\[ \begin{bmatrix} A\vektor{p}_1 & A\vektor{p}_2 & \cdots & A\vektor{p}_n \end{bmatrix} = \begin{bmatrix} \lambda_1\vektor{p}_1 & \lambda_2\vektor{p}_2 & \cdots & \lambda_n\vektor{p}_n \end{bmatrix} \]
Dengan menyamakan kolom-kolomnya, kita mendapatkan $A\vektor{p}_i = \lambda_i\vektor{p}_i$ untuk setiap $i=1, \dots, n$.
Ini adalah persamaan nilai eigen! Persamaan ini menunjukkan bahwa kolom-kolom dari matriks pendiagonal $P$ adalah vektor-vektor eigen dari $A$, dan entri-entri diagonal dari $D$ adalah nilai-nilai eigen yang bersesuaian.
Karena $P$ harus invertibel, kolom-kolomnya (yaitu, vektor-vektor eigen $\vektor{p}_1, \dots, \vektor{p}_n$) harus membentuk himpunan yang bebas linear.

\begin{teorema}[Teorema Diagonalisasi]
Sebuah matriks persegi $A$ berukuran $n \times n$ dapat didiagonalkan jika dan hanya jika $A$ memiliki $n$ buah vektor eigen yang bebas linear.
\end{teorema}
\begin{catatan}
Jika $A$ terdiagonalkan, maka matriks pendiagonal $P$ dapat dibentuk dengan menjadikan $n$ vektor eigen yang bebas linear tersebut sebagai kolom-kolomnya. Matriks diagonal $D = P^{-1}AP$ akan memiliki nilai-nilai eigen yang bersesuaian pada diagonal utamanya, dengan urutan yang sama dengan urutan vektor eigen dalam $P$.
\end{catatan}

\begin{teorema}
Misalkan $A$ adalah matriks $n \times n$. Jika $A$ memiliki $n$ nilai eigen yang \textbf{berbeda}, maka $A$ pasti dapat didiagonalkan.
\end{teorema}
\begin{proof}
Vektor-vektor eigen yang bersesuaian dengan nilai-nilai eigen yang berbeda selalu bebas linear. Jika $A$ memiliki $n$ nilai eigen yang berbeda, maka $A$ akan memiliki $n$ vektor eigen yang bebas linear. Berdasarkan teorema sebelumnya, $A$ dapat didiagonalkan.
\end{proof}
\textbf{Peringatan}: Kebalikan dari teorema ini tidak berlaku. Sebuah matriks dapat didiagonalkan meskipun nilai eigennya tidak semuanya berbeda. Kunci utamanya adalah apakah kita dapat menemukan cukup banyak vektor eigen yang bebas linear.

\subsection{Multiplisitas Aljabar dan Geometris}
Kondisi diagonalisasi dapat dinyatakan lebih formal menggunakan konsep multiplisitas.
\begin{itemize}
    \item \textbf{Multiplisitas Aljabar} dari nilai eigen $\lambda$ adalah banyaknya kemunculan $\lambda$ sebagai akar dari polinom karakteristik.
    \item \textbf{Multiplisitas Geometris} dari nilai eigen $\lambda$ adalah dimensi dari ruang eigen $E_\lambda$, yaitu $\dimV(E_\lambda) = \nulitas(A-\lambda I)$.
\end{itemize}

\begin{teorema}
Sebuah matriks persegi $A$ berukuran $n \times n$ dapat didiagonalkan jika dan hanya jika kedua kondisi berikut terpenuhi:
\begin{enumerate}
    \item Polinom karakteristik dari $A$ memiliki $n$ akar (termasuk multiplisitas) di dalam lapangan skalar yang digunakan (misalnya, $\R$).
    \item Untuk setiap nilai eigen $\lambda$ dari $A$, multiplisitas geometrisnya sama dengan multiplisitas aljabarnya.
\end{enumerate}
\end{teorema}
Kondisi kedua adalah yang paling krusial. Jika ada satu saja nilai eigen yang multiplisitas geometrisnya lebih kecil dari multiplisitas aljabarnya, maka matriks tersebut tidak dapat didiagonalkan.

\section{Prosedur untuk Mendiagonalkan Matriks}
Berikut adalah langkah-langkah untuk mendiagonalkan sebuah matriks $A$ berukuran $n \times n$:
\begin{description}
    \item[Langkah 1: Cari Nilai Eigen] Selesaikan persamaan karakteristik $\dettext{A - \lambda I} = 0$ untuk menemukan semua nilai eigen $\lambda_1, \lambda_2, \dots, \lambda_k$ beserta multiplisitas aljabarnya.
    \item[Langkah 2: Cari Basis Ruang Eigen] Untuk setiap nilai eigen $\lambda_i$, temukan basis untuk ruang eigen $E_{\lambda_i}$ dengan menyelesaikan sistem homogen $(A - \lambda_i I)\vektor{x} = \nol$. Dimensi dari ruang eigen ini adalah multiplisitas geometris dari $\lambda_i$.
    \item[Langkah 3: Periksa Kemampuan Diagonalisasi]
    \begin{itemize}
        \item Jika untuk setiap nilai eigen, multiplisitas geometrisnya sama dengan multiplisitas aljabarnya, maka matriks tersebut \textbf{dapat didiagonalkan}. Jumlah total vektor eigen basis yang ditemukan akan sama dengan $n$.
        \item Jika terdapat setidaknya satu nilai eigen yang multiplisitas geometrisnya lebih kecil dari multiplisitas aljabarnya, maka matriks tersebut \textbf{tidak dapat didiagonalkan}. Proses berhenti di sini.
    \end{itemize}
    \item[Langkah 4: Bentuk Matriks P dan D (Jika Terdiagonalkan)]
    \begin{itemize}
        \item Bentuk matriks pendiagonal $P$ dengan menempatkan vektor-vektor basis dari semua ruang eigen sebagai kolom-kolomnya.
        \item Bentuk matriks diagonal $D$ dengan menempatkan nilai-nilai eigen yang bersesuaian pada diagonal utamanya. Urutan nilai eigen di $D$ harus sesuai dengan urutan vektor eigen di $P$.
    \end{itemize}
    \item[Langkah 5: Verifikasi (Opsional)] Periksa bahwa $AP = PD$.
\end{description}

\section{Contoh-Contoh Diagonalisasi}

\begin{contoh}[Kasus Nilai Eigen Berulang, Terdiagonalkan]
Periksa apakah matriks $B = \matriks{1 & 3 & 0 \\ 0 & -2 & 0 \\ 0 & 6 & 1}$ terdiagonalkan. Jika ya, berikan $P$ dan $D$.
\textbf{Solusi}:
\begin{enumerate}
    \item \textbf{Nilai Eigen}: $B$ adalah matriks segitiga bawah. Nilai eigennya adalah entri diagonal: $\lambda_1 = -2$ (multiplisitas aljabar 1) dan $\lambda_2 = 1$ (multiplisitas aljabar 2).
    
    \item \textbf{Basis Ruang Eigen}:
    \begin{itemize}
        \item Untuk $\lambda_1 = -2$:
        $(B+2I)\vektor{x} = \nol \implies \matriks{3 & 3 & 0 \\ 0 & 0 & 0 \\ 0 & 6 & 3}\vektor{x} = \nol$.
        $6x_2+3x_3=0 \Rightarrow x_3=-2x_2$.
        $3x_1+3x_2=0 \Rightarrow x_1=-x_2$.
        Misalkan $x_2=t$. Solusi: $t(-1,1,-2)^T$. Basis $E_{-2} = \{(-1,1,-2)^T\}$. $\dimV(E_{-2})=1$.
        (Multiplisitas geometris = aljabar, sejauh ini baik).
        
        \item Untuk $\lambda_2 = 1$:
        $(B-I)\vektor{x} = \nol \implies \matriks{0 & 3 & 0 \\ 0 & -3 & 0 \\ 0 & 6 & 0}\vektor{x} = \nol$.
        Semua baris menghasilkan $3x_2=0 \Rightarrow x_2=0$.
        Variabel $x_1$ dan $x_3$ bebas. Misalkan $x_1=s, x_3=t$.
        Vektor eigen: $s(1,0,0)^T + t(0,0,1)^T$.
        Basis $E_1 = \left\{ \matriks{1 \\ 0 \\ 0}, \matriks{0 \\ 0 \\ 1} \right\}$. $\dimV(E_1)=2$.
        (Multiplisitas geometris = aljabar).
    \end{itemize}
    \item \textbf{Kesimpulan Diagonalisasi}: Karena untuk setiap nilai eigen, multiplisitas geometris sama dengan multiplisitas aljabar, dan jumlah dimensi ruang eigen ($1+2=3$) sama dengan ukuran matriks, maka matriks $B$ \textbf{terdiagonalkan}.

    \item \textbf{Matriks $P$ dan $D$}:
    Ambil vektor-vektor basis ruang eigen sebagai kolom-kolom $P$. Urutan bisa bebas, asalkan konsisten dengan $D$.
    \[ P = \matriks{-1 & 1 & 0 \\ 1 & 0 & 0 \\ -2 & 0 & 1}, \quad D = \matriks{-2 & 0 & 0 \\ 0 & 1 & 0 \\ 0 & 0 & 1} \]
\end{enumerate}
\end{contoh}

\begin{contoh}[Kasus Tidak Terdiagonalkan]
Periksa apakah matriks $C = \matriks{1 & 2 & 2 \\ 0 & 3 & 0 \\ 0 & 2 & 3}$ terdiagonalkan.
\textbf{Solusi}:
\begin{enumerate}
    \item \textbf{Nilai Eigen}: $\lambda_1=1$ (mult. aljabar 1), $\lambda_2=3$ (mult. aljabar 2).
    \item \textbf{Basis Ruang Eigen}:
    \begin{itemize}
        \item Untuk $\lambda_1=1$: Basis $E_1 = \{ (1,0,0)^T \}$. $\dimV(E_1)=1$.
        \item Untuk $\lambda_2=3$:
        $(C-3I)\vektor{x} = \nol \implies \matriks{-2 & 2 & 2 \\ 0 & 0 & 0 \\ 0 & 2 & 0}\vektor{x} = \nol$.
        $2x_2=0 \Rightarrow x_2=0$.
        $-2x_1+2x_2+2x_3=0 \Rightarrow -2x_1+2x_3=0 \Rightarrow x_1=x_3$.
        Misalkan $x_3=t$. Solusi: $t(1,0,1)^T$.
        Basis $E_3 = \{ (1,0,1)^T \}$. $\dimV(E_3)=1$.
    \end{itemize}
    \item \textbf{Kesimpulan Diagonalisasi}: Untuk nilai eigen $\lambda=3$, multiplisitas aljabar adalah 2, tetapi multiplisitas geometrisnya hanya 1. Karena ada nilai eigen yang multiplisitas geometrisnya lebih kecil dari aljabarnya, matriks $C$ \textbf{tidak terdiagonalkan}.
\end{enumerate}
\end{contoh}

\subsection{Diagonalisasi Matriks Simetris}
Matriks simetris ($A=A^T$) memiliki sifat diagonalisasi yang sangat baik.
\begin{teorema}[Teorema Spektral untuk Matriks Simetris]
Jika $A$ adalah matriks simetris real $n \times n$, maka:
\begin{enumerate}
    \item $A$ selalu terdiagonalkan.
    \item Semua nilai eigen dari $A$ adalah bilangan real.
    \item Vektor-vektor eigen yang berasal dari ruang eigen yang berbeda secara otomatis ortogonal.
    \item Terdapat matriks \textbf{ortogonal} $P$ (yaitu, $P^{-1}=P^T$) sedemikian sehingga $P^TAP=D$ adalah matriks diagonal. Proses ini disebut \textbf{diagonalisasi ortogonal}.
\end{enumerate}
\end{teorema}
Untuk menemukan matriks ortogonal $P$, kita perlu mencari basis ortonormal untuk setiap ruang eigen (menggunakan proses Gram-Schmidt jika perlu) dan menggabungkannya untuk membentuk kolom-kolom $P$.

\end{document}
