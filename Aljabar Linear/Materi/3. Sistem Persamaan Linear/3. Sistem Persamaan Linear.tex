\documentclass[12pt, a4paper]{article}
\usepackage{amsmath} % Untuk lingkungan matriks dan perintah matematika lainnya
\usepackage{amsfonts} % Untuk simbol matematika
\usepackage{amssymb} % Untuk simbol \mathbb{R}
\usepackage{geometry} % Untuk mengatur margin
\usepackage[indonesian]{babel} % Mengatur bahasa Indonesia
\usepackage{array} % Untuk kolom tabel yang lebih baik
\usepackage{booktabs} % Untuk garis tabel yang lebih baik
\usepackage{graphicx} % Untuk menyertakan gambar (jika diperlukan)
\usepackage{hyperref} % Untuk hyperlink (jika diperlukan)
\usepackage{amsthm} % Untuk lingkungan definisi dan teorema
\usepackage{systeme}   % Untuk sistem persamaan
\usepackage{esvect}

\geometry{a4paper, margin=1in} % Mengatur margin halaman

% Definisi lingkungan baru
\newtheorem{definisi}{Definisi}[section]
\newtheorem{contoh}{Contoh}[section]
\newtheorem{teorema}{Teorema}[section]
\newtheorem{fakta}{Fakta}[section]

% Definisi perintah baru untuk kemudahan
\newcommand{\matriks}[1]{\begin{pmatrix} #1 \end{pmatrix}} % Perintah untuk matriks
\newcommand{\R}{\mathbb{R}} % Simbol R untuk bilangan real
\newcommand{\M}[1]{\mathcal{M}_{#1}} % Simbol M untuk himpunan matriks
\newcommand{\rank}{\text{rank}} % Perintah untuk rank
\newcommand{\vektor}[1]{\mathbf{#1}} % Perintah untuk vektor tebal

\title{Sistem Persamaan Linear (SPL)}
\author{Rafi Kamindra 2201006}
\date{}

\begin{document}
\maketitle

\section{Pengantar Sistem Persamaan Linear}
Sebuah persamaan linear sederhana dalam satu variabel adalah $ax=b$, dimana $a \neq 0$. Persamaan ini selalu memiliki solusi unik $x = b/a$. Persamaan ini dapat dipandang sebagai model matematika sederhana dimana $a, x,$ dan $b$ masing-masing "berdimensi" 1.

Dalam aljabar linear, kita seringkali berhadapan dengan sistem persamaan linear yang lebih kompleks, yang dapat ditulis dalam bentuk matriks $AX=B$, dimana $A \in M_{m \times n}$ (matriks $m \times n$), $X$ adalah vektor kolom variabel, dan $B$ adalah vektor kolom konstanta. Pertanyaan yang muncul adalah:
\begin{itemize}
    \item Apakah sistem tersebut selalu mempunyai solusi?
    \item Bagaimana teknik untuk mendapatkan solusinya?
\end{itemize}

\section{Representasi Matriks dari SPL}
Sebuah sistem persamaan linear (SPL) dengan $m$ persamaan linear dan $n$ variabel $x_1, x_2, \dots, x_n$ dapat ditulis dalam bentuk umum:
\begin{align*}
    a_{11}x_1 + a_{12}x_2 + \cdots + a_{1n}x_n &= b_1 \\
    a_{21}x_1 + a_{22}x_2 + \cdots + a_{2n}x_n &= b_2 \\
    \vdots \quad & \quad \vdots \quad & \quad \vdots \\
    a_{m1}x_1 + a_{m2}x_2 + \cdots + a_{mn}x_n &= b_m
\end{align*}
Sistem ini dapat dituliskan dalam bentuk perkalian matriks $AX=B$, yaitu:
\[ \underbrace{\matriks{
a_{11} & a_{12} & \cdots & a_{1n} \\
a_{21} & a_{22} & \cdots & a_{2n} \\
\vdots & \vdots & \ddots & \vdots \\
a_{m1} & a_{m2} & \cdots & a_{mn}
}}_{A}
\underbrace{\matriks{x_1 \\ x_2 \\ \vdots \\ x_n}}_{X}
=
\underbrace{\matriks{b_1 \\ b_2 \\ \vdots \\ b_m}}_{B}
\]
Di sini, $A$ disebut \textbf{matriks koefisien}. Matriks yang dibentuk dengan menggabungkan matriks $A$ dan vektor kolom $B$, yaitu $[A|B]$, disebut \textbf{matriks lengkap} atau \textbf{matriks augmented}.
\[ [A|B] = \left[ \begin{array}{cccc|c}
a_{11} & a_{12} & \cdots & a_{1n} & b_1 \\
a_{21} & a_{22} & \cdots & a_{2n} & b_2 \\
\vdots & \vdots & \ddots & \vdots & \vdots \\
a_{m1} & a_{m2} & \cdots & a_{mn} & b_m
\end{array} \right] \]

\begin{contoh}
Perhatikan SPL berikut:
\begin{align*}
    x_1 - x_2 + 2x_3 &= 5 \\
    x_1 + x_2 - 2x_3 &= -3 \\
    -x_1 - 3x_2 + x_3 &= -4
\end{align*}
SPL ini dapat dituliskan dalam bentuk matriks $AX=B$ sebagai:
\[ \matriks{1 & -1 & 2 \\ 1 & 1 & -2 \\ -1 & -3 & 1} \matriks{x_1 \\ x_2 \\ x_3} = \matriks{5 \\ -3 \\ -4} \]
Matriks lengkap SPL ini adalah:
\[ [A|B] = \left[ \begin{array}{ccc|c} 1 & -1 & 2 & 5 \\ 1 & 1 & -2 & -3 \\ -1 & -3 & 1 & -4 \end{array} \right] \]
\end{contoh}

\section{Solusi Sistem Persamaan Linear}
\begin{definisi}
Misalkan $AX=B$ menyatakan suatu SPL dengan $m$ persamaan dan $n$ variabel. Sebuah vektor $\vektor{Y} \in \R^n$ disebut \textbf{solusi} dari SPL tersebut jika memenuhi persamaan $A\vektor{Y}=B$.
\end{definisi}

\begin{contoh}
Perhatikan bahwa $\vektor {Y} = \matriks{1 \\ 2 \\ 3}$ adalah solusi dari SPL $\matriks{1 & -1 & 2 \\ 1 & 1 & -2 \\ -1 & -3 & 1} \matriks{x_1 \\ x_2 \\ x_3} = \matriks{5 \\ -3 \\ -4}$, karena:
\[ \matriks{1 & -1 & 2 \\ 1 & 1 & -2 \\ -1 & -3 & 1} \matriks{1 \\ 2 \\ 3} = \matriks{1(1) - 1(2) + 2(3) \\ 1(1) + 1(2) - 2(3) \\ -1(1) - 3(2) + 1(3)} = \matriks{1 - 2 + 6 \\ 1 + 2 - 6 \\ -1 - 6 + 3} = \matriks{5 \\ -3 \\ -4} \]
\end{contoh}

\subsection{Metode Mendapatkan Solusi SPL}
Untuk mendapatkan solusi SPL, kita dapat menggunakan Operasi Baris Elementer (OBE) pada matriks lengkap $[A|B]$.
Misalkan $A \in M_{m \times n}$ dan setelah $r$ kali operasi baris elementer, diperoleh matriks eselon $A'$. Kita dapat menuliskan hubungan ini sebagai $E_r \cdots E_2 E_1 A = A'$, dimana $E_i$ adalah matriks elementer yang berpadanan dengan OBE yang dilakukan. Karena setiap matriks elementer $E_i$ mempunyai invers, maka berlaku $A = E_1^{-1} \cdots E_r^{-1} A'$.
Jadi, $A = EA'$ untuk suatu matriks nonsingular $E$ (hasil kali dari invers matriks-matriks elementer).

\begin{teorema}
Misalkan $[A|B]$ adalah matriks lengkap suatu SPL dan $[A|B] \sim [A'|B']$ (artinya $[A'|B']$ diperoleh dari $[A|B]$ melalui OBE). Maka berlaku: $\vektor{Y}$ adalah solusi $AX=B$ jika dan hanya jika $\vektor{Y}$ adalah solusi $A'X=B'$.
\end{teorema}
\begin{proof}
Jika $A\vektor{Y}=B$, maka $E A' \vektor{Y}=B$ (karena $A=EA'$). Karena $E$ nonsingular, kita dapat mengalikan dengan $E^{-1}$ dari kiri: $E^{-1} E A' \vektor{Y} = E^{-1}B$, sehingga $A'\vektor{Y} = E^{-1}B$. Jika kita definisikan $B' = E^{-1}B$, maka $A'\vektor{Y}=B'$.
Sebaliknya, jika $A'\vektor{Y}=B'$, maka substitusi $A' = E^{-1}A$ memberikan $E^{-1}A\vektor{Y}=B'$. Kalikan dengan $E$ dari kiri: $E E^{-1}A\vektor{Y}=EB'$, sehingga $A\vektor{Y}=EB'$. Jika kita definisikan $B=EB'$, maka $A\vektor{Y}=B$.
Intinya, operasi baris elementer tidak mengubah himpunan solusi dari sistem persamaan linear.
\end{proof}

\subsection{Metode Eliminasi Gauss}
Misalkan $AX=B$ suatu SPL. Langkah-langkah untuk menyelesaikannya menggunakan Eliminasi Gauss adalah:
\begin{enumerate}
    \item Tuliskan matriks lengkap $[A|B]$.
    \item Ubah $[A|B]$ menjadi bentuk eselon baris $[A'|B']$ menggunakan OBE.
    \item Tuliskan SPL yang bersesuaian dengan $[A'|B']$, yaitu $A'X=B'$.
    \item Lakukan \textbf{substitusi balik} (back-substitution) pada $A'X=B'$ untuk mendapatkan nilai variabel-variabel.
    \item Jika ada kolom pada $A'$ yang tidak mempunyai 1 utama (pivot), maka variabel yang bersesuaian dengan kolom tersebut adalah \textbf{variabel bebas} dan dapat diberikan parameter (misalnya $t, s, \dots$).
\end{enumerate}
Jika pada langkah 2, kita mengubah matriks hingga bentuk eselon baris \textit{tereduksi}, maka metode ini disebut \textbf{Eliminasi Gauss-Jordan}. Dalam kasus ini, solusi dapat langsung dibaca dari matriks tanpa substitusi balik.

\begin{contoh}[Solusi Tunggal]
Carilah solusi dari:
\sysdelim..
\systeme{
x_1 - x_2 + 2x_3 = 5,
x_2 - 2x_3 = -3,
-x_1 + 3x_2 = 1
}
Solusi:
Matriks lengkapnya adalah $\left[ \begin{array}{ccc|c} 1 & -1 & 2 & 5 \\ 0 & 1 & -2 & -3 \\ -1 & 3 & 0 & 1 \end{array} \right]$.
$R_3 \to R_3 + R_1$:
$\left[ \begin{array}{ccc|c} 1 & -1 & 2 & 5 \\ 0 & 1 & -2 & -3 \\ 0 & 2 & 2 & 6 \end{array} \right]$.
$R_3 \to R_3 - 2R_2$:
$\left[ \begin{array}{ccc|c} 1 & -1 & 2 & 5 \\ 0 & 1 & -2 & -3 \\ 0 & 0 & 6 & 12 \end{array} \right]$.
$R_3 \to \frac{1}{6}R_3$:
$\left[ \begin{array}{ccc|c} 1 & -1 & 2 & 5 \\ 0 & 1 & -2 & -3 \\ 0 & 0 & 1 & 2 \end{array} \right]$. Ini adalah bentuk eselon baris.
SPL yang bersesuaian:
\begin{align*} x_1 - x_2 + 2x_3 &= 5 \\ x_2 - 2x_3 &= -3 \\ x_3 &= 2 \end{align*}
Substitusi balik:
Dari persamaan ketiga, $x_3 = 2$.
Substitusi $x_3=2$ ke persamaan kedua: $x_2 - 2(2) = -3 \Rightarrow x_2 - 4 = -3 \Rightarrow x_2 = 1$.
Substitusi $x_2=1$ dan $x_3=2$ ke persamaan pertama: $x_1 - 1 + 2(2) = 5 \Rightarrow x_1 - 1 + 4 = 5 \Rightarrow x_1 + 3 = 5 \Rightarrow x_1 = 2$.
Jadi, SPL mempunyai solusi tunggal $\vektor{Y} = \matriks{2 \\ 1 \\ 2}$.
\end{contoh}

\begin{contoh}[Banyak Solusi]
Carilah solusi dari:
\sysdelim..
\systeme{
x_1 + 2x_2 - x_3 = 1,
x_1 + x_2 + 2x_3 = 1,
2x_1 + 3x_2 + x_3 = 2
}
Solusi:
Matriks lengkapnya adalah $\left[ \begin{array}{ccc|c} 1 & 2 & -1 & 1 \\ 1 & 1 & 2 & 1 \\ 2 & 3 & 1 & 2 \end{array} \right]$.
$R_2 \to R_2 - R_1$, $R_3 \to R_3 - 2R_1$:
$\left[ \begin{array}{ccc|c} 1 & 2 & -1 & 1 \\ 0 & -1 & 3 & 0 \\ 0 & -1 & 3 & 0 \end{array} \right]$.
$R_3 \to R_3 - R_2$: $\left[ \begin{array}{ccc|c} 1 & 2 & -1 & 1 \\ 0 & -1 & 3 & 0 \\ 0 & 0 & 0 & 0 \end{array} \right]$.
$R_2 \to -R_2$: $\left[ \begin{array}{ccc|c} 1 & 2 & -1 & 1 \\ 0 & 1 & -3 & 0 \\ 0 & 0 & 0 & 0 \end{array} \right]$. Ini bentuk eselon baris.
Kolom ketiga ($x_3$) tidak mempunyai 1 utama, jadi $x_3$ adalah variabel bebas. Misalkan $x_3 = t \in \R$.
Dari baris kedua: $x_2 - 3x_3 = 0 \Rightarrow x_2 = 3x_3 = 3t$.
Dari baris pertama: $x_1 + 2x_2 - x_3 = 1 \Rightarrow x_1 = 1 - 2x_2 + x_3 = 1 - 2(3t) + t = 1 - 6t + t = 1 - 5t$.
Solusi umumnya adalah $(1-5t, 3t, t)$. SPL ini mempunyai solusi tak hingga banyak.
\end{contoh}

\begin{contoh}[Tidak Ada Solusi]
Carilah solusi dari:
\sysdelim..
\systeme{
x_1 + 2x_2 - x_3 = 1,
x_1 + x_2 + 2x_3 = 1,
2x_1 + 3x_2 + x_3 = 1
}
Solusi:
Matriks lengkapnya adalah $\left[ \begin{array}{ccc|c} 1 & 2 & -1 & 1 \\ 1 & 1 & 2 & 1 \\ 2 & 3 & 1 & 1 \end{array} \right]$.
$R_2 \to R_2 - R_1$, $R_3 \to R_3 - 2R_1$:
$\left[ \begin{array}{ccc|c} 1 & 2 & -1 & 1 \\ 0 & -1 & 3 & 0 \\ 0 & -1 & 3 & -1 \end{array} \right]$.
$R_3 \to R_3 - R_2$: $\left[ \begin{array}{ccc|c} 1 & 2 & -1 & 1 \\ 0 & -1 & 3 & 0 \\ 0 & 0 & 0 & -1 \end{array} \right]$.
Baris ketiga menyatakan $0x_1 + 0x_2 + 0x_3 = -1$, atau $0 = -1$. Ini adalah kontradiksi.
Karena tidak ada nilai $x_1, x_2, x_3$ yang memenuhi $0=-1$, maka SPL ini \textbf{tidak mempunyai solusi}.
\end{contoh}

\section{Kemungkinan Solusi SPL}
Setiap sistem persamaan linear memiliki salah satu dari tiga kemungkinan berikut:
\begin{enumerate}
    \item Mempunyai \textbf{solusi tunggal}.
    \item Mempunyai \textbf{solusi tak hingga banyak}.
    \item \textbf{Tidak mempunyai solusi} (inkonsisten).
\end{enumerate}

\subsection{Kaitan Rank dengan Eksistensi Solusi}
\begin{teorema}[Teorema Rouché–Capelli]
Sistem Persamaan Linear $AX=B$ mempunyai solusi jika dan hanya jika $\rank(A) = \rank([A|B])$.
\begin{itemize}
    \item Jika $\rank(A) = \rank([A|B]) = n$ (jumlah variabel), maka SPL memiliki solusi tunggal.
    \item Jika $\rank(A) = \rank([A|B]) < n$, maka SPL memiliki solusi tak hingga banyak (jumlah parameter bebas adalah $n - \rank(A)$).
    \item Jika $\rank(A) < \rank([A|B])$, maka SPL tidak memiliki solusi. (Ini terjadi jika ada baris $[0\;0\;\dots\;0 \mid k]$ dengan $k \neq 0$ dalam bentuk eselon baris dari matriks augmented).
\end{itemize}
\end{teorema}

\end{document}
