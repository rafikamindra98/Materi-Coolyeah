\documentclass{article}
\usepackage{amsmath} % Untuk lingkungan matriks dan perintah matematika lainnya
\usepackage{amsfonts} % Untuk simbol
\usepackage{amssymb} % Untuk simbol
\usepackage{geometry} % Untuk mengatur margin
\usepackage{systeme} % Untuk sistem persamaan linear
\geometry{a4paper, margin=1in}

\title{Aljabar Linear: Sistem Persamaan Linear}
\author{Rafi Kamindra 2201006}
\date{} % Kosongkan tanggal jika tidak ingin ditampilkan

\begin{document}
\maketitle
\pagenumbering{gobble} % Menghilangkan nomor halaman jika tidak diinginkan untuk halaman judul

\section*{Carilah solusi dari}

\subsection*{1.
\sysdelim..
\systeme{
2x_1 - x_2 + x_3 = 0,
x_1 + x_2 - x_3 = 2,
-x_1 - x_2 + x_3 = 1
}}

\textbf{Jawaban:}

Matriks augmented:
\[ \left[ \begin{array}{ccc|c} 2 & -1 & 1 & 0 \\ 1 & 1 & -1 & 2 \\ -1 & -1 & 1 & 1 \end{array} \right] \]
$R_1 \leftrightarrow R_2$:
\[ \left[ \begin{array}{ccc|c} 1 & 1 & -1 & 2 \\ 2 & -1 & 1 & 0 \\ -1 & -1 & 1 & 1 \end{array} \right] \]
$R_2 \rightarrow R_2 - 2R_1$:
\[ \left[ \begin{array}{ccc|c} 1 & 1 & -1 & 2 \\ 0 & -3 & 3 & -4 \\ -1 & -1 & 1 & 1 \end{array} \right] \]
$R_3 \rightarrow R_3 + R_1$:
\[ \left[ \begin{array}{ccc|c} 1 & 1 & -1 & 2 \\ 0 & -3 & 3 & -4 \\ 0 & 0 & 0 & 3 \end{array} \right] \]
Dari baris ketiga, $0x_1 + 0x_2 + 0x_3 = 3$, yang berarti $0 = 3$. Ini adalah kontradiksi.
Maka, SPL tidak mempunyai solusi.

\subsection*{2. 
\sysdelim..
\systeme{
x_1 - x_2 + 2x_3 = 1,
-x_1 - 2x_2 - x_3 = -3,
2x_1 + x_2 + x_3 = 4
}}

\textbf{Jawaban:}

Matriks augmented:
\[ \left[ \begin{array}{ccc|c} 1 & -1 & 2 & 1 \\ -1 & -2 & -1 & -3 \\ 2 & 1 & 1 & 4 \end{array} \right] \]
$R_2 \rightarrow R_2 + R_1$:
\[ \left[ \begin{array}{ccc|c} 1 & -1 & 2 & 1 \\ 0 & -3 & 1 & -2 \\ 2 & 1 & 1 & 4 \end{array} \right] \]
$R_3 \rightarrow R_3 - 2R_1$:
\[ \left[ \begin{array}{ccc|c} 1 & -1 & 2 & 1 \\ 0 & -3 & 1 & -2 \\ 0 & 3 & -3 & 2 \end{array} \right] \]
$R_3 \rightarrow R_3 + R_2$:
\[ \left[ \begin{array}{ccc|c} 1 & -1 & 2 & 1 \\ 0 & -3 & 1 & -2 \\ 0 & 0 & -2 & 0 \end{array} \right] \]
Dari $R_3: -2x_3 = 0 \Rightarrow x_3 = 0$.
Dari $R_2: -3x_2 + x_3 = -2 \Rightarrow -3x_2 + 0 = -2 \Rightarrow -3x_2 = -2 \Rightarrow x_2 = 2/3$.
Dari $R_1: x_1 - x_2 + 2x_3 = 1 \Rightarrow x_1 - 2/3 + 2(0) = 1 \Rightarrow x_1 - 2/3 = 1 \Rightarrow x_1 = 1 + 2/3 = 5/3$.
Solusi: $x_1 = 5/3, x_2 = 2/3, x_3 = 0$.

\subsection*{3. 
\sysdelim..
\systeme{
-x_1 + x_2 + 2x_3 = 2,
x_1 + x_2 - 2x_3 = 0,
2x_1 + 4x_2 - 4x_3 = 2
}}

\textbf{Jawaban:}


Matriks augmented:
\[ \left[ \begin{array}{ccc|c} -1 & 1 & 2 & 2 \\ 1 & 1 & -2 & 0 \\ 2 & 4 & -4 & 2 \end{array} \right] \]
$R_1 \rightarrow -R_1$:
\[ \left[ \begin{array}{ccc|c} 1 & -1 & -2 & -2 \\ 1 & 1 & -2 & 0 \\ 2 & 4 & -4 & 2 \end{array} \right] \]
$R_2 \rightarrow R_2 - R_1$:
\[ \left[ \begin{array}{ccc|c} 1 & -1 & -2 & -2 \\ 0 & 2 & 0 & 2 \\ 2 & 4 & -4 & 2 \end{array} \right] \]
$R_3 \rightarrow R_3 - 2R_1$:
\[ \left[ \begin{array}{ccc|c} 1 & -1 & -2 & -2 \\ 0 & 2 & 0 & 2 \\ 0 & 6 & 0 & 6 \end{array} \right] \]
$R_2 \rightarrow \frac{1}{2}R_2$:
\[ \left[ \begin{array}{ccc|c} 1 & -1 & -2 & -2 \\ 0 & 1 & 0 & 1 \\ 0 & 6 & 0 & 6 \end{array} \right] \]
$R_3 \rightarrow R_3 - 6R_2$:
\[ \left[ \begin{array}{ccc|c} 1 & -1 & -2 & -2 \\ 0 & 1 & 0 & 1 \\ 0 & 0 & 0 & 0 \end{array} \right] \]
Dari $R_2: x_2 = 1$.
Dari $R_1: x_1 - x_2 - 2x_3 = -2 \Rightarrow x_1 - 1 - 2x_3 = -2 \Rightarrow x_1 = 2x_3 - 1$.
Misalkan $x_3 = t$ (parameter).
Solusi: $x_1 = 2t - 1, x_2 = 1, x_3 = t$. SPL ini mempunyai banyak solusi.

\subsection*{4. 
\sysdelim..
\systeme{
-x_1 + x_2 + 2x_3 + x_4 = 2,
x_1 + x_2 - 2x_3 - 2x_4 = 0,
2x_1 - x_2 + 2x_3 - x_4 = 1
}}

\textbf{Jawaban:}

Matriks augmented:
\[ \left[ \begin{array}{cccc|c} -1 & 1 & 2 & 1 & 2 \\ 1 & 1 & -2 & -2 & 0 \\ 2 & -1 & 2 & -1 & 1 \end{array} \right] \]
$R_1 \rightarrow -R_1$:
\[ \left[ \begin{array}{cccc|c} 1 & -1 & -2 & -1 & -2 \\ 1 & 1 & -2 & -2 & 0 \\ 2 & -1 & 2 & -1 & 1 \end{array} \right] \]
$R_2 \rightarrow R_2 - R_1$:
\[ \left[ \begin{array}{cccc|c} 1 & -1 & -2 & -1 & -2 \\ 0 & 2 & 0 & -1 & 2 \\ 2 & -1 & 2 & -1 & 1 \end{array} \right] \]
$R_3 \rightarrow R_3 - 2R_1$:
\[ \left[ \begin{array}{cccc|c} 1 & -1 & -2 & -1 & -2 \\ 0 & 2 & 0 & -1 & 2 \\ 0 & 1 & 6 & 1 & 5 \end{array} \right] \]
$R_2 \leftrightarrow R_3$:
\[ \left[ \begin{array}{cccc|c} 1 & -1 & -2 & -1 & -2 \\ 0 & 1 & 6 & 1 & 5 \\ 0 & 2 & 0 & -1 & 2 \end{array} \right] \]
$R_3 \rightarrow R_3 - 2R_2$:
\[ \left[ \begin{array}{cccc|c} 1 & -1 & -2 & -1 & -2 \\ 0 & 1 & 6 & 1 & 5 \\ 0 & 0 & -12 & -3 & -8 \end{array} \right] \]
$R_3 \rightarrow -\frac{1}{12}R_3$ (atau biarkan untuk menghindari pecahan terlalu dini):
Misalkan $x_4 = t$.
Dari $R_3: -12x_3 - 3x_4 = -8 \Rightarrow -12x_3 - 3t = -8 \Rightarrow -12x_3 = 3t - 8 \Rightarrow x_3 = \frac{8-3t}{12} = \frac{2}{3} - \frac{1}{4}t$.
Dari $R_2: x_2 + 6x_3 + x_4 = 5 \Rightarrow x_2 + 6(\frac{2}{3} - \frac{1}{4}t) + t = 5 \Rightarrow x_2 + 4 - \frac{3}{2}t + t = 5 \Rightarrow x_2 + 4 - \frac{1}{2}t = 5 \Rightarrow x_2 = 1 + \frac{1}{2}t$.
Dari $R_1: x_1 - x_2 - 2x_3 - x_4 = -2 \Rightarrow x_1 - (1 + \frac{1}{2}t) - 2(\frac{2}{3} - \frac{1}{4}t) - t = -2$
$x_1 - 1 - \frac{1}{2}t - \frac{4}{3} + \frac{1}{2}t - t = -2 \Rightarrow x_1 - 1 - \frac{4}{3} - t = -2 \Rightarrow x_1 - \frac{7}{3} - t = -2 \Rightarrow x_1 = t + \frac{7}{3} - 2 = t + \frac{1}{3}$.
Solusi: $x_1 = t + 1/3, x_2 = 1 + t/2, x_3 = 2/3 - t/4, x_4 = t$. SPL ini mempunyai banyak solusi.

\subsection*{5. Perhatikan SPL 
\sysdelim..
\systeme{
x_1 + x_2 = a,
2x_1 + 2x_2 = b
}\\
berapa saja, SPL ini mempunyai solusi?}

\textbf{Jawaban:}

Matriks augmented:
\[ \left[ \begin{array}{cc|c} 1 & 1 & a \\ 2 & 2 & b \end{array} \right] \]
$R_2 \rightarrow R_2 - 2R_1$:
\[ \left[ \begin{array}{cc|c} 1 & 1 & a \\ 0 & 0 & b - 2a \end{array} \right] \]
Agar SPL ini mempunyai solusi, baris terakhir tidak boleh menghasilkan kontradiksi.
Ini berarti $0x_1 + 0x_2 = b - 2a$ harus konsisten.
Jadi, $b - 2a = 0$, atau $b = 2a$.
SPL ini mempunyai solusi jika dan hanya jika $b = 2a$. Jika kondisi ini terpenuhi, akan ada banyak solusi (misalkan $x_2 = t$, maka $x_1 = a - t$). Jika $b \neq 2a$, SPL tidak mempunyai solusi.

\subsection*{6. Diketahui}
$A = \begin{pmatrix} 1 & 2 \\ -2 & 3 \end{pmatrix}$, $C = \begin{pmatrix} 1 & 1 \\ 3 & 2 \end{pmatrix}$
Carilah matriks $B$ yang memenuhi $AB=C$.

\textbf{Jawaban:}

Jika $AB=C$, maka $B = A^{-1}C$.
Pertama, cari $A^{-1}$.
$\det(A) = (1)(3) - (2)(-2) = 3 - (-4) = 3+4 = 7$.
$A^{-1} = \frac{1}{7} \begin{pmatrix} 3 & -2 \\ 2 & 1 \end{pmatrix} = \begin{pmatrix} 3/7 & -2/7 \\ 2/7 & 1/7 \end{pmatrix}$.
Sekarang hitung $B = A^{-1}C$:
\begin{align*} B &= \frac{1}{7} \begin{pmatrix} 3 & -2 \\ 2 & 1 \end{pmatrix} \begin{pmatrix} 1 & 1 \\ 3 & 2 \end{pmatrix} \\ &= \frac{1}{7} \begin{pmatrix} (3)(1)+(-2)(3) & (3)(1)+(-2)(2) \\ (2)(1)+(1)(3) & (2)(1)+(1)(2) \end{pmatrix} \\ &= \frac{1}{7} \begin{pmatrix} 3-6 & 3-4 \\ 2+3 & 2+2 \end{pmatrix} = \frac{1}{7} \begin{pmatrix} -3 & -1 \\ 5 & 4 \end{pmatrix} \\ &= \begin{pmatrix} -3/7 & -1/7 \\ 5/7 & 4/7 \end{pmatrix} \end{align*}
Jadi, $B = \begin{pmatrix} -3/7 & -1/7 \\ 5/7 & 4/7 \end{pmatrix}$.

\end{document}
