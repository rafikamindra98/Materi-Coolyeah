\documentclass[12pt, a4paper]{article}
\usepackage{amsmath} % Untuk lingkungan matriks dan perintah matematika lainnya
\usepackage{amsfonts} % Untuk simbol matematika
\usepackage{amssymb} % Untuk simbol \mathbb{R}
\usepackage{geometry} % Untuk mengatur margin
\usepackage[indonesian]{babel} % Mengatur bahasa Indonesia
\usepackage{array} % Untuk kolom tabel yang lebih baik
\usepackage{booktabs} % Untuk garis tabel yang lebih baik
\usepackage{graphicx} % Untuk menyertakan gambar (jika diperlukan)
\usepackage{hyperref} % Untuk hyperlink (jika diperlukan)
\usepackage{amsthm} % Untuk lingkungan definisi dan teorema
\usepackage{enumitem}

\geometry{a4paper, margin=1in} % Mengatur margin halaman

% Definisi lingkungan baru
\newtheorem{definisi}{Definisi}[section]
\newtheorem{contoh}{Contoh}[section]
\newtheorem{teorema}{Teorema}[section]
\newtheorem{fakta}{Fakta}[section]

% Definisi perintah baru untuk kemudahan
\newcommand{\matriks}[1]{\begin{pmatrix} #1 \end{pmatrix}} % Perintah untuk matriks
\newcommand{\R}{\mathbb{R}} % Simbol R untuk bilangan real
\newcommand{\M}[1]{\mathcal{M}_{#1}} % Simbol M untuk himpunan matriks
\newcommand{\rank}{\text{rank}} % Perintah untuk rank

\title{Operasi Baris Elementer, Bentuk Eselon, dan Rank Matriks}
\author{Rafi Kamindra 2201006}
\date{}

\begin{document}
\maketitle

\section{Pengantar}
Operasi Baris Elementer (OBE) adalah alat fundamental dalam aljabar linear yang digunakan untuk menyederhanakan matriks dan menyelesaikan sistem persamaan linear. Melalui OBE, kita dapat mengubah matriks ke bentuk yang lebih sederhana yang disebut bentuk eselon baris atau bentuk eselon baris tereduksi. Bentuk-bentuk ini mengungkapkan informasi penting tentang matriks, termasuk \textbf{rank}-nya, yang merupakan salah satu karakteristik utama dari sebuah matriks.

\begin{itemize}
    \item \textbf{Ekivalensi Matriks}: Dua matriks dikatakan ekivalen baris jika satu dapat diperoleh dari yang lain melalui serangkaian OBE. Matriks-matriks yang ekivalen baris menyimpan "informasi" inti yang sama, seperti ruang solusi dari sistem persamaan linear yang bersesuaian.
    \item \textbf{Rank Matriks}: Rank sebuah matriks adalah informasi terpenting yang dapat diperoleh darinya, yang berkaitan dengan dimensi ruang baris dan ruang kolom, serta jumlah variabel bebas dalam sistem persamaan linear yang bersesuaian.
\end{itemize}

\section{Operasi Baris Elementer (OBE)}
\begin{definisi}
Misalkan $A$ suatu matriks. \textbf{Operasi Baris Elementer (OBE)} pada $A$ mencakup tiga jenis operasi berikut:
\begin{enumerate}
    \item Menukar baris ke-$i$ dengan baris ke-$j$. Dinotasikan sebagai $R_i \leftrightarrow R_j$ (atau $b_i \leftrightarrow b_j$).
    \item Mengalikan baris ke-$i$ dengan konstanta tak nol $k$. Dinotasikan sebagai $kR_i \to R_i$ (atau $kb_i$).
    \item Menjumlahkan hasil perkalian suatu baris ke-$i$ dengan skalar $k$ ke baris ke-$j$ (dimana $i \neq j$). Dinotasikan sebagai $R_j + kR_i \to R_j$ (atau $b_j + kb_i$).
\end{enumerate}
\end{definisi}

\begin{definisi}
Matriks $B$ dikatakan \textbf{ekuivalen baris} dengan matriks $A$, dinotasikan $A \sim B$, jika $B$ dapat diperoleh dari $A$ melalui serangkaian (satu atau lebih) Operasi Baris Elementer.
\end{definisi}

\begin{contoh}
Perhatikan matriks $A = \matriks{2 & 1 & -3 \\ 1 & 0 & -1 \\ -3 & 2 & 0}$.
Kita dapat melakukan serangkaian OBE sebagai berikut:
\[ \matriks{2 & 1 & -3 \\ 1 & 0 & -1 \\ -3 & 2 & 0} \xrightarrow{R_1 \leftrightarrow R_2}
   \matriks{1 & 0 & -1 \\ 2 & 1 & -3 \\ -3 & 2 & 0} \]
\[ \xrightarrow{R_2 \to R_2 - 2R_1}
   \matriks{1 & 0 & -1 \\ 0 & 1 & -1 \\ -3 & 2 & 0} \]
\[ \xrightarrow{R_3 \to R_3 + 3R_1}
   \matriks{1 & 0 & -1 \\ 0 & 1 & -1 \\ 0 & 2 & -3} \]
\[ \xrightarrow{R_3 \to R_3 - 2R_2}
   \matriks{1 & 0 & -1 \\ 0 & 1 & -1 \\ 0 & 0 & -1} \]
Matriks terakhir ini ekuivalen baris dengan matriks $A$.
\end{contoh}

\section{Bentuk Eselon Baris dan Eselon Baris Tereduksi}
Tujuan utama dari OBE adalah untuk mengubah matriks menjadi bentuk yang lebih sederhana yang mengungkapkan sifat-sifat pentingnya.

\subsection{Bentuk Eselon Baris}
\begin{definisi}
Suatu matriks $A$ dikatakan dalam \textbf{bentuk eselon baris} jika memenuhi semua sifat berikut:
\begin{enumerate}[label=(\alph*)]
    \item Untuk setiap baris tak nol, entri tak nol pertama dari kiri (disebut \textbf{entri utama} atau \textbf{pivot}) adalah 1. Angka 1 ini disebut \textbf{1 utama}.
    \item Untuk sebarang dua baris tak nol, 1 utama pada baris yang lebih bawah terletak di kolom yang lebih kanan daripada 1 utama pada baris yang lebih atas.
    \item Jika terdapat baris-baris yang seluruhnya terdiri dari nol (baris nol), maka baris-baris tersebut terletak di bagian bawah matriks.
\end{enumerate}
\end{definisi}

\begin{contoh}
Perhatikan tiga matriks berikut:
\[ A = \matriks{1 & 2 & 1 & 3 \\ 0 & 1 & 3 & 1 \\ 0 & 0 & 1 & 1}, \quad
   B = \matriks{1 & -1 & 1 & 2 \\ 0 & 1 & 1 & 1 \\ 0 & 0 & 0 & 1}, \quad
   C = \matriks{0 & 1 & 0 & -1 \\ 0 & 1 & 1 & 1 \\ 0 & 0 & 0 & 0} \]
Matriks $A$ dan $B$ keduanya adalah matriks dalam bentuk eselon baris.
Matriks $C$ \textbf{bukan} dalam bentuk eselon baris karena pada kolom kedua, entri 1 pada baris kedua tidak terletak lebih kanan dari entri 1 pada baris pertama (bahkan, pada baris pertama, entri utama adalah 1 di kolom kedua, dan pada baris kedua, entri utama juga 1 di kolom kedua, ini melanggar sifat (b) karena tidak 'lebih kanan', juga melanggar sifat (a) jika kita anggap baris pertama, entri utama 1 ada di kolom kedua, lalu baris kedua, entri utama 1 juga di kolom kedua, seharusnya 1 utama baris bawah lebih kanan). Sifat (a) untuk baris kedua C adalah 1 di $a_{22}$, untuk baris pertama C adalah 1 di $a_{12}$. Sifat (b) tidak terpenuhi.

Agar C menjadi eselon baris, misalnya:
$C \xrightarrow{R_2 \to R_2 - R_1} \matriks{0 & 1 & 0 & -1 \\ 0 & 0 & 1 & 2 \\ 0 & 0 & 0 & 0}$. Ini adalah bentuk eselon baris.
\end{contoh}

\subsection{Bentuk Eselon Baris Tereduksi}
\begin{definisi}
Matriks dalam bentuk eselon baris yang juga memenuhi sifat tambahan berikut:
\begin{enumerate}[label=(\alph*), resume]
    \item Setiap kolom yang memuat 1 utama memiliki entri nol di semua posisi lainnya dalam kolom tersebut.
\end{enumerate}
dikatakan dalam \textbf{bentuk eselon baris tereduksi} (atau bentuk kanonik baris).
\end{definisi}

\begin{contoh}
\[ A = \matriks{1 & 0 & 0 & 3 \\ 0 & 1 & 0 & 1 \\ 0 & 0 & 1 & 1}, \quad
   B = \matriks{1 & 0 & 1 & 0 \\ 0 & 1 & 1 & 0 \\ 0 & 0 & 0 & 1} \]
Matriks $A$ adalah dalam bentuk eselon baris tereduksi.
Matriks $B$ adalah dalam bentuk eselon baris, tetapi \textbf{bukan} eselon baris tereduksi karena kolom ketiga yang memuat 1 utama (dari $B_{34}=1$) seharusnya memiliki 0 di atasnya, tetapi $B_{13}=1$ dan $B_{23}=1$. Juga kolom keempat, $B_{14}=0, B_{24}=0$ diperlukan.
Untuk $B$ menjadi tereduksi:
$B \xrightarrow{R_1 \to R_1 - R_3} \matriks{1 & 0 & 1 & 0 \\ 0 & 1 & 1 & 0 \\ 0 & 0 & 0 & 1} \xrightarrow{R_2 \to R_2 - R_3} \text{operasi salah, tidak ada 1 utama di } R_3 \text{ untuk kolom 3}$.
Bentuk eselon tereduksi dari $B$ (setelah $R_2 \to R_2 - R_3$ yang salah diinterpretasi):
$\matriks{1 & 0 & 1 & 0 \\ 0 & 1 & 1 & 0 \\ 0 & 0 & 0 & 1}$. Ini sudah dalam bentuk eselon baris tereduksi.
\end{contoh}

\subsection{Mengubah Matriks menjadi Bentuk Eselon (Eliminasi Gauss)}
Proses mengubah matriks menjadi bentuk eselon baris disebut \textbf{Eliminasi Gauss}.
Prinsipnya adalah sebagai berikut:
\begin{enumerate}
    \item Cari kolom paling kiri yang tidak seluruhnya nol.
    \item Jika entri teratas kolom tersebut adalah nol, tukar baris teratas dengan baris lain yang memiliki entri tak nol di kolom tersebut.
    \item Jadikan entri teratas (pivot) menjadi 1 utama dengan mengalikan baris tersebut dengan skalar yang sesuai.
    \item Gunakan 1 utama tersebut untuk membuat semua entri di bawahnya dalam kolom yang sama menjadi nol, dengan menjumlahkan kelipatan yang sesuai dari baris yang memuat 1 utama ke baris-baris di bawahnya.
    \item Abaikan baris yang sudah memuat 1 utama, dan ulangi proses dari langkah 1 untuk submatriks yang tersisa. Lanjutkan hingga seluruh matriks dalam bentuk eselon baris.
\end{enumerate}

\subsection{Mengubah Matriks menjadi Bentuk Eselon Baris Tereduksi (Eliminasi Gauss-Jordan)}
Proses mengubah matriks menjadi bentuk eselon baris tereduksi disebut \textbf{Eliminasi Gauss-Jordan}. Ini adalah kelanjutan dari Eliminasi Gauss:
\begin{enumerate}
    \item Setelah matriks dalam bentuk eselon baris, mulai dari 1 utama paling kanan, gunakan operasi baris untuk membuat semua entri di atas 1 utama tersebut menjadi nol.
    \item Ulangi untuk setiap 1 utama, bergerak dari kanan ke kiri, dari bawah ke atas.
\end{enumerate}

\begin{teorema}
Setiap matriks dapat diubah menjadi bentuk eselon baris (dan bentuk eselon baris tereduksi) melalui serangkaian Operasi Baris Elementer. Bentuk eselon baris tereduksi dari sebuah matriks adalah unik.
\end{teorema}

\begin{contoh}[Mengubah ke Bentuk Eselon Baris Tereduksi]
Perhatikan $A = \matriks{2 & 0 & 1 \\ -1 & 1 & 3 \\ 3 & -2 & 1}$.
Langkah-langkah OBE (mungkin berbeda dari contoh di slide, tujuannya sama):
\[ \matriks{2 & 0 & 1 \\ -1 & 1 & 3 \\ 3 & -2 & 1} \xrightarrow{R_1 \leftrightarrow R_2} \matriks{-1 & 1 & 3 \\ 2 & 0 & 1 \\ 3 & -2 & 1} \xrightarrow{R_1 \to -R_1} \matriks{1 & -1 & -3 \\ 2 & 0 & 1 \\ 3 & -2 & 1} \]
\[ \xrightarrow{R_2 \to R_2 - 2R_1} \matriks{1 & -1 & -3 \\ 0 & 2 & 7 \\ 3 & -2 & 1} \xrightarrow{R_3 \to R_3 - 3R_1} \matriks{1 & -1 & -3 \\ 0 & 2 & 7 \\ 0 & 1 & 10} \]
\[ \xrightarrow{R_2 \leftrightarrow R_3} \matriks{1 & -1 & -3 \\ 0 & 1 & 10 \\ 0 & 2 & 7} \xrightarrow{R_3 \to R_3 - 2R_2} \matriks{1 & -1 & -3 \\ 0 & 1 & 10 \\ 0 & 0 & -13} \]
Ini adalah bentuk eselon baris (setelah $R_3 \to -\frac{1}{13}R_3$ untuk mendapatkan 1 utama).
\[ \xrightarrow{R_3 \to -\frac{1}{13}R_3} \matriks{1 & -1 & -3 \\ 0 & 1 & 10 \\ 0 & 0 & 1} \]
Lanjutkan ke bentuk eselon baris tereduksi:
\[ \xrightarrow{R_1 \to R_1 + R_2} \matriks{1 & 0 & 7 \\ 0 & 1 & 10 \\ 0 & 0 & 1} \quad (\text{Bekerja dari atas ke bawah dulu untuk 0 di bawah 1 utama}) \]
\[ \xrightarrow{R_1 \to R_1 - 7R_3} \matriks{1 & 0 & 0 \\ 0 & 1 & 10 \\ 0 & 0 & 1} \xrightarrow{R_2 \to R_2 - 10R_3} \matriks{1 & 0 & 0 \\ 0 & 1 & 0 \\ 0 & 0 & 1} = I_3 \]
Jadi, matriks $A$ ekuivalen baris dengan matriks identitas $I_3$.
\end{contoh}

\section{Matriks Elementer}
\begin{definisi}
Sebuah \textbf{matriks elementer} $E$ adalah matriks yang diperoleh dengan melakukan \textit{satu} Operasi Baris Elementer pada matriks identitas $I_n$.
\end{definisi}

\begin{contoh}
Perhatikan matriks identitas $I_3 = \matriks{1 & 0 & 0 \\ 0 & 1 & 0 \\ 0 & 0 & 1}$.
\begin{itemize}
    \item Menukar $R_1 \leftrightarrow R_2$: $E_1 = \matriks{0 & 1 & 0 \\ 1 & 0 & 0 \\ 0 & 0 & 1}$.
    \item Mengalikan $R_2$ dengan $k \neq 0$: $E_2 = \matriks{1 & 0 & 0 \\ 0 & k & 0 \\ 0 & 0 & 1}$.
    \item Menjumlahkan $kR_1$ ke $R_2$ ($R_2+kR_1 \to R_2$): $E_3 = \matriks{1 & 0 & 0 \\ k & 1 & 0 \\ 0 & 0 & 1}$.
\end{itemize}
\end{contoh}

\subsection{Matriks Elementer dan OBE}
\begin{fakta}
Misalkan $A \in M_{m \times n}$ dan $I_m$ adalah matriks identitas $m \times m$. Jika sebuah Operasi Baris Elementer dilakukan pada $A$ untuk menghasilkan matriks $A'$, dan operasi yang sama dilakukan pada $I_m$ untuk menghasilkan matriks elementer $E$, maka $A' = EA$.
Artinya, melakukan OBE pada $A$ sama dengan mengalikan $A$ dari kiri dengan matriks elementer yang bersesuaian.
\end{fakta}

\begin{contoh}[Ilustrasi Fakta]
Misalkan $A = \matriks{a_{11} & a_{12} & a_{13} \\ a_{21} & a_{22} & a_{23} \\ a_{31} & a_{32} & a_{33}}$.
Operasi $R_2 + 2R_1 \to R_2$ pada $A$ menghasilkan:
\[ A' = \matriks{a_{11} & a_{12} & a_{13} \\ a_{21}+2a_{11} & a_{22}+2a_{12} & a_{23}+2a_{13} \\ a_{31} & a_{32} & a_{33}} \]
Matriks elementer $E$ yang bersesuaian dari $I_3$ dengan operasi $R_2+2R_1 \to R_2$ adalah $E = \matriks{1 & 0 & 0 \\ 2 & 1 & 0 \\ 0 & 0 & 1}$.
Maka $EA = \matriks{1 & 0 & 0 \\ 2 & 1 & 0 \\ 0 & 0 & 1} \matriks{a_{11} & a_{12} & a_{13} \\ a_{21} & a_{22} & a_{23} \\ a_{31} & a_{32} & a_{33}} = \matriks{a_{11} & a_{12} & a_{13} \\ 2a_{11}+a_{21} & 2a_{12}+a_{22} & 2a_{13}+a_{23} \\ a_{31} & a_{32} & a_{33}} = A'$.
\end{contoh}

\section{Rank Matriks}
\begin{definisi}
Misalkan $A$ suatu matriks berukuran $m \times n$. Jika $A'$ adalah bentuk eselon baris dari $A$, maka banyaknya baris tak nol pada $A'$ disebut \textbf{rank} dari matriks $A$, dinotasikan $\rank(A)$.
\end{definisi}
Rank matriks juga sama dengan jumlah 1 utama dalam bentuk eselon barisnya (atau bentuk eselon baris tereduksinya).
Rank matriks adalah properti fundamental yang mengindikasikan "dimensi" dari ruang yang direntang oleh baris-barisnya (atau kolom-kolomnya).

\begin{contoh}
Tentukan rank dari matriks-matriks berikut:
\[ A = \matriks{1 & 2 \\ -2 & 1}, \quad B = \matriks{1 & 0 & 1 \\ -2 & 1 & 0 \\ 0 & 1 & 3} \]
\textbf{Jawaban untuk A}:
\[ \matriks{1 & 2 \\ -2 & 1} \xrightarrow{R_2 \to R_2 + 2R_1} \matriks{1 & 2 \\ 0 & 5} \xrightarrow{R_2 \to \frac{1}{5}R_2} \matriks{1 & 2 \\ 0 & 1} \]
Bentuk eselon baris memiliki 2 baris tak nol. Jadi, $\rank(A)=2$.
\textbf{Jawaban untuk B}:
\[ \matriks{1 & 0 & 1 \\ -2 & 1 & 0 \\ 0 & 1 & 3} \xrightarrow{R_2 \to R_2 + 2R_1} \matriks{1 & 0 & 1 \\ 0 & 1 & 2 \\ 0 & 1 & 3} \xrightarrow{R_3 \to R_3 - R_2} \matriks{1 & 0 & 1 \\ 0 & 1 & 2 \\ 0 & 0 & 1} \]
Bentuk eselon baris memiliki 3 baris tak nol. Jadi, $\rank(B)=3$.
\end{contoh}

\end{document}
