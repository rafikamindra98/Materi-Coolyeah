\documentclass{article}
\usepackage{amsmath} % Untuk lingkungan matriks dan perintah matematika lainnya
\usepackage{amsfonts} % Untuk simbol
\usepackage{amssymb} % Untuk simbol \mathbb{R}
\usepackage{geometry} % Untuk mengatur margin
\geometry{a4paper, margin=1in}

\title{Aljabar Linear: Komposisi Isomorfisma}
\author{Rafi Kamindra 2201006}
\date{} % Kosongkan tanggal jika tidak ingin ditampilkan

\newcommand{\vektor}[1]{\mathbf{#1}} % Perintah untuk vektor
\newcommand{\R}{\mathbb{R}} % Simbol R
\newcommand{\Poly}{\mathbb{P}} % Simbol P untuk polinom
\newcommand{\M}{\mathbb{M}} % Simbol M untuk matriks
\newcommand{\Transformasi}[1]{T(#1)} % Notasi Transformasi
\newcommand{\kernel}{\text{ker}} % Kernel
\newcommand{\jangkauan}{\text{R}} % Jangkauan (Range)
\newcommand{\nulitas}{\text{nulitas}} % Nulitas
\newcommand{\rank}{\text{rank}} % Rank
\newcommand{\matriks}[1]{\begin{pmatrix} #1 \end{pmatrix}} % Perintah untuk matriks
\newcommand{\spanof}[1]{\text{span}\{#1\}}

\begin{document}
\maketitle
\pagenumbering{gobble} % Menghilangkan nomor halaman jika tidak diinginkan untuk halaman judul

\section*{Problem (Injektif)}
\textbf{Periksa, apakah pemetaan berikut injektif.}

\subsection*{a. $T: \R^3 \rightarrow \R^2$ dengan $T\matriks{x \\ y \\ z} = \matriks{2x+y-z \\ x-2y+2z}$.}
\textbf{Jawaban:}\\
Transformasi $T$ injektif jika $\kernel(T) = \{\vektor{0}\}$.
Matriks standar $A = \matriks{2 & 1 & -1 \\ 1 & -2 & 2}$.
Selesaikan $A\vektor{v} = \vektor{0}$:
$\left[ \begin{array}{ccc|c} 2 & 1 & -1 & 0 \\ 1 & -2 & 2 & 0 \end{array} \right]$
$R_1 \leftrightarrow R_2$: $\left[ \begin{array}{ccc|c} 1 & -2 & 2 & 0 \\ 2 & 1 & -1 & 0 \end{array} \right]$
$R_2 \rightarrow R_2 - 2R_1$: $\left[ \begin{array}{ccc|c} 1 & -2 & 2 & 0 \\ 0 & 5 & -5 & 0 \end{array} \right]$
$R_2 \rightarrow \frac{1}{5}R_2$: $\left[ \begin{array}{ccc|c} 1 & -2 & 2 & 0 \\ 0 & 1 & -1 & 0 \end{array} \right]$
$R_1 \rightarrow R_1 + 2R_2$: $\left[ \begin{array}{ccc|c} 1 & 0 & 0 & 0 \\ 0 & 1 & -1 & 0 \end{array} \right]$.
Dari sini, $x=0$ dan $y-z=0 \Rightarrow y=z$.
Misalkan $z=t$, maka $y=t$, $x=0$.
Solusi adalah $(0,t,t) = t(0,1,1)$.
Karena ada solusi non-trivial (misal $t=1$, $(0,1,1) \neq \vektor{0}$), maka $\kernel(T) \neq \{\vektor{0}\}$.
Jadi, $T$ tidak injektif.
(Alternatif: $\dim(\text{domain}) = 3$, $\dim(\text{kodomain}) = 2$. Karena $\dim(\text{domain}) > \dim(\text{kodomain})$, $T$ tidak mungkin injektif menurut teorema).

\subsection*{b. $T: \R^3 \rightarrow \R^3$ dengan $T\matriks{x \\ y \\ z} = \matriks{x+y-3z \\ -x-2y+z \\ 2x-y-z}$.}
\textbf{Jawaban:}\\
Matriks standar $A = \matriks{1 & 1 & -3 \\ -1 & -2 & 1 \\ 2 & -1 & -1}$.
Selesaikan $A\vektor{v} = \vektor{0}$:
$\left[ \begin{array}{ccc|c} 1 & 1 & -3 & 0 \\ -1 & -2 & 1 & 0 \\ 2 & -1 & -1 & 0 \end{array} \right]$
$R_2 \rightarrow R_2 + R_1$: $\left[ \begin{array}{ccc|c} 1 & 1 & -3 & 0 \\ 0 & -1 & -2 & 0 \\ 2 & -1 & -1 & 0 \end{array} \right]$
$R_3 \rightarrow R_3 - 2R_1$: $\left[ \begin{array}{ccc|c} 1 & 1 & -3 & 0 \\ 0 & -1 & -2 & 0 \\ 0 & -3 & 5 & 0 \end{array} \right]$
$R_2 \rightarrow -R_2$: $\left[ \begin{array}{ccc|c} 1 & 1 & -3 & 0 \\ 0 & 1 & 2 & 0 \\ 0 & -3 & 5 & 0 \end{array} \right]$
$R_3 \rightarrow R_3 + 3R_2$: $\left[ \begin{array}{ccc|c} 1 & 1 & -3 & 0 \\ 0 & 1 & 2 & 0 \\ 0 & 0 & 11 & 0 \end{array} \right]$.
Dari $11z=0 \Rightarrow z=0$.
Dari $y+2z=0 \Rightarrow y=0$.
Dari $x+y-3z=0 \Rightarrow x=0$.
Satu-satunya solusi adalah $(0,0,0)$. Jadi $\kernel(T) = \{\vektor{0}\}$.
Maka, $T$ injektif.

\subsection*{c. $T: \R^3 \rightarrow \R^4$ dengan $T\matriks{x \\ y \\ z} = \matriks{x+y-2z \\ 2x-2y-z \\ 3x-y-3z \\ x+y+z}$.}
\textbf{Jawaban:}\\
Matriks standar $A = \matriks{1 & 1 & -2 \\ 2 & -2 & -1 \\ 3 & -1 & -3 \\ 1 & 1 & 1}$.
Selesaikan $A\vektor{v} = \vektor{0}$:
$\left[ \begin{array}{ccc|c} 1 & 1 & -2 & 0 \\ 2 & -2 & -1 & 0 \\ 3 & -1 & -3 & 0 \\ 1 & 1 & 1 & 0 \end{array} \right]$
$R_2 \rightarrow R_2 - 2R_1$, $R_3 \rightarrow R_3 - 3R_1$, $R_4 \rightarrow R_4 - R_1$:
$\left[ \begin{array}{ccc|c} 1 & 1 & -2 & 0 \\ 0 & -4 & 3 & 0 \\ 0 & -4 & 3 & 0 \\ 0 & 0 & 3 & 0 \end{array} \right]$
$R_3 \rightarrow R_3 - R_2$:
$\left[ \begin{array}{ccc|c} 1 & 1 & -2 & 0 \\ 0 & -4 & 3 & 0 \\ 0 & 0 & 0 & 0 \\ 0 & 0 & 3 & 0 \end{array} \right]$
Tukar $R_3$ dan $R_4$:
$\left[ \begin{array}{ccc|c} 1 & 1 & -2 & 0 \\ 0 & -4 & 3 & 0 \\ 0 & 0 & 3 & 0 \\ 0 & 0 & 0 & 0 \end{array} \right]$.
Dari $3z=0 \Rightarrow z=0$.
Dari $-4y+3z=0 \Rightarrow -4y=0 \Rightarrow y=0$.
Dari $x+y-2z=0 \Rightarrow x=0$.
Satu-satunya solusi adalah $(0,0,0)$. Jadi $\kernel(T) = \{\vektor{0}\}$.
Maka, $T$ injektif.

\subsection*{d. $T: \Poly_2 \rightarrow \Poly_2$ dengan $T(a+bx+cx^2) = (a+b-c) + (2a-b+2c)x + (3a+c)x^2$.}
\textbf{Jawaban:}\\
Matriks standar terhadap basis $\{1,x,x^2\}$:
$T(1) = 1+2x+3x^2 \rightarrow (1,2,3)$
$T(x) = 1-x \rightarrow (1,-1,0)$
$T(x^2) = -1+2x+x^2 \rightarrow (-1,2,1)$
$A = \matriks{1 & 1 & -1 \\ 2 & -1 & 2 \\ 3 & 0 & 1}$.
$\det(A) = 1(-1-0) - 1(2-6) - 1(0-(-3)) = -1 - (-4) - 3 = -1+4-3 = 0$.
Karena $\det(A)=0$, matriks $A$ singular, sehingga $A\vektor{v}=\vektor{0}$ memiliki solusi non-trivial.
Maka $\kernel(T) \neq \{\vektor{0}\}$.
Jadi, $T$ tidak injektif.

\section*{Problem (Surjektif)}
\textbf{Periksa, apakah pemetaan berikut surjektif.}

\subsection*{a. $T: \R^3 \rightarrow \R^2$ dengan $T\matriks{x \\ y \\ z} = \matriks{x-y-z \\ x+2y+3z}$.}
\textbf{Jawaban:}\\
Matriks standar $A = \matriks{1 & -1 & -1 \\ 1 & 2 & 3}$.
Rank $A$ paling banyak adalah $\min(3,2)=2$.
$\left[ \begin{array}{ccc} 1 & -1 & -1 \\ 1 & 2 & 3 \end{array} \right]$
$R_2 \rightarrow R_2 - R_1$: $\left[ \begin{array}{ccc} 1 & -1 & -1 \\ 0 & 3 & 4 \end{array} \right]$.
Rank $A = 2$.
Karena $\rank(T) = \dim(R(T)) = 2$, dan $\dim(\text{kodomain}) = \dim(\R^2)=2$.
Karena $\rank(T) = \dim(\text{kodomain})$, maka $T$ surjektif.

\subsection*{b. $T: \R^3 \rightarrow \R^3$ dengan $T\matriks{x \\ y \\ z} = \matriks{x+y-2z \\ 2x-2y-z \\ 3x-y-3z}$.}
\textbf{Jawaban:}\\
Matriks standar $A = \matriks{1 & 1 & -2 \\ 2 & -2 & -1 \\ 3 & -1 & -3}$.
$\det(A) = 1(6-1) - 1(-6-(-3)) - 2(-2-(-6)) = 1(5) - 1(-3) - 2(4) = 5+3-8 = 0$.
Karena $\det(A)=0$, maka $\rank(A) < 3$.
Karena $\rank(T) < \dim(\text{kodomain})$, maka $T$ tidak surjektif.

\subsection*{c. $T: \R^3 \rightarrow \R^4$ dengan $T\matriks{x \\ y \\ z} = \matriks{x+y-z \\ -x-2y-z \\ 3x-y-3z \\ x+3y+2z}$.}
\textbf{Jawaban:}\\
$\dim(\text{domain}) = 3$, $\dim(\text{kodomain}) = 4$.
$\rank(T) \le \min(\dim(\text{domain}), \dim(\text{kodomain})) = \min(3,4) = 3$.
Karena $\rank(T) \le 3$ dan $\dim(\text{kodomain}) = 4$, maka $\rank(T) < \dim(\text{kodomain})$.
Jadi, $T$ tidak surjektif.

\subsection*{d. $T: \Poly_2 \rightarrow \Poly_2$ linear dengan $T(2)=x+2x^2, T(1+x)=1-x-x^2, T(1-x^2)=1+x^2$.}
\textbf{Jawaban:}\\
Basis standar untuk $\Poly_2$ adalah $B=\{1,x,x^2\}$.
$T(2) = 0 \cdot 1 + 1 \cdot x + 2 \cdot x^2 \rightarrow (0,1,2)$. Karena $T$ linear, $2T(1) = (0,1,2) \Rightarrow T(1) = (0, 1/2, 1)$.
$T(1+x) = T(1)+T(x) = (1,-1,-1)$.
Maka $T(x) = T(1+x) - T(1) = (1,-1,-1) - (0,1/2,1) = (1, -3/2, -2)$.
$T(1-x^2) = T(1)-T(x^2) = (1,0,1)$.
Maka $T(x^2) = T(1) - T(1-x^2) = (0,1/2,1) - (1,0,1) = (-1,1/2,0)$.
Matriks standar $A = \matriks{T(1) & T(x) & T(x^2)} = \matriks{0 & 1 & -1 \\ 1/2 & -3/2 & 1/2 \\ 1 & -2 & 0}$.
$\det(A) = 0 - 1(0 - (-1/2)) - 1(-1 - (-3/2)) = -1/2 - 1(1/2) = -1/2 - 1/2 = -1$.
Karena $\det(A) = -1 \neq 0$, maka $\rank(A)=3$.
Karena $\rank(T) = \dim(\text{kodomain}) = 3$, maka $T$ surjektif.

\section*{Problem (Lanjutan)}

\subsection*{1. Diketahui transformasi $T: \R^3 \rightarrow \R^3$, dengan $T(x,y,z)=(ax-y, x+ay+3z, -ax+y)$. Jika nulitas $T$ paling kecil 1, berapa nilai $a$ yang mungkin?}
\textbf{Jawaban:}\\
Nulitas $T \ge 1$ berarti $T$ tidak injektif, yang berarti $\kernel(T) \neq \{\vektor{0}\}$.
Ini terjadi jika matriks standar $A$ dari $T$ adalah singular, yaitu $\det(A)=0$.
$A = \matriks{a & -1 & 0 \\ 1 & a & 3 \\ -a & 1 & 0}$.
$\det(A) = a(0-3) - (-1)(0 - (-3a)) + 0 = a(-3) + 1(-3a) = -3a - 3a = -6a$.
Agar nulitas $T \ge 1$, maka $\det(A)=0$.
$-6a = 0 \Rightarrow a=0$.
Jadi, nilai $a$ yang mungkin adalah $a=0$.

\subsection*{2. Diketahui transformasi linear $T: \R^3 \rightarrow \R^3$, dengan $T(1,1,1)=(1,0,1)$, $T(1,1,0)=(0,2,1)$, $T(1,0,0)=(1,0,-1)$. Tentukan rumus $T(a,b,c)$.}
\textbf{Jawaban:}\\
Misalkan $\vektor{v}_1=(1,1,1), \vektor{v}_2=(1,1,0), \vektor{v}_3=(1,0,0)$. Himpunan ini adalah basis.
Kita ingin menyatakan $(a,b,c)$ sebagai kombinasi linear $k_1\vektor{v}_1+k_2\vektor{v}_2+k_3\vektor{v}_3$.
$(a,b,c) = k_1(1,1,1) + k_2(1,1,0) + k_3(1,0,0) = (k_1+k_2+k_3, k_1+k_2, k_1)$.
$k_1 = c$.
$k_1+k_2 = b \Rightarrow c+k_2=b \Rightarrow k_2 = b-c$.
$k_1+k_2+k_3 = a \Rightarrow c+(b-c)+k_3=a \Rightarrow b+k_3=a \Rightarrow k_3 = a-b$.
$T(a,b,c) = k_1 T(\vektor{v}_1) + k_2 T(\vektor{v}_2) + k_3 T(\vektor{v}_3)$
$= c(1,0,1) + (b-c)(0,2,1) + (a-b)(1,0,-1)$
$= (c + 0 + a-b, 0 + 2(b-c) + 0, c + (b-c) - (a-b))$
$= (a-b+c, 2b-2c, c+b-c-a+b) = (a-b+c, 2b-2c, -a+2b)$.
Jadi, $T(a,b,c) = (a-b+c, 2b-2c, -a+2b)$.

\subsection*{3. Carilah basis untuk kernel dan jangkauan dari $T: \Poly_2 \rightarrow \Poly_2$ dengan $T[a+bx+cx^2] = (a-b) + (a+b+c)x + (b-c)x^2$.}
\textbf{Jawaban:}\\
Matriks standar $A$:
$T(1) = 1+x \rightarrow (1,1,0)$
$T(x) = -1+x+x^2 \rightarrow (-1,1,1)$
$T(x^2) = c x - c x^2 \rightarrow (0,c,-c)$ (Ada kesalahan di soal, seharusnya $T(x^2) = 0+cx-cx^2$ atau $T(x^2)=0+x-x^2$ jika $c=1$ di $(b-c)x^2$ dan $a+b+c$ dari $c$ di $cx^2$).
Mengacu pada $T[a+bx+cx^2] = (a-b) + (a+b+c)x + (b-c)x^2$:
$T(1) = (1-0) + (1+0+0)x + (0-0)x^2 = 1+x \rightarrow (1,1,0)$.
$T(x) = (0-1) + (0+1+0)x + (1-0)x^2 = -1+x+x^2 \rightarrow (-1,1,1)$.
$T(x^2) = (0-0) + (0+0+1)x + (0-1)x^2 = cx - x^2 \rightarrow (0,1,-1)$. (Mengasumsikan $c$ di $cx^2$ adalah koefisien $x^2$)
$A = \matriks{1 & -1 & 0 \\ 1 & 1 & 1 \\ 0 & 1 & -1}$.
$\det(A) = 1(-1-1) - (-1)(-1-0) + 0 = -2 - 1(-1) = -2+1 = -1 \neq 0$.
Karena $\det(A) \neq 0$, maka $A$ non-singular.
$\kernel(T) = \{\vektor{0}\}$. Basis $\kernel(T) = \emptyset$. $\nulitas(T)=0$.
$\rank(T) = \dim(\Poly_2) - \nulitas(T) = 3-0=3$.
$R(T) = \Poly_2$. Basis $R(T) = \{1,x,x^2\}$.

\subsection*{4. Diketahui transformasi $T: \Poly_2 \rightarrow \Poly_2$, dengan $T(p(x)) = p'(x)$. Berikan matriks transformasi terhadap basis baku $B=\{1,x,x^2\}$.}
\textbf{Jawaban:}\\
$T(1) = (1)' = 0 = 0 \cdot 1 + 0 \cdot x + 0 \cdot x^2 \rightarrow (0,0,0)$.
$T(x) = (x)' = 1 = 1 \cdot 1 + 0 \cdot x + 0 \cdot x^2 \rightarrow (1,0,0)$.
$T(x^2) = (x^2)' = 2x = 0 \cdot 1 + 2 \cdot x + 0 \cdot x^2 \rightarrow (0,2,0)$.
Matriks transformasi $A = \matriks{0 & 1 & 0 \\ 0 & 0 & 2 \\ 0 & 0 & 0}$.

\subsection*{5. Diketahui transformasi linear $T: \R^3 \rightarrow \R^3$, dengan $T(x,y,z) = (x-y, y-z, z-x)$. Tentukan matriks transformasi $T$ terhadap basis $B=\{(1,0,0), (1,1,0), (1,1,1)\}$.}
\textbf{Jawaban:}\\
Misalkan $\vektor{b}_1=(1,0,0), \vektor{b}_2=(1,1,0), \vektor{b}_3=(1,1,1)$.
$T(\vektor{b}_1) = T(1,0,0) = (1-0, 0-0, 0-1) = (1,0,-1)$.
$T(\vektor{b}_2) = T(1,1,0) = (1-1, 1-0, 0-1) = (0,1,-1)$.
$T(\vektor{b}_3) = T(1,1,1) = (1-1, 1-1, 1-1) = (0,0,0)$.
Sekarang nyatakan $T(\vektor{b}_i)$ dalam basis $B$.
$P_B = \matriks{1 & 1 & 1 \\ 0 & 1 & 1 \\ 0 & 0 & 1}$. $P_B^{-1} = \matriks{1 & -1 & 0 \\ 0 & 1 & -1 \\ 0 & 0 & 1}$.
$[T(\vektor{b}_1)]_B = P_B^{-1} (1,0,-1)^T = \matriks{1 & -1 & 0 \\ 0 & 1 & -1 \\ 0 & 0 & 1} \matriks{1 \\ 0 \\ -1} = \matriks{1 \\ 1 \\ -1}$.
$[T(\vektor{b}_2)]_B = P_B^{-1} (0,1,-1)^T = \matriks{1 & -1 & 0 \\ 0 & 1 & -1 \\ 0 & 0 & 1} \matriks{0 \\ 1 \\ -1} = \matriks{-1 \\ 2 \\ -1}$.
$[T(\vektor{b}_3)]_B = P_B^{-1} (0,0,0)^T = \matriks{0 \\ 0 \\ 0}$.
Matriks $[T]_{B,B} = \matriks{1 & -1 & 0 \\ 1 & 2 & 0 \\ -1 & -1 & 0}$.

\subsection*{6. Diketahui transformasi linear $T: \Poly_2 \rightarrow \Poly_2$ dengan $T[a+bx+cx^2] = (3a+2b) + (2a+b-c)x + (a+b+c)x^2$. Periksa, apakah $T$ suatu isomorfisma.}
\textbf{Jawaban:}\\
$T$ adalah isomorfisma jika $T$ injektif dan surjektif. Ini ekuivalen dengan $\kernel(T)=\{\vektor{0}\}$ atau $\rank(T) = \dim(\Poly_2)=3$.
Matriks standar $A$:
$T(1) = 3+2x+x^2 \rightarrow (3,2,1)$.
$T(x) = 2+x+x^2 \rightarrow (2,1,1)$.
$T(x^2) = -x+x^2 \rightarrow (0,-1,1)$.
$A = \matriks{3 & 2 & 0 \\ 2 & 1 & -1 \\ 1 & 1 & 1}$.
$\det(A) = 3(1-(-1)) - 2(2-(-1)) + 0 = 3(2) - 2(3) = 6-6=0$.
Karena $\det(A)=0$, $T$ tidak injektif (nulitas $>0$) dan tidak surjektif (rank $<3$).
Jadi, $T$ bukan isomorfisma.

\subsection*{7. Diketahui transformasi $T: \Poly_3 \rightarrow \Poly_2$, dengan $T(1)=1, T(1+x)=1+x, T(x+x^2)=ax+x^2, T(x^3)=2+x$. Jika $T$ suatu pemetaan pada (surjektif), berapa nilai $a$?}
\textbf{Jawaban:}\\
$T(1)=(1,0,0)$.
$T(x) = T(1+x)-T(1) = (1,1,0)-(1,0,0) = (0,1,0)$.
$T(x^2) = T(x+x^2)-T(x) = (0,a,1)-(0,1,0) = (0, a-1, 1)$.
$T(x^3)=(2,1,0)$.
Jangkauan $R(T)$ direntang oleh $\{T(1), T(x), T(x^2), T(x^3)\}$.
$R(T) = \spanof{(1,0,0), (0,1,0), (0,a-1,1), (2,1,0)}$.
Agar $T$ surjektif ke $\Poly_2$, $\dim(R(T))$ harus 3.
Bentuk matriks dari vektor-vektor ini:
$\matriks{1 & 0 & 0 & 2 \\ 0 & 1 & a-1 & 1 \\ 0 & 0 & 1 & 0}$.
Matriks ini sudah dalam bentuk eselon (jika kolom ke-4 diabaikan untuk rank 3).
Agar ranknya 3, tiga baris ini harus linear independen. Ini sudah jelas dari bentuknya.
Vektor $(1,0,0), (0,1,0), (0,a-1,1)$ harus merentang $\R^3$.
Determinan $\begin{vmatrix} 1 & 0 & 0 \\ 0 & 1 & a-1 \\ 0 & 0 & 1 \end{vmatrix} = 1(1-0) = 1 \neq 0$.
Ini berarti $T(1), T(x), T(x^2)$ selalu linear independen, tidak peduli nilai $a$.
Maka $\dim(R(T))$ selalu 3. Jadi $T$ selalu surjektif untuk semua $a \in \R$.

\subsection*{8. Diketahui transformasi linear $T: \R^2 \rightarrow \R^3$, dengan $T(1,2)=(1,a,3)$, $T(2,1)=(a,1,6)$. Berapa nilai $a$ supaya $T$ satu-satu (injektif)?}
\textbf{Jawaban:}\\
$T$ injektif jika $\kernel(T)=\{\vektor{0}\}$.
Basis domain $\{(1,2), (2,1)\}$.
$T$ injektif jika $\{T(1,2), T(2,1)\}$ linear independen di $\R^3$.
Vektor $(1,a,3)$ dan $(a,1,6)$ linear independen jika salah satunya bukan kelipatan skalar dari yang lain.
Jika mereka dependen, maka $(1,a,3) = k(a,1,6)$ untuk suatu $k$.
$1=ka$, $a=k$, $3=6k \Rightarrow k=1/2$.
Jika $k=1/2$, maka $a=1/2$.
Dan $1 = (1/2)a \Rightarrow 1 = (1/2)(1/2) = 1/4$, yang salah.
Jadi, mereka tidak mungkin dependen kecuali jika salah satunya vektor nol (tidak mungkin di sini).
Agar $T$ injektif, rank matriks standar harus 2.
Matriks standar $A = \matriks{T(1,0) & T(0,1)}$.
$(1,0) = c_1(1,2)+c_2(2,1) = (c_1+2c_2, 2c_1+c_2)$.
$c_1+2c_2=1, 2c_1+c_2=0 \Rightarrow c_2=-2c_1$.
$c_1+2(-2c_1)=1 \Rightarrow c_1-4c_1=1 \Rightarrow -3c_1=1 \Rightarrow c_1=-1/3$. $c_2=2/3$.
$T(1,0) = -\frac{1}{3}(1,a,3) + \frac{2}{3}(a,1,6) = (-\frac{1}{3}+\frac{2a}{3}, -\frac{a}{3}+\frac{2}{3}, -1+4) = (\frac{2a-1}{3}, \frac{2-a}{3}, 3)$.
$(0,1) = d_1(1,2)+d_2(2,1) = (d_1+2d_2, 2d_1+d_2)$.
$d_1+2d_2=0 \Rightarrow d_1=-2d_2$. $2d_1+d_2=1 \Rightarrow 2(-2d_2)+d_2=1 \Rightarrow -3d_2=1 \Rightarrow d_2=-1/3$. $d_1=2/3$.
$T(0,1) = \frac{2}{3}(1,a,3) - \frac{1}{3}(a,1,6) = (\frac{2}{3}-\frac{a}{3}, \frac{2a}{3}-\frac{1}{3}, 2-2) = (\frac{2-a}{3}, \frac{2a-1}{3}, 0)$.
$A = \frac{1}{3} \matriks{2a-1 & 2-a \\ 2-a & 2a-1 \\ 9 & 0}$.
Agar $T$ injektif, kolom-kolom $A$ harus linear independen. Ini berarti rank $A=2$.
Ini juga berarti $\{T(1,2), T(2,1)\}$ harus linear independen.
Vektor $(1,a,3)$ dan $(a,1,6)$ linear independen jika tidak ada $k$ sehingga $(1,a,3)=k(a,1,6)$.
$1=ka, a=k, 3=6k \Rightarrow k=1/2$.
Jika $k=1/2$, maka $a=1/2$. Substitusi ke $1=ka$: $1=(1/2)(1/2)=1/4$, salah.
Jadi $(1,a,3)$ dan $(a,1,6)$ selalu linear independen kecuali jika salah satunya adalah kelipatan yang lain.
Mereka dependen jika $\frac{1}{a} = \frac{a}{1} = \frac{3}{6}$.
$\frac{3}{6} = 1/2$.
$\frac{a}{1} = 1/2 \Rightarrow a=1/2$.
$\frac{1}{a} = 1/2 \Rightarrow a=2$.
Karena $a$ tidak bisa $1/2$ dan $2$ sekaligus, maka kedua vektor tersebut selalu linear independen untuk semua $a \in \R$.
Jadi, $T$ selalu injektif untuk semua $a \in \R$.

\subsection*{9. Diketahui transformasi linear $T: \Poly_2 \rightarrow \Poly_2$ dengan $T(1)=2+x, T(1+x)=1+x+x^2, T(1+x+x^2)=-1+x^2$. Carilah $p \in R(T)$ sehingga $\Poly_2 = \spanof{1, x+x^2} \oplus \spanof{p}$.}
\textbf{Jawaban:}\\
$T(1)=(2,1,0)$.
$T(x) = T(1+x)-T(1) = (1,1,1)-(2,1,0) = (-1,0,1)$.
$T(x^2) = T(1+x+x^2) - T(1+x) = (-1,0,1) - (1,1,1) = (-2,-1,0)$.
Matriks $A = \matriks{2 & -1 & -2 \\ 1 & 0 & -1 \\ 0 & 1 & 0}$.
$\det(A) = 2(0-(-1)) - (-1)(0-0) - 2(1-0) = 2(1) - 0 - 2(1) = 2-2=0$.
Rank $A < 3$. $R(T) \neq \Poly_2$.
$\spanof{1, x+x^2}$ adalah subruang $W_1$ yang direntang oleh $(1,0,0)$ dan $(0,1,1)$.
Kita perlu $p \in R(T)$ sehingga $\Poly_2 = W_1 \oplus \spanof{p}$. Ini berarti $p \notin W_1$ dan $W_1 + \spanof{p} = \Poly_2$. Dimensi $W_1$ adalah 2. $\dim(\spanof{p})=1$. $\dim(W_1 \oplus \spanof{p})=3$.
$R(T)$ direntang oleh $\{(2,1,0), (-1,0,1), (-2,-1,0)\}$.
Vektor $(-2,-1,0) = -1 \cdot (2,1,0)$. Jadi $R(T) = \spanof{(2,1,0), (-1,0,1)}$. $\dim(R(T))=2$.
Kita perlu $p \in R(T)$ sehingga $p$ tidak di $W_1 = \spanof{(1,0,0), (0,1,1)}$.
Misalkan $p = (2,1,0) \in R(T)$. Apakah $(2,1,0) \in W_1$?
$(2,1,0) = k_1(1,0,0) + k_2(0,1,1) = (k_1, k_2, k_2)$.
$k_1=2, k_2=1, k_2=0$. Kontradiksi. Jadi $(2,1,0) \notin W_1$.
Maka $p = 2+x$ adalah pilihan yang valid.
Basis untuk $\Poly_2$ adalah $\{(1,0,0), (0,1,1), (2,1,0)\}$.
$\begin{vmatrix} 1 & 0 & 2 \\ 0 & 1 & 1 \\ 0 & 1 & 0 \end{vmatrix} = 1(0-1) = -1 \neq 0$.
Jadi $p=2+x$ memenuhi.

\subsection*{10. Diketahui transformasi $T, S: \R^3 \rightarrow \R^3$, dengan $S\matriks{x\\y\\z} = \matriks{ax-y+z \\ ay-z \\ -x+z}$, $T\matriks{x\\y\\z} = \matriks{x-y-z \\ 2x+y-z \\ 3x-2z}$, $a \in \R$. Berapa rank $ST$ yang mungkin?}
\textbf{Jawaban:}\\
Matriks untuk $S$: $M_S = \matriks{a & -1 & 1 \\ 0 & a & -1 \\ -1 & 0 & 1}$.
Matriks untuk $T$: $M_T = \matriks{1 & -1 & -1 \\ 2 & 1 & -1 \\ 3 & 0 & -2}$.
$\det(M_T) = 1(-2-0) - (-1)(-4-(-3)) -1(0-3) = -2 - (-1)(-1) - (-3) = -2 - 1 + 3 = 0$.
Rank $M_T < 3$.
$R_2 \rightarrow R_2 - 2R_1$, $R_3 \rightarrow R_3 - 3R_1$: $\matriks{1 & -1 & -1 \\ 0 & 3 & 1 \\ 0 & 3 & 1}$. $R_3 \rightarrow R_3-R_2$: $\matriks{1 & -1 & -1 \\ 0 & 3 & 1 \\ 0 & 0 & 0}$.
Rank $M_T = 2$.
$\det(M_S) = a(a-0) - (-1)(0-1) + 1(0-(-a)) = a^2 - 1(-1) + a = a^2+a+1$.
Diskriminan $D = 1^2 - 4(1)(1) = 1-4 = -3 < 0$. Karena koefisien $a^2$ positif dan $D<0$, maka $a^2+a+1 > 0$ untuk semua $a \in \R$.
Jadi $\det(M_S) \neq 0$, sehingga rank $M_S = 3$.
$\rank(ST) \le \min(\rank(S), \rank(T)) = \min(3,2) = 2$.
Juga, $\rank(S) + \rank(T) - n \le \rank(ST)$, dimana $n=3$.
$3+2-3 \le \rank(ST) \Rightarrow 2 \le \rank(ST)$.
Karena $2 \le \rank(ST) \le 2$, maka $\rank(ST)$ yang mungkin adalah 2.

\subsection*{11. Misalkan $B=\{\vektor{v}_1, \vektor{v}_2, \vektor{v}_3\}$ adalah suatu basis terurut untuk $V$. Asumsikan $T:V \rightarrow V$ linear, dimana $T(\vektor{v}_1)=\vektor{v}_2+\vektor{v}_3, T(\vektor{v}_2)=\vektor{v}_3, T(\vektor{v}_3)=\vektor{v}_1-\vektor{v}_2$.}
\subsubsection*{a. Carilah matriks representasi $[T]_{B,B}$.}
\textbf{Jawaban:}\\
$[T(\vektor{v}_1)]_B = [0\vektor{v}_1 + 1\vektor{v}_2 + 1\vektor{v}_3]_B = (0,1,1)^T$.
$[T(\vektor{v}_2)]_B = [0\vektor{v}_1 + 0\vektor{v}_2 + 1\vektor{v}_3]_B = (0,0,1)^T$.
$[T(\vektor{v}_3)]_B = [1\vektor{v}_1 - 1\vektor{v}_2 + 0\vektor{v}_3]_B = (1,-1,0)^T$.
$[T]_{B,B} = \matriks{0 & 0 & 1 \\ 1 & 0 & -1 \\ 1 & 1 & 0}$.

\subsubsection*{b. Jika $B'=\{\vektor{v}_2, \vektor{v}_3+\vektor{v}_1, \vektor{v}_1-\vektor{v}_2\}$ carilah $[T]_{B',B'}$.}
\textbf{Jawaban:}\\
Misalkan $\vektor{u}_1=\vektor{v}_2, \vektor{u}_2=\vektor{v}_3+\vektor{v}_1, \vektor{u}_3=\vektor{v}_1-\vektor{v}_2$.
Matriks transisi $P_{B' \rightarrow B} = \matriks{[\vektor{u}_1]_B & [\vektor{u}_2]_B & [\vektor{u}_3]_B}$.
$[\vektor{u}_1]_B = (0,1,0)^T$.
$[\vektor{u}_2]_B = (1,0,1)^T$.
$[\vektor{u}_3]_B = (1,-1,0)^T$.
$P_{B' \rightarrow B} = \matriks{0 & 1 & 1 \\ 1 & 0 & -1 \\ 0 & 1 & 0}$.
$\det(P_{B' \rightarrow B}) = 0 - 1(0-0) + 1(1-0) = 1 \neq 0$. Jadi $B'$ adalah basis.
$P_{B \rightarrow B'} = (P_{B' \rightarrow B})^{-1}$.
$\left[ \begin{array}{ccc|ccc} 0 & 1 & 1 & 1 & 0 & 0 \\ 1 & 0 & -1 & 0 & 1 & 0 \\ 0 & 1 & 0 & 0 & 0 & 1 \end{array} \right]$
$R_1 \leftrightarrow R_2$: $\left[ \begin{array}{ccc|ccc} 1 & 0 & -1 & 0 & 1 & 0 \\ 0 & 1 & 1 & 1 & 0 & 0 \\ 0 & 1 & 0 & 0 & 0 & 1 \end{array} \right]$
$R_3 \rightarrow R_3 - R_2$: $\left[ \begin{array}{ccc|ccc} 1 & 0 & -1 & 0 & 1 & 0 \\ 0 & 1 & 1 & 1 & 0 & 0 \\ 0 & 0 & -1 & -1 & 0 & 1 \end{array} \right]$
$R_3 \rightarrow -R_3$: $\left[ \begin{array}{ccc|ccc} 1 & 0 & -1 & 0 & 1 & 0 \\ 0 & 1 & 1 & 1 & 0 & 0 \\ 0 & 0 & 1 & 1 & 0 & -1 \end{array} \right]$
$R_1 \rightarrow R_1 + R_3$: $\left[ \begin{array}{ccc|ccc} 1 & 0 & 0 & 1 & 1 & -1 \\ 0 & 1 & 1 & 1 & 0 & 0 \\ 0 & 0 & 1 & 1 & 0 & -1 \end{array} \right]$
$R_2 \rightarrow R_2 - R_3$: $\left[ \begin{array}{ccc|ccc} 1 & 0 & 0 & 1 & 1 & -1 \\ 0 & 1 & 0 & 0 & 0 & 1 \\ 0 & 0 & 1 & 1 & 0 & -1 \end{array} \right]$.
$P_{B \rightarrow B'} = \matriks{1 & 1 & -1 \\ 0 & 0 & 1 \\ 1 & 0 & -1}$.
$[T]_{B',B'} = P_{B \rightarrow B'} [T]_{B,B} P_{B' \rightarrow B}$
$= \matriks{1 & 1 & -1 \\ 0 & 0 & 1 \\ 1 & 0 & -1} \matriks{0 & 0 & 1 \\ 1 & 0 & -1 \\ 1 & 1 & 0} \matriks{0 & 1 & 1 \\ 1 & 0 & -1 \\ 0 & 1 & 0}$
$= \matriks{1(0)+1(1)-1(1) & 1(0)+1(0)-1(1) & 1(1)+1(-1)-1(0) \\ 0(0)+0(1)+1(1) & 0(0)+0(0)+1(1) & 0(1)+0(-1)+1(0) \\ 1(0)+0(1)-1(1) & 1(0)+0(0)-1(1) & 1(1)+0(-1)-1(0)} \matriks{0 & 1 & 1 \\ 1 & 0 & -1 \\ 0 & 1 & 0}$
$= \matriks{0 & -1 & 0 \\ 1 & 1 & 0 \\ -1 & -1 & 1} \matriks{0 & 1 & 1 \\ 1 & 0 & -1 \\ 0 & 1 & 0}$
$= \matriks{0(0)-1(1)+0(0) & 0(1)-1(0)+0(1) & 0(1)-1(-1)+0(0) \\ 1(0)+1(1)+0(0) & 1(1)+1(0)+0(1) & 1(1)+1(-1)+0(0) \\ -1(0)-1(1)+1(0) & -1(1)-1(0)+1(1) & -1(1)-1(-1)+1(0)}$
$= \matriks{-1 & 0 & 1 \\ 1 & 1 & 0 \\ -1 & 0 & 0}$.

\subsection*{12. Diketahui pemetaan linear $T: \R^3 \rightarrow \R^3$ dengan matriks representasi $A = \matriks{1 & 0 & 1 \\ 1 & 1 & 0 \\ -1 & 2 & 1}$ terhadap basis $B=\{\vektor{b}_1=(-1,1,1), \vektor{b}_2=(1,0,-1), \vektor{b}_3=(0,1,1)\}$. Carilah $T((x,y,z))$.}
\textbf{Jawaban:}\\
Misalkan $\vektor{v}=(x,y,z)$. Kita perlu $[\vektor{v}]_B$.
Matriks $P_B = \matriks{-1 & 1 & 0 \\ 1 & 0 & 1 \\ 1 & -1 & 1}$.
$P_B^{-1}$: $\det(P_B) = -1(0-(-1)) - 1(1-1) + 0 = -1(1) = -1$.
$C_{11}=1, C_{12}=-(1-1)=0, C_{13}=-1$.
$C_{21}=- (1-0)=-1, C_{22}=-1, C_{23}=-(-1-1)=2$.
$C_{31}=1, C_{32}=-(-1-0)=1, C_{33}=-1$.
$\text{adj}(P_B) = \matriks{1 & -1 & 1 \\ 0 & -1 & 1 \\ -1 & 2 & -1}$.
$P_B^{-1} = \frac{1}{-1} \matriks{1 & -1 & 1 \\ 0 & -1 & 1 \\ -1 & 2 & -1} = \matriks{-1 & 1 & -1 \\ 0 & 1 & -1 \\ 1 & -2 & 1}$.
$[\vektor{v}]_B = P_B^{-1} \vektor{v} = \matriks{-1 & 1 & -1 \\ 0 & 1 & -1 \\ 1 & -2 & 1} \matriks{x \\ y \\ z} = \matriks{-x+y-z \\ y-z \\ x-2y+z}$.
$[T(\vektor{v})]_B = A [\vektor{v}]_B = \matriks{1 & 0 & 1 \\ 1 & 1 & 0 \\ -1 & 2 & 1} \matriks{-x+y-z \\ y-z \\ x-2y+z} = \matriks{(-x+y-z)+(x-2y+z) \\ (-x+y-z)+(y-z) \\ -(-x+y-z)+2(y-z)+(x-2y+z)}$.
$= \matriks{-y \\ -x+2y-2z \\ x-y+z+2y-2z+x-2y+z} = \matriks{-y \\ -x+2y-2z \\ 2x-y}$.
$T(\vektor{v}) = P_B [T(\vektor{v})]_B = \matriks{-1 & 1 & 0 \\ 1 & 0 & 1 \\ 1 & -1 & 1} \matriks{-y \\ -x+2y-2z \\ 2x-y}$
$= \matriks{-1(-y)+1(-x+2y-2z)+0 \\ 1(-y)+0+1(2x-y) \\ 1(-y)-1(-x+2y-2z)+1(2x-y)}$
$= \matriks{y-x+2y-2z \\ -y+2x-y \\ -y+x-2y+2z+2x-y} = \matriks{-x+3y-2z \\ 2x-2y \\ 3x-4y+2z}$.
Jadi, $T(x,y,z) = (-x+3y-2z, 2x-2y, 3x-4y+2z)$.

\subsection*{13. Diketahui pemetaan linear $T: \R^4 \rightarrow \R^3$ dengan matriks representasi terhadap basis baku $A = \matriks{1 & -2 & 0 & a \\ 3 & -6 & 1 & b \\ -2 & 4 & 1 & c}$ dengan $T(1,-1,-2,1)=(2,8,-2)$. Carilah $a,b,c$ dan basis untuk $R(T)$.}
\textbf{Jawaban:}\\
$T(1,-1,-2,1) = A \matriks{1 \\ -1 \\ -2 \\ 1} = \matriks{1 & -2 & 0 & a \\ 3 & -6 & 1 & b \\ -2 & 4 & 1 & c} \matriks{1 \\ -1 \\ -2 \\ 1} = \matriks{1(1)-2(-1)+0(-2)+a(1) \\ 3(1)-6(-1)+1(-2)+b(1) \\ -2(1)+4(-1)+1(-2)+c(1)} = \matriks{1+2+a \\ 3+6-2+b \\ -2-4-2+c} = \matriks{3+a \\ 7+b \\ -8+c}$.
Diberikan $T(1,-1,-2,1)=(2,8,-2)$.
$3+a=2 \Rightarrow a = -1$.
$7+b=8 \Rightarrow b = 1$.
$-8+c=-2 \Rightarrow c = 6$.
Jadi $A = \matriks{1 & -2 & 0 & -1 \\ 3 & -6 & 1 & 1 \\ -2 & 4 & 1 & 6}$.
Basis untuk $R(T)$ adalah basis ruang kolom $A$.
$R_2 \rightarrow R_2 - 3R_1$: $R_3 \rightarrow R_3 + 2R_1$:
$\matriks{1 & -2 & 0 & -1 \\ 0 & 0 & 1 & 1- (-3) \\ 0 & 0 & 1 & 6+ (-2)} = \matriks{1 & -2 & 0 & -1 \\ 0 & 0 & 1 & 4 \\ 0 & 0 & 1 & 4}$.
$R_3 \rightarrow R_3 - R_2$:
$\matriks{1 & -2 & 0 & -1 \\ 0 & 0 & 1 & 4 \\ 0 & 0 & 0 & 0}$.
Kolom pivot adalah kolom 1 dan 3.
Basis $R(T) = \left\{ \matriks{1 \\ 3 \\ -2}, \matriks{0 \\ 1 \\ 1} \right\}$.

\end{document}
